% ========================================
% MACROS POUR RÉSOLUTION AUTOMATIQUE DE TRINÔMES
% ========================================

\usepackage{siunitx} % Pour la notation avec virgule
\usepackage{xfp}

% Configuration siunitx pour affichage français
\sisetup{%
  output-decimal-marker={,}
}

% Commande personnalisée pour affichage avec virgule
\newcommand{\cperso}[1]{\num{\fpeval{#1}}}

% ========================================
% ACCOLADES GRANDES ET ENSEMBLE DISCRET
% ========================================

% Accolade gauche grande avec TikZ
\newcommand{\accoladeG}[1][1]{%
    \tikz[baseline=0.1ex,scale=#1]{
        \draw[line width=0.8pt,line cap=round]
            (0.15,0.45) .. controls (0.08,0.42) and (0.05,0.35) ..
            (0.05,0.25) .. controls (0.05,0.15) and (0.02,0.08) ..
            (0,0) .. controls (0.02,-0.08) and (0.05,-0.15) ..
            (0.05,-0.25) .. controls (0.05,-0.35) and (0.08,-0.42) ..
            (0.15,-0.45);
    }%
}

% Accolade droite grande avec TikZ
\newcommand{\accoladeD}[1][1]{%
    \tikz[baseline=0.1ex,scale=#1]{
        \draw[line width=0.8pt,line cap=round]
            (0,0.45) .. controls (0.07,0.42) and (0.1,0.35) ..
            (0.1,0.25) .. controls (0.1,0.15) and (0.13,0.08) ..
            (0.15,0) .. controls (0.13,-0.08) and (0.1,-0.15) ..
            (0.1,-0.25) .. controls (0.1,-0.35) and (0.07,-0.42) ..
            (0,-0.45);
    }%
}

% Ensemble discret avec grandes accolades à hauteur fixe
\newcommand{\ensembleDiscret}[2][1]{%
    \ifcase#1\relax
        \mathopen{\{}\,#2\,\mathclose{\}}%
    \or
        \mathopen{\{}\,#2\,\mathclose{\}}%
    \or
        \mathopen{\big\{}\,#2\,\mathclose{\big\}}%
    \or
        \mathopen{\Big\{}\,#2\,\mathclose{\Big\}}%
    \fi
}

% ========================================
% Variables globales pour les coefficients
\newcommand{\seta}[1]{\def\coeffa{#1}}
\newcommand{\setb}[1]{\def\coeffb{#1}}
\newcommand{\setc}[1]{\def\coeffc{#1}}

% Commandes pour afficher les coefficients avec signe
\newcommand{\affsignea}{\ifdim\coeffa pt<0pt-\fi}
\newcommand{\affsigneb}{\ifdim\coeffb pt<0pt-\fi}
\newcommand{\affsignec}{\ifdim\coeffc pt<0pt-\fi}

% Valeurs absolues
\newcommand{\absa}{\fpeval{abs(\coeffa)}}
\newcommand{\absb}{\fpeval{abs(\coeffb)}}
\newcommand{\absc}{\fpeval{abs(\coeffc)}}

% Affichage avec parenthèses si négatif (après un signe)
\newcommand{\parencoeffa}{\ifdim\coeffa pt<0pt(\coeffa)\else\coeffa\fi}
\newcommand{\parencoeffb}{\ifdim\coeffb pt<0pt(\coeffb)\else\coeffb\fi}
\newcommand{\parencoeffc}{\ifdim\coeffc pt<0pt(\coeffc)\else\coeffc\fi}

% Affichage direct (début de ligne ou sans multiplication)
\newcommand{\directcoeffa}{\coeffa}
\newcommand{\directcoeffb}{\coeffb}
\newcommand{\directcoeffc}{\coeffc}

% Calcul du discriminant
\newcommand{\calcdelta}{\fpeval{\coeffb*\coeffb - 4*\coeffa*\coeffc}}
\newcommand{\deltasimpl}{\cperso{round(\calcdelta,4)}}

% Affichage du calcul du discriminant avec gestion intelligente des signes
\newcommand{\affcalcdelta}{%
    \directcoeffb^2-4\times\parencoeffa\times\parencoeffc = %
    \cperso{\coeffb*\coeffb}%
    \ifdim\fpeval{4*\coeffa*\coeffc} pt<0pt
        +\cperso{abs(4*\coeffa*\coeffc)}
    \else
        -\cperso{4*\coeffa*\coeffc}
    \fi = %
    \deltasimpl
}

% Signe de delta
\newcommand{\signedelta}{%
    \ifnum\numexpr\fpeval{round(\calcdelta,0)}<0
        <
    \else
        \ifnum\numexpr\fpeval{round(\calcdelta,0)}>0
            >
        \else
            =
        \fi
    \fi
}

% Test si delta >= 0
\newcommand{\ifdeltapositif}[2]{%
    \ifnum\numexpr\fpeval{round(\calcdelta,0)}<0
        #2%
    \else
        #1%
    \fi
}

% Calcul de x0 (cas delta = 0)
\newcommand{\calcxzero}{\fpeval{round(-\coeffb/(2*\coeffa),4)}}
\newcommand{\calcxzerofrac}{\cperso{round(-\coeffb/(2*\coeffa),4)}}

% Affichage du calcul de x0
\newcommand{\affcalcxzero}{%
    -\dfrac{b}{2a} = -\dfrac{\coeffb}{2\times\parencoeffa} = %
    \ifdim\coeffb pt<0pt
        \dfrac{\absb}{2\times\parencoeffa}
    \else
        -\dfrac{\coeffb}{2\times\parencoeffa}
    \fi = %
    \ifdim\fpeval{-\coeffb} pt<0pt
        \dfrac{\cperso{abs(-\coeffb)}}{\cperso{2*\coeffa}}
    \else
        -\dfrac{\cperso{-\coeffb}}{\cperso{abs(2*\coeffa)}}
    \fi = %
    \calcxzerofrac
}

% Calcul des racines (cas delta > 0)
\newcommand{\calcxun}{\fpeval{round((-\coeffb - sqrt(\calcdelta))/(2*\coeffa),4)}}
\newcommand{\calcxdeux}{\fpeval{round((-\coeffb + sqrt(\calcdelta))/(2*\coeffa),4)}}

% Racine carrée de delta arrondie
\newcommand{\sqrtdelta}{\fpeval{round(sqrt(\calcdelta),4)}}
\newcommand{\sqrtdeltafrac}{\cperso{round(sqrt(\calcdelta),4)}}

% Affichage du calcul de x1
\newcommand{\affcalcxun}{%
    \dfrac{-b-\sqrt{\Delta}}{2a} = %
    \dfrac{%
        \ifdim\coeffb pt<0pt
            \absb
        \else
            -\coeffb
        \fi
        -\sqrt{\deltasimpl}%
    }{2\times\parencoeffa} = %
    \ifdim\issimplexun pt=1pt
        % Si simple, montrer les étapes de calcul
        \dfrac{%
            \ifdim\coeffb pt<0pt
                \absb
            \else
                -\coeffb
            \fi
            -\sqrtdeltafrac%
        }{\cperso{2*\coeffa}} = %
        \dfrac{\cperso{-\coeffb-\sqrtdelta}}{\cperso{2*\coeffa}} = %
        \xunaffichage
    \else
        % Si complexe, afficher directement la forme exacte simplifiée
        \xunaffichage
    \fi
}

% Affichage du calcul de x2
\newcommand{\affcalcxdeux}{%
    \dfrac{-b+\sqrt{\Delta}}{2a} = %
    \dfrac{%
        \ifdim\coeffb pt<0pt
            \absb
        \else
            -\coeffb
        \fi
        +\sqrt{\deltasimpl}%
    }{2\times\parencoeffa} = %
    \ifdim\issimplexdeux pt=1pt
        % Si simple, montrer les étapes de calcul
        \dfrac{%
            \ifdim\coeffb pt<0pt
                \absb
            \else
                -\coeffb
            \fi
            +\sqrtdeltafrac%
        }{\cperso{2*\coeffa}} = %
        \dfrac{\cperso{-\coeffb+\sqrtdelta}}{\cperso{2*\coeffa}} = %
        \xdeuxaffichage
    \else
        % Si complexe, afficher directement la forme exacte simplifiée
        \xdeuxaffichage
    \fi
}

% Simplification des racines si possibles (affichage fraction avec virgule)
\newcommand{\xunfrac}{%
    \cperso{abs(round(\calcxun,0) - \calcxun) < 0.001 ? round(\calcxun,0) : \calcxun}%
}

\newcommand{\xdeuxfrac}{%
    \cperso{abs(round(\calcxdeux,0) - \calcxdeux) < 0.001 ? round(\calcxdeux,0) : \calcxdeux}%
}

% Plus petite et plus grande racine (pour calculs)
\newcommand{\xpetit}{\fpeval{min(\calcxun,\calcxdeux)}}
\newcommand{\xgrand}{\fpeval{max(\calcxun,\calcxdeux)}}

% Plus petite et plus grande racine (pour affichage avec virgule)
\newcommand{\xpetitfrac}{\cperso{min(\calcxun,\calcxdeux)}}
\newcommand{\xgrandfrac}{\cperso{max(\calcxun,\calcxdeux)}}

% ========================================
% AFFICHAGE FORME EXACTE DES RACINES
% ========================================

% Test si une racine est "simple" (entière ou 1 décimale max)
\newcommand{\issimplexun}{%
    \fpeval{abs(round(\calcxun,1) - \calcxun) < 0.01 ? 1 : 0}%
}
\newcommand{\issimplexdeux}{%
    \fpeval{abs(round(\calcxdeux,1) - \calcxdeux) < 0.01 ? 1 : 0}%
}

% Forme exacte de x1 = (-b - √Δ)/(2a)
\newcommand{\xunexact}{%
    \ifdim\fpeval{2*\coeffa} pt=-1pt
        % Cas 2a = -1 : résultat = b + √Δ
        \ifdim\coeffb pt<0pt
            \fpeval{abs(\coeffb)} - \sqrt{\deltasimpl}%
        \else
            \coeffb + \sqrt{\deltasimpl}%
        \fi
    \else\ifdim\fpeval{2*\coeffa} pt=1pt
        % Cas 2a = 1 : résultat = -b - √Δ
        \ifdim\coeffb pt<0pt
            \fpeval{abs(\coeffb)} - \sqrt{\deltasimpl}%
        \else
            -\coeffb - \sqrt{\deltasimpl}%
        \fi
    \else
        % Cas général : fraction
        \dfrac{%
            \ifdim\coeffb pt<0pt
                \fpeval{abs(\coeffb)}%
            \else
                -\coeffb
            \fi
            - \sqrt{\deltasimpl}%
        }{\cperso{2*\coeffa}}%
    \fi\fi
}

% Forme exacte de x2 = (-b + √Δ)/(2a)
\newcommand{\xdeuxexact}{%
    \ifdim\fpeval{2*\coeffa} pt=-1pt
        % Cas 2a = -1 : résultat = b - √Δ
        \ifdim\coeffb pt<0pt
            \fpeval{abs(\coeffb)} + \sqrt{\deltasimpl}%
        \else
            \coeffb - \sqrt{\deltasimpl}%
        \fi
    \else\ifdim\fpeval{2*\coeffa} pt=1pt
        % Cas 2a = 1 : résultat = -b + √Δ
        \ifdim\coeffb pt<0pt
            \fpeval{abs(\coeffb)} + \sqrt{\deltasimpl}%
        \else
            -\coeffb + \sqrt{\deltasimpl}%
        \fi
    \else
        % Cas général : fraction
        \dfrac{%
            \ifdim\coeffb pt<0pt
                \fpeval{abs(\coeffb)}%
            \else
                -\coeffb
            \fi
            + \sqrt{\deltasimpl}%
        }{\cperso{2*\coeffa}}%
    \fi\fi
}

% Affichage intelligent de x1 (forme exacte si complexe, décimale si simple)
\newcommand{\xunaffichage}{%
    \ifdim\issimplexun pt=1pt
        \xunfrac
    \else
        \xunexact
    \fi
}

% Affichage intelligent de x2 (forme exacte si complexe, décimale si simple)
\newcommand{\xdeuxaffichage}{%
    \ifdim\issimplexdeux pt=1pt
        \xdeuxfrac
    \else
        \xdeuxexact
    \fi
}

% Affichage intelligent du petit (min)
\newcommand{\xpetitaffichage}{%
    \ifdim\calcxun pt<\calcxdeux pt
        \xunaffichage
    \else
        \xdeuxaffichage
    \fi
}

% Affichage intelligent du grand (max)
\newcommand{\xgrandaffichage}{%
    \ifdim\calcxun pt>\calcxdeux pt
        \xunaffichage
    \else
        \xdeuxaffichage
    \fi
}

% Signe de a (pour tableau de signe)
\newcommand{\signea}{%
    \ifdim\coeffa pt<0pt
        -
    \else
        +
    \fi
}

% Signe de -a
\newcommand{\signemoina}{%
    \ifdim\coeffa pt<0pt
        +
    \else
        -
    \fi
}

% ========================================
% COMMANDE PRINCIPALE : RÉSOLUTION ÉQUATION
% ========================================
\newcommand{\resoudreequation}[4]{%
    \seta{#1}\setb{#2}\setc{#3}%
    \begin{tcbenumerate}[2]
    \tcbitem \textbf{Identification des coefficients :}

    $a=\coeffa$, $b=\coeffb$ et $c=\coeffc$

    \tcbitem \textbf{Calcul du discriminant :}

    $\Delta = b^2-4ac = \affcalcdelta$

    \ifdeltapositif{%
        \ifnum\numexpr\fpeval{round(\calcdelta,0)}=0
            % Cas delta = 0
            \tcbitem[raster multicolumn=2]\textbf{Détermination de la racine :}   $\Delta = 0$ donc l'équation admet une solution double :

            $x_0 = \affcalcxzero$

            \tcbitem[raster multicolumn=2]\textbf{Ensemble des solutions :}

            $S = \ensembleDiscret{ \calcxzero }$
        \else
            % Cas delta > 0
            \tcbitem[raster multicolumn=2]\textbf{Détermination des racines :}
            $\Delta = \deltasimpl > 0$ donc l'équation admet deux solutions réelles distinctes :

            \begin{MultiColonnes}{2}[halign=center]
            \tcbitem $x_1 = \affcalcxun$
            \tcbitem $x_2 = \affcalcxdeux$
            \end{MultiColonnes}

            \tcbitem[raster multicolumn=2]\textbf{Ensemble des solutions :} \encadrer[red]{$S = \ensembleDiscret{ \xunaffichage ; \xdeuxaffichage }$}
        \fi
    }{%
        % Cas delta < 0
        \tcbitem[raster multicolumn=2]\textbf{Conclusion :}
        $\Delta = \deltasimpl < 0$ donc l'équation n'admet pas de solution réelle.

        $S = \emptyset$
    }%
    \end{tcbenumerate}
}

% ========================================
% COMMANDE PRINCIPALE : RÉSOLUTION INÉQUATION
% ========================================
\newcommand{\resoudreinequation}[4]{%
    % #1 = a, #2 = b, #3 = c, #4 = type (geq, leq, g, l)
    \seta{#1}\setb{#2}\setc{#3}%
    \def\typeineg{#4}%
    \begin{tcbenumerate}[2]
    \tcbitem \textbf{Identification des coefficients :}

    $a=\coeffa$, $b=\coeffb$ et $c=\coeffc$

    \tcbitem\textbf{Calcul du discriminant :}

    $\Delta = b^2-4ac = \affcalcdelta$

    \tcbitem[raster multicolumn=2]\textbf{Détermination des racines :}

    \ifdeltapositif{%
        \ifnum\numexpr\fpeval{round(\calcdelta,0)}=0
            $\Delta = 0$ donc le trinôme admet une racine double :

            $x_0 = \affcalcxzero$
        \else
            $\Delta = \deltasimpl > 0$ donc le trinôme admet deux racines réelles distinctes :

            \begin{MultiColonnes}{2}[halign=center,boxrule=0.4pt,colframe=black,colback=white]
            \tcbitem $x_1 = \affcalcxun$
            \tcbitem $x_2 = \affcalcxdeux$
            \end{MultiColonnes}
        \fi
    }{%
        $\Delta = \deltasimpl < 0$ donc pas de racine réelle.
    }%
        \end{tcbenumerate}
        \begin{MultiColonnes}{2}
            \tcbitem \begin{tcbenumerate}[2][4]\tcbitem[raster multicolumn=2]\textbf{Tableau de signe} de $f:x\mapsto ax^2+bx+c$ :

    On a $a=\coeffa$ \ifdim\coeffa pt<0pt{$<0$}\else{$>0$}\fi{} et %
    \ifdeltapositif{%
        \ifnum\numexpr\fpeval{round(\calcdelta,0)}=0
            $x_0 = \calcxzero$
        \else
            $x_1 = \xunaffichage$ et $x_2 = \xdeuxaffichage$ donc $\xpetitaffichage < \xgrandaffichage$
        \fi
    }{pas de racine}

    \ifdeltapositif{%
        \ifnum\numexpr\fpeval{round(\calcdelta,0)}=0
            % Delta = 0
            \begin{tikzpicture}
            \tkzTabInit[espcl=2,lgt=1.5]{$x$/0.8,$f(x)$/0.8}
            {$-\infty$,$\calcxzero$,$+\infty$}
            \tkzTabLine{,\signea,z,\signea}
            \end{tikzpicture}
        \else
            % Delta > 0
            \begin{tikzpicture}
            \tkzTabInit[espcl=2,lgt=1.5]{$x$/0.8,$f(x)$/0.8}
            {$-\infty$,$\xpetitaffichage$,$\xgrandaffichage$,$+\infty$}
            \tkzTabLine{,\signea,z,\signemoina,z,\signea}
            \end{tikzpicture}
        \fi
    }{%
        % Delta < 0
        \begin{tikzpicture}
        \tkzTabInit[espcl=3,lgt=1.5]{$x$/0.8,$f(x)$/0.8}
        {$-\infty$,$+\infty$}
        \tkzTabLine{,\signea}
        \end{tikzpicture}
    }%

    \tcbitem[raster multicolumn=2]\textbf{Solution de l'inéquation :}

    % Solution selon le type et le signe
    \def\tempgeq{geq}\def\templeq{leq}\def\tempg{g}\def\templ{l}%
    \ifx\typeineg\tempgeq
        On cherche où le trinôme est $\geq 0$, donc les signes $+$ et $0$ :
    \fi
    \ifx\typeineg\templeq
        On cherche où le trinôme est $\leq 0$, donc les signes $-$ et $0$ :
    \fi
    \ifx\typeineg\tempg
        On cherche où le trinôme est $> 0$, donc les signes $+$ (strictement) :
    \fi
    \ifx\typeineg\templ
        On cherche où le trinôme est $< 0$, donc les signes $-$ (strictement) :
    \fi

    % Déterminer la solution selon delta et type
    \ifdeltapositif{%
        \ifnum\numexpr\fpeval{round(\calcdelta,0)}=0
            % Delta = 0 : solution selon signe de a et type
            \ifx\typeineg\tempgeq
                \ifdim\coeffa pt>0pt
                    $S = \R$
                \else
                    $S = \ensembleDiscret{\calcxzerofrac \right\rbrace$}
                \fi
            \fi
            \ifx\typeineg\templeq
                \ifdim\coeffa pt<0pt
                    $S = \R$
                \else
                    $S = \ensembleDiscret{ \calcxzerofrac }$
                \fi
            \fi
            \ifx\typeineg\tempg
                \ifdim\coeffa pt>0pt
                    $S = \R \setminus \ensembleDiscret{\calcxzerofrac}$
                \else
                    $S = \emptyset$
                \fi
            \fi
            \ifx\typeineg\templ
                \ifdim\coeffa pt<0pt
                    $S = \R \setminus \ensembleDiscret{\calcxzerofrac}$
                \else
                    $S = \emptyset$
                \fi
            \fi
        \else
            % Delta > 0 : solution selon signe de a et type
            \ifx\typeineg\tempgeq
                \ifdim\coeffa pt>0pt
                    $S = \CrochetD-\infty;\xpetitaffichage\,\,\CrochetD \cup \CrochetG\xgrandaffichage;+\infty\,\,\CrochetG$
                \else
                    $S = \CrochetG\xpetitaffichage;\xgrandaffichage\,\,\CrochetD$
                \fi
            \fi
            \ifx\typeineg\templeq
                \ifdim\coeffa pt<0pt
                    $S = \CrochetD-\infty;\xpetitaffichage\CrochetD \cup \CrochetG\xgrandaffichage;+\infty\,\,\CrochetG$
                \else
                    $S = \CrochetG\xpetitaffichage;\xgrandaffichage\,\,\CrochetD$
                \fi
            \fi
            \ifx\typeineg\tempg
                \ifdim\coeffa pt>0pt
                    $S = \CrochetD-\infty;\xpetitaffichage\,\,\CrochetG \cup \CrochetD\xgrandaffichage;+\infty\,\,\CrochetG$
                \else
                    $S = \CrochetD\xpetitaffichage;\xgrandaffichage\,\,\CrochetG$
                \fi
            \fi
            \ifx\typeineg\templ
                \ifdim\coeffa pt<0pt
                    $S = \CrochetD-\infty;\xpetitaffichage\,\,\CrochetG \cup \CrochetD\xgrandaffichage;+\infty\,\,\CrochetG$
                \else
                    $S = \CrochetD\xpetitaffichage;\xgrandaffichage\,\,\CrochetG$
                \fi
            \fi
        \fi
    }{%
        % Delta < 0 : solution selon signe de a et type
        \ifx\typeineg\tempgeq
            \ifdim\coeffa pt>0pt
                $S = \R$
            \else
                $S = \emptyset$
            \fi
        \fi
        \ifx\typeineg\templeq
            \ifdim\coeffa pt<0pt
                $S = \R$
            \else
                $S = \emptyset$
            \fi
        \fi
        \ifx\typeineg\tempg
            \ifdim\coeffa pt>0pt
                $S = \R$
            \else
                $S = \emptyset$
            \fi
        \fi
        \ifx\typeineg\templ
            \ifdim\coeffa pt<0pt
                $S = \R$
            \else
                $S = \emptyset$
            \fi
        \fi
    }%
        \end{tcbenumerate}
            \tcbitem[halign=center,valign=center] \setrdcrep{seyes=false,correction color=black}\begin{crep}[colback=white,halign=center]
\begin{tikzpicture}[scale=0.7]
    % Calcul des bornes adaptatives
    \pgfmathsetmacro{\ecart}{abs(\xgrand - \xpetit)}
    \pgfmathsetmacro{\marge}{max(1.5, \ecart * 0.4)}
    \pgfmathsetmacro{\xmin}{\xpetit - \marge}
    \pgfmathsetmacro{\xmax}{\xgrand + \marge}

    % Axes avec bornes adaptatives
    \draw[->] (\xmin,0) -- (\xmax,0) node[right] {$x$};
    \draw[->] (0,-3) -- (0,4) node[above] {};
    \draw[thick,blue,domain=\xmin:\xmax,samples=100] plot (\x,{\coeffa*(\x)^2+\coeffb*(\x)+\coeffc});
    \draw (\xpetit,0) node {$+$} node[below=3pt] {$\xpetitaffichage$};
    \draw (\xgrand,0) node {$+$} node[below=3pt] {$\xgrandaffichage$};
    \node[blue,above right] at (0.5,3) {$f(x)=\coeffa x^2
  \ifdim\coeffb pt=1pt
      +x
  \else\ifdim\coeffb pt=-1pt
      -x
  \else\ifdim\coeffb pt<0pt
      \coeffb x
  \else
      +\coeffb x
  \fi\fi\fi
  \ifdim\coeffc pt<0pt
      \coeffc
  \else
      +\coeffc
  \fi$};
    % Variables pour la logique d'affichage
    \def\tempgeq{geq}\def\templeq{leq}\def\tempg{g}\def\templ{l}%

    % Dessiner ligne rouge selon type
    % Pour geq (≥0)
    \ifx\typeineg\tempgeq
        \ifdim\coeffa pt>0pt
            % a>0, f(x)≥0 : extérieur (]-∞,x1]∪[x2,+∞[)
            \draw[red,line width=0.8pt,<-] (\xpetit,0) -- (\xmin,0);
            \draw[red,line width=0.8pt,->] (\xgrand,0) -- (\xmax,0);
            % Crochets fermés
            \draw[red, line width=1.5pt] (\xpetit-0.1,-0.3) -- (\xpetit,-0.3) -- (\xpetit,0.3) -- (\xpetit-0.1,0.3);
            \draw[red, line width=1.5pt] (\xgrand+0.1,-0.3) -- (\xgrand,-0.3) -- (\xgrand,0.3) -- (\xgrand+0.1,0.3);
        \else
            % a<0, f(x)≥0 : intérieur ([x1,x2])
            \draw[red,line width=0.8pt] (\xpetit,0) -- (\xgrand,0);
            % Crochets fermés
            \draw[red, line width=1.5pt] (\xpetit-0.1,-0.3) -- (\xpetit,-0.3) -- (\xpetit,0.3) -- (\xpetit-0.1,0.3);
            \draw[red, line width=1.5pt] (\xgrand+0.1,-0.3) -- (\xgrand,-0.3) -- (\xgrand,0.3) -- (\xgrand+0.1,0.3);
        \fi
    \fi

    % Pour leq (≤0)
    \ifx\typeineg\templeq
        \ifdim\coeffa pt<0pt
            % a<0, f(x)≤0 : extérieur (]-∞,x1]∪[x2,+∞[)
            \draw[red,line width=0.8pt,<-] (\xpetit,0) -- (\xmin,0);
            \draw[red,line width=0.8pt,->] (\xgrand,0) -- (\xmax,0);
            % Crochets fermés
            \draw[red, line width=1.5pt] (\xpetit-0.1,-0.3) -- (\xpetit,-0.3) -- (\xpetit,0.3) -- (\xpetit-0.1,0.3);
            \draw[red, line width=1.5pt] (\xgrand+0.1,-0.3) -- (\xgrand,-0.3) -- (\xgrand,0.3) -- (\xgrand+0.1,0.3);
        \else
            % a>0, f(x)≤0 : intérieur ([x1,x2])
            \draw[red,line width=0.8pt] (\xpetit,0) -- (\xgrand,0);
            % Crochets fermés
            \draw[red, line width=1.5pt] (\xpetit-0.1,-0.3) -- (\xpetit,-0.3) -- (\xpetit,0.3) -- (\xpetit-0.1,0.3);
            \draw[red, line width=1.5pt] (\xgrand+0.1,-0.3) -- (\xgrand,-0.3) -- (\xgrand,0.3) -- (\xgrand+0.1,0.3);
        \fi
    \fi

    % Pour g (>0)
    \ifx\typeineg\tempg
        \ifdim\coeffa pt>0pt
            % a>0, f(x)>0 : extérieur (]-∞,x1[∪]x2,+∞[)
            \draw[red,line width=0.8pt,<-] (\xpetit,0) -- (\xmin,0);
            \draw[red,line width=0.8pt,->] (\xgrand,0) -- (\xmax,0);
            % Crochets ouverts
            \draw[red, line width=1.5pt] (\xpetit,-0.3) -- (\xpetit,0.3) -- (\xpetit+0.1,0.3) (\xpetit,-0.3) -- (\xpetit+0.1,-0.3);
            \draw[red, line width=1.5pt] (\xgrand,-0.3) -- (\xgrand,0.3) -- (\xgrand-0.1,0.3) (\xgrand,-0.3) -- (\xgrand-0.1,-0.3);
        \else
            % a<0, f(x)>0 : intérieur (]x1,x2[)
            \draw[red,line width=0.8pt] (\xpetit,0) -- (\xgrand,0);
            % Crochets ouverts
            \draw[red, line width=1.5pt] (\xpetit,-0.3) -- (\xpetit,0.3) -- (\xpetit+0.1,0.3) (\xpetit,-0.3) -- (\xpetit+0.1,-0.3);
            \draw[red, line width=1.5pt] (\xgrand,-0.3) -- (\xgrand,0.3) -- (\xgrand-0.1,0.3) (\xgrand,-0.3) -- (\xgrand-0.1,-0.3);
        \fi
    \fi

    % Pour l (<0)
    \ifx\typeineg\templ
        \ifdim\coeffa pt<0pt
            % a<0, f(x)<0 : extérieur (]-∞,x1[∪]x2,+∞[)
            \draw[red,line width=0.8pt,<-] (\xpetit,0) -- (\xmin,0);
            \draw[red,line width=0.8pt,->] (\xgrand,0) -- (\xmax,0);
            % Crochets ouverts
            \draw[red, line width=1.5pt] (\xpetit,-0.3) -- (\xpetit,0.3) -- (\xpetit+0.1,0.3) (\xpetit,-0.3) -- (\xpetit+0.1,-0.3);
            \draw[red, line width=1.5pt] (\xgrand,-0.3) -- (\xgrand,0.3) -- (\xgrand-0.1,0.3) (\xgrand,-0.3) -- (\xgrand-0.1,-0.3);
        \else
            % a>0, f(x)<0 : intérieur (]x1,x2[)
            \draw[red,line width=0.8pt] (\xpetit,0) -- (\xgrand,0);
            % Crochets ouverts
            \draw[red, line width=1.5pt] (\xpetit,-0.3) -- (\xpetit,0.3) -- (\xpetit+0.1,0.3) (\xpetit,-0.3) -- (\xpetit+0.1,-0.3);
            \draw[red, line width=1.5pt] (\xgrand,-0.3) -- (\xgrand,0.3) -- (\xgrand-0.1,0.3) (\xgrand,-0.3) -- (\xgrand-0.1,-0.3);
        \fi
    \fi
\end{tikzpicture}
\end{crep}
        \end{MultiColonnes}
    }