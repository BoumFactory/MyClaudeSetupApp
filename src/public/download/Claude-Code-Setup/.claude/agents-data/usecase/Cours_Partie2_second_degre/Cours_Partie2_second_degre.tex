\documentclass[a4paper,11pt,fleqn]{article}

\usepackage[left=1cm,right=1cm,top=0.5cm,bottom=2cm]{geometry}

\usepackage{bfcours}
\usepackage{bfcours-fonts}
%\usepackage{bfcours-fonts-dys}

\def\rdifficulty{1}
\setrdexo{%left skip=1cm,
display exotitle,
exo header = tcolorbox,
%display tags,
skin = bouyachakka,
lower ={box=crep},
display score,
display level,
save lower,
score=\points,
level=\rdifficulty,
overlay={\node[inner sep=0pt,
anchor=west,rotate=90, yshift=0.3cm]%,xshift=-3em], yshift=0.45cm
at (frame.south west) {\thetags[0]} ;}
]%obligatoire
}
\edef\currentseyesoption{\ifcorrection false\else true\fi}
\setrdcrep{seyes=\currentseyesoption, correction=true, correction color=prop, correction font = \large\bfseries}

\newcommand{\tikzinclude}[1]{%
    \stepcounter{tikzfigcounter}%
    \csname tikzfig#1\endcsname
}
\newcommand{\tikzfigTnVi}{
\begin{tikzpicture}[line cap=round,line join=round,>=triangle 45,x=1.0cm,y=1.0cm,scale=0.6]
\begin{axis}[
x=1.0cm,y=1.0cm,
axis lines=middle,
ymajorgrids=true,
xmajorgrids=true,
xmin=-1.5,
xmax=3.5,
ymin=-3.2,
ymax=3.2,
xtick={-1.0,0.0,...,3.0},
ytick={-3.0,-2.0,...,3.0},]
\clip(-1.5,-3.2) rectangle (3.5,3.2);
\draw[line width=2.pt,color=blue,smooth,samples=100,domain=-1.5:3.5] plot(\x,{-0.6*((\x)-1.0)^(2.0)+2.0});
\begin{scriptsize}
\draw[color=blue] (-1,1) node {\large{$\mathcal{C}_f$}};
\end{scriptsize}
\end{axis}
\end{tikzpicture}
}

\newcommand{\tikzfigXWyE}{
\begin{tikzpicture}[line cap=round,line join=round,>=triangle 45,x=1.0cm,y=1.0cm,scale=0.6]
\begin{axis}[
x=1.0cm,y=1.0cm,
axis lines=middle,
ymajorgrids=true,
xmajorgrids=true,
xmin=-1.5,
xmax=3.5,
ymin=-3.2,
ymax=3.2,
xtick={-1.0,0.0,...,3.0},
ytick={-3.0,-2.0,...,3.0},]
\clip(-1.5,-3.2) rectangle (3.5,3.2);
\draw[line width=2.pt,color=blue,smooth,samples=100,domain=-1.5:3.5] plot(\x,{0.5*((\x)-2.0)^(2.0)-3.0});
\begin{scriptsize}
\draw[color=blue] (-1,2.5) node {\large{$\mathcal{C}_g$}};
\end{scriptsize}
\end{axis}
\end{tikzpicture}
}

\newcommand{\tikzfigbxUA}{
\begin{tikzpicture}[line cap=round,line join=round,>=triangle 45,x=1.0cm,y=1.0cm,scale=0.6]
\begin{axis}[
x=1.0cm,y=1.0cm,
axis lines=middle,
ymajorgrids=true,
xmajorgrids=true,
xmin=-1.5,
xmax=3.5,
ymin=-3.2,
ymax=3.2,
xtick={-1.0,0.0,...,3.0},
ytick={-3.0,-2.0,...,3.0},]
\clip(-1.5,-3.2) rectangle (3.5,3.2);
\draw[line width=2.pt,color=blue,smooth,samples=100,domain=-1.5:3.5] plot(\x,{0-0.3*((\x)-1.0)^(2.0)-1.0});
\begin{scriptsize}
\draw[color=blue] (2.5,-1) node {\large{$\mathcal{C}_h$}};
\end{scriptsize}
\end{axis}
\end{tikzpicture}
}

\newcommand{\tikzfigzaPA}{
\begin{tikzpicture}[line cap=round,line join=round,>=triangle 45,x=1.0cm,y=1.0cm,scale=0.6]
\begin{axis}[
x=1.0cm,y=1.0cm,
axis lines=middle,
ymajorgrids=true,
xmajorgrids=true,
xmin=-1.5,
xmax=3.5,
ymin=-3.2,
ymax=3.2,
xtick={-1.0,0.0,...,3.0},
ytick={-3.0,-2.0,...,3.0},]
\clip(-1.5,-3.2) rectangle (3.5,3.2);
\draw[line width=2.pt,color=blue,smooth,samples=100,domain=-1.5:3.5] plot(\x,{0-1*((\x)+0.0)^(2.0)+2.0});
\begin{scriptsize}
\draw[color=blue] (-1,2.5) node {\large{$\mathcal{C}_i$}};
\end{scriptsize}
\end{axis}
\end{tikzpicture}
}

\newcommand{\tikzfigmWht}{
\begin{tikzpicture}[line cap=round,line join=round,>=triangle 45,x=1.0cm,y=1.0cm,scale=0.8]
\clip(2.5,0.5) rectangle (9.5,5.5);
\fill[line width=2.pt,color=ffffff,fill=ffffff,fill opacity=1.0] (3.,5.) -- (9.,5.) -- (9.,1.) -- (3.,1.) -- cycle;
\fill[line width=0.pt,color=ffqqqq,fill=ffqqqq,fill opacity=1.0] (3.,5.) -- (3.,3.5) -- (4.5,3.5) -- (4.5,5.) -- cycle;
\fill[line width=0.pt,color=ffqqqq,fill=ffqqqq,fill opacity=1.0] (3.,1.) -- (3.,2.5) -- (4.5,2.5) -- (4.5,1.) -- cycle;
\fill[line width=0.pt,color=ffqqqq,fill=ffqqqq,fill opacity=1.0] (5.5,1.) -- (5.5,2.5) -- (9.,2.5) -- (9.,1.) -- cycle;
\fill[line width=0.pt,color=ffqqqq,fill=ffqqqq,fill opacity=1.0] (5.5,5.) -- (5.5,3.5) -- (9.,3.5) -- (9.,5.) -- cycle;
\draw [line width=2.pt] (3.,5.)-- (9.,5.);
\draw [line width=2.pt] (9.,5.)-- (9.,1.);
\draw [line width=2.pt] (9.,1.)-- (3.,1.);
\draw [line width=2.pt] (3.,1.)-- (3.,5.);
\end{tikzpicture}
}

\newcommand{\tikzfigFuya}{
\begin{tikzpicture}[line cap=round,line join=round,>=triangle 45,x=1.0cm,y=1.0cm,scale=1]
\clip(-0.1,-0.1) rectangle (4.1,4.1);
\fill[line width=1.pt,color=blue,fill=blue!30,fill opacity=1] (0,0) -- (3.,0.) -- (0.,1.) -- cycle;
\fill[line width=1.pt,color=red,fill=blue!30,fill opacity=1] (4,0) -- (4,3) -- (3,0) -- cycle;
\fill[line width=1.pt,color=blue,fill=blue!30,fill opacity=1] (4,4) -- (1,4) -- (4,3) -- cycle;
\fill[line width=1.pt,color=blue,fill=blue!30,fill opacity=1] (0,4) -- (0,1) -- (1,4) -- cycle;
\draw [line width=1.pt,color=blue] (0,0)-- (4,0);
\draw [line width=1.pt,color=blue] (4,0)-- (4,4);
\draw [line width=1.pt,color=blue] (4,4.)-- (0,4);
\draw [line width=1.pt,color=blue] (0,4)-- (0,0);
\draw [line width=1.pt,color=blue] (0,1)-- (3,0);
\draw [line width=1.pt,color=blue] (3,0)-- (4,3);
\draw [line width=1.pt,color=blue] (4,3)-- (1,4);
\draw [line width=1.pt,color=blue] (1,4)-- (0,1);
\end{tikzpicture}
}



\hypersetup{
    pdfauthor={R.Deschamps},
    pdfsubject={},
    pdfkeywords={},
    pdfproducer={LuaLaTeX},
    pdfcreator={Boum Factory}
}


% ========================================
% MACROS POUR RÉSOLUTION AUTOMATIQUE DE TRINÔMES
% ========================================

\usepackage{siunitx} % Pour la notation avec virgule
\usepackage{xfp}

% Configuration siunitx pour affichage français
\sisetup{%
  output-decimal-marker={,}
}

% Commande personnalisée pour affichage avec virgule
\newcommand{\cperso}[1]{\num{\fpeval{#1}}}

% ========================================
% ACCOLADES GRANDES ET ENSEMBLE DISCRET
% ========================================

% Accolade gauche grande avec TikZ
\newcommand{\accoladeG}[1][1]{%
    \tikz[baseline=0.1ex,scale=#1]{
        \draw[line width=0.8pt,line cap=round]
            (0.15,0.45) .. controls (0.08,0.42) and (0.05,0.35) ..
            (0.05,0.25) .. controls (0.05,0.15) and (0.02,0.08) ..
            (0,0) .. controls (0.02,-0.08) and (0.05,-0.15) ..
            (0.05,-0.25) .. controls (0.05,-0.35) and (0.08,-0.42) ..
            (0.15,-0.45);
    }%
}

% Accolade droite grande avec TikZ
\newcommand{\accoladeD}[1][1]{%
    \tikz[baseline=0.1ex,scale=#1]{
        \draw[line width=0.8pt,line cap=round]
            (0,0.45) .. controls (0.07,0.42) and (0.1,0.35) ..
            (0.1,0.25) .. controls (0.1,0.15) and (0.13,0.08) ..
            (0.15,0) .. controls (0.13,-0.08) and (0.1,-0.15) ..
            (0.1,-0.25) .. controls (0.1,-0.35) and (0.07,-0.42) ..
            (0,-0.45);
    }%
}

% Ensemble discret avec grandes accolades à hauteur fixe
\newcommand{\ensembleDiscret}[2][1]{%
    \ifcase#1\relax
        \mathopen{\{}\,#2\,\mathclose{\}}%
    \or
        \mathopen{\{}\,#2\,\mathclose{\}}%
    \or
        \mathopen{\big\{}\,#2\,\mathclose{\big\}}%
    \or
        \mathopen{\Big\{}\,#2\,\mathclose{\Big\}}%
    \fi
}

% ========================================
% Variables globales pour les coefficients
\newcommand{\seta}[1]{\def\coeffa{#1}}
\newcommand{\setb}[1]{\def\coeffb{#1}}
\newcommand{\setc}[1]{\def\coeffc{#1}}

% Commandes pour afficher les coefficients avec signe
\newcommand{\affsignea}{\ifdim\coeffa pt<0pt-\fi}
\newcommand{\affsigneb}{\ifdim\coeffb pt<0pt-\fi}
\newcommand{\affsignec}{\ifdim\coeffc pt<0pt-\fi}

% Valeurs absolues
\newcommand{\absa}{\fpeval{abs(\coeffa)}}
\newcommand{\absb}{\fpeval{abs(\coeffb)}}
\newcommand{\absc}{\fpeval{abs(\coeffc)}}

% Affichage avec parenthèses si négatif (après un signe)
\newcommand{\parencoeffa}{\ifdim\coeffa pt<0pt(\coeffa)\else\coeffa\fi}
\newcommand{\parencoeffb}{\ifdim\coeffb pt<0pt(\coeffb)\else\coeffb\fi}
\newcommand{\parencoeffc}{\ifdim\coeffc pt<0pt(\coeffc)\else\coeffc\fi}

% Affichage direct (début de ligne ou sans multiplication)
\newcommand{\directcoeffa}{\coeffa}
\newcommand{\directcoeffb}{\coeffb}
\newcommand{\directcoeffc}{\coeffc}

% Calcul du discriminant
\newcommand{\calcdelta}{\fpeval{\coeffb*\coeffb - 4*\coeffa*\coeffc}}
\newcommand{\deltasimpl}{\cperso{round(\calcdelta,4)}}

% Affichage du calcul du discriminant avec gestion intelligente des signes
\newcommand{\affcalcdelta}{%
    \directcoeffb^2-4\times\parencoeffa\times\parencoeffc = %
    \cperso{\coeffb*\coeffb}%
    \ifdim\fpeval{4*\coeffa*\coeffc} pt<0pt
        +\cperso{abs(4*\coeffa*\coeffc)}
    \else
        -\cperso{4*\coeffa*\coeffc}
    \fi = %
    \deltasimpl
}

% Signe de delta
\newcommand{\signedelta}{%
    \ifnum\numexpr\fpeval{round(\calcdelta,0)}<0
        <
    \else
        \ifnum\numexpr\fpeval{round(\calcdelta,0)}>0
            >
        \else
            =
        \fi
    \fi
}

% Test si delta >= 0
\newcommand{\ifdeltapositif}[2]{%
    \ifnum\numexpr\fpeval{round(\calcdelta,0)}<0
        #2%
    \else
        #1%
    \fi
}

% Calcul de x0 (cas delta = 0)
\newcommand{\calcxzero}{\fpeval{round(-\coeffb/(2*\coeffa),4)}}
\newcommand{\calcxzerofrac}{\cperso{round(-\coeffb/(2*\coeffa),4)}}

% Affichage du calcul de x0
\newcommand{\affcalcxzero}{%
    -\dfrac{b}{2a} = -\dfrac{\coeffb}{2\times\parencoeffa} = %
    \ifdim\coeffb pt<0pt
        \dfrac{\absb}{2\times\parencoeffa}
    \else
        -\dfrac{\coeffb}{2\times\parencoeffa}
    \fi = %
    \ifdim\fpeval{-\coeffb} pt<0pt
        \dfrac{\cperso{abs(-\coeffb)}}{\cperso{2*\coeffa}}
    \else
        -\dfrac{\cperso{-\coeffb}}{\cperso{abs(2*\coeffa)}}
    \fi = %
    \calcxzerofrac
}

% Calcul des racines (cas delta > 0)
\newcommand{\calcxun}{\fpeval{round((-\coeffb - sqrt(\calcdelta))/(2*\coeffa),4)}}
\newcommand{\calcxdeux}{\fpeval{round((-\coeffb + sqrt(\calcdelta))/(2*\coeffa),4)}}

% Racine carrée de delta arrondie
\newcommand{\sqrtdelta}{\fpeval{round(sqrt(\calcdelta),4)}}
\newcommand{\sqrtdeltafrac}{\cperso{round(sqrt(\calcdelta),4)}}

% Affichage du calcul de x1
\newcommand{\affcalcxun}{%
    \dfrac{-b-\sqrt{\Delta}}{2a} = %
    \dfrac{%
        \ifdim\coeffb pt<0pt
            \absb
        \else
            -\coeffb
        \fi
        -\sqrt{\deltasimpl}%
    }{2\times\parencoeffa} = %
    \ifdim\issimplexun pt=1pt
        % Si simple, montrer les étapes de calcul
        \dfrac{%
            \ifdim\coeffb pt<0pt
                \absb
            \else
                -\coeffb
            \fi
            -\sqrtdeltafrac%
        }{\cperso{2*\coeffa}} = %
        \dfrac{\cperso{-\coeffb-\sqrtdelta}}{\cperso{2*\coeffa}} = %
        \xunaffichage
    \else
        % Si complexe, afficher directement la forme exacte simplifiée
        \xunaffichage
    \fi
}

% Affichage du calcul de x2
\newcommand{\affcalcxdeux}{%
    \dfrac{-b+\sqrt{\Delta}}{2a} = %
    \dfrac{%
        \ifdim\coeffb pt<0pt
            \absb
        \else
            -\coeffb
        \fi
        +\sqrt{\deltasimpl}%
    }{2\times\parencoeffa} = %
    \ifdim\issimplexdeux pt=1pt
        % Si simple, montrer les étapes de calcul
        \dfrac{%
            \ifdim\coeffb pt<0pt
                \absb
            \else
                -\coeffb
            \fi
            +\sqrtdeltafrac%
        }{\cperso{2*\coeffa}} = %
        \dfrac{\cperso{-\coeffb+\sqrtdelta}}{\cperso{2*\coeffa}} = %
        \xdeuxaffichage
    \else
        % Si complexe, afficher directement la forme exacte simplifiée
        \xdeuxaffichage
    \fi
}

% Simplification des racines si possibles (affichage fraction avec virgule)
\newcommand{\xunfrac}{%
    \cperso{abs(round(\calcxun,0) - \calcxun) < 0.001 ? round(\calcxun,0) : \calcxun}%
}

\newcommand{\xdeuxfrac}{%
    \cperso{abs(round(\calcxdeux,0) - \calcxdeux) < 0.001 ? round(\calcxdeux,0) : \calcxdeux}%
}

% Plus petite et plus grande racine (pour calculs)
\newcommand{\xpetit}{\fpeval{min(\calcxun,\calcxdeux)}}
\newcommand{\xgrand}{\fpeval{max(\calcxun,\calcxdeux)}}

% Plus petite et plus grande racine (pour affichage avec virgule)
\newcommand{\xpetitfrac}{\cperso{min(\calcxun,\calcxdeux)}}
\newcommand{\xgrandfrac}{\cperso{max(\calcxun,\calcxdeux)}}

% ========================================
% AFFICHAGE FORME EXACTE DES RACINES
% ========================================

% Test si une racine est "simple" (entière ou 1 décimale max)
\newcommand{\issimplexun}{%
    \fpeval{abs(round(\calcxun,1) - \calcxun) < 0.01 ? 1 : 0}%
}
\newcommand{\issimplexdeux}{%
    \fpeval{abs(round(\calcxdeux,1) - \calcxdeux) < 0.01 ? 1 : 0}%
}

% Forme exacte de x1 = (-b - √Δ)/(2a)
\newcommand{\xunexact}{%
    \ifdim\fpeval{2*\coeffa} pt=-1pt
        % Cas 2a = -1 : résultat = b + √Δ
        \ifdim\coeffb pt<0pt
            \fpeval{abs(\coeffb)} - \sqrt{\deltasimpl}%
        \else
            \coeffb + \sqrt{\deltasimpl}%
        \fi
    \else\ifdim\fpeval{2*\coeffa} pt=1pt
        % Cas 2a = 1 : résultat = -b - √Δ
        \ifdim\coeffb pt<0pt
            \fpeval{abs(\coeffb)} - \sqrt{\deltasimpl}%
        \else
            -\coeffb - \sqrt{\deltasimpl}%
        \fi
    \else
        % Cas général : fraction
        \dfrac{%
            \ifdim\coeffb pt<0pt
                \fpeval{abs(\coeffb)}%
            \else
                -\coeffb
            \fi
            - \sqrt{\deltasimpl}%
        }{\cperso{2*\coeffa}}%
    \fi\fi
}

% Forme exacte de x2 = (-b + √Δ)/(2a)
\newcommand{\xdeuxexact}{%
    \ifdim\fpeval{2*\coeffa} pt=-1pt
        % Cas 2a = -1 : résultat = b - √Δ
        \ifdim\coeffb pt<0pt
            \fpeval{abs(\coeffb)} + \sqrt{\deltasimpl}%
        \else
            \coeffb - \sqrt{\deltasimpl}%
        \fi
    \else\ifdim\fpeval{2*\coeffa} pt=1pt
        % Cas 2a = 1 : résultat = -b + √Δ
        \ifdim\coeffb pt<0pt
            \fpeval{abs(\coeffb)} + \sqrt{\deltasimpl}%
        \else
            -\coeffb + \sqrt{\deltasimpl}%
        \fi
    \else
        % Cas général : fraction
        \dfrac{%
            \ifdim\coeffb pt<0pt
                \fpeval{abs(\coeffb)}%
            \else
                -\coeffb
            \fi
            + \sqrt{\deltasimpl}%
        }{\cperso{2*\coeffa}}%
    \fi\fi
}

% Affichage intelligent de x1 (forme exacte si complexe, décimale si simple)
\newcommand{\xunaffichage}{%
    \ifdim\issimplexun pt=1pt
        \xunfrac
    \else
        \xunexact
    \fi
}

% Affichage intelligent de x2 (forme exacte si complexe, décimale si simple)
\newcommand{\xdeuxaffichage}{%
    \ifdim\issimplexdeux pt=1pt
        \xdeuxfrac
    \else
        \xdeuxexact
    \fi
}

% Affichage intelligent du petit (min)
\newcommand{\xpetitaffichage}{%
    \ifdim\calcxun pt<\calcxdeux pt
        \xunaffichage
    \else
        \xdeuxaffichage
    \fi
}

% Affichage intelligent du grand (max)
\newcommand{\xgrandaffichage}{%
    \ifdim\calcxun pt>\calcxdeux pt
        \xunaffichage
    \else
        \xdeuxaffichage
    \fi
}

% Signe de a (pour tableau de signe)
\newcommand{\signea}{%
    \ifdim\coeffa pt<0pt
        -
    \else
        +
    \fi
}

% Signe de -a
\newcommand{\signemoina}{%
    \ifdim\coeffa pt<0pt
        +
    \else
        -
    \fi
}

% ========================================
% COMMANDE PRINCIPALE : RÉSOLUTION ÉQUATION
% ========================================
\newcommand{\resoudreequation}[4]{%
    \seta{#1}\setb{#2}\setc{#3}%
    \begin{tcbenumerate}[2]
    \tcbitem \textbf{Identification des coefficients :}

    $a=\coeffa$, $b=\coeffb$ et $c=\coeffc$

    \tcbitem \textbf{Calcul du discriminant :}

    $\Delta = b^2-4ac = \affcalcdelta$

    \ifdeltapositif{%
        \ifnum\numexpr\fpeval{round(\calcdelta,0)}=0
            % Cas delta = 0
            \tcbitem[raster multicolumn=2]\textbf{Détermination de la racine :}   $\Delta = 0$ donc l'équation admet une solution double :

            $x_0 = \affcalcxzero$

            \tcbitem[raster multicolumn=2]\textbf{Ensemble des solutions :}

            $S = \ensembleDiscret{ \calcxzero }$
        \else
            % Cas delta > 0
            \tcbitem[raster multicolumn=2]\textbf{Détermination des racines :}
            $\Delta = \deltasimpl > 0$ donc l'équation admet deux solutions réelles distinctes :

            \begin{MultiColonnes}{2}[halign=center]
            \tcbitem $x_1 = \affcalcxun$
            \tcbitem $x_2 = \affcalcxdeux$
            \end{MultiColonnes}

            \tcbitem[raster multicolumn=2]\textbf{Ensemble des solutions :} \encadrer[red]{$S = \ensembleDiscret{ \xunaffichage ; \xdeuxaffichage }$}
        \fi
    }{%
        % Cas delta < 0
        \tcbitem[raster multicolumn=2]\textbf{Conclusion :}
        $\Delta = \deltasimpl < 0$ donc l'équation n'admet pas de solution réelle.

        $S = \emptyset$
    }%
    \end{tcbenumerate}
}

% ========================================
% COMMANDE PRINCIPALE : RÉSOLUTION INÉQUATION
% ========================================
\newcommand{\resoudreinequation}[4]{%
    % #1 = a, #2 = b, #3 = c, #4 = type (geq, leq, g, l)
    \seta{#1}\setb{#2}\setc{#3}%
    \def\typeineg{#4}%
    \begin{tcbenumerate}[2]
    \tcbitem \textbf{Identification des coefficients :}

    $a=\coeffa$, $b=\coeffb$ et $c=\coeffc$

    \tcbitem\textbf{Calcul du discriminant :}

    $\Delta = b^2-4ac = \affcalcdelta$

    \tcbitem[raster multicolumn=2]\textbf{Détermination des racines :}

    \ifdeltapositif{%
        \ifnum\numexpr\fpeval{round(\calcdelta,0)}=0
            $\Delta = 0$ donc le trinôme admet une racine double :

            $x_0 = \affcalcxzero$
        \else
            $\Delta = \deltasimpl > 0$ donc le trinôme admet deux racines réelles distinctes :

            \begin{MultiColonnes}{2}[halign=center,boxrule=0.4pt,colframe=black,colback=white]
            \tcbitem $x_1 = \affcalcxun$
            \tcbitem $x_2 = \affcalcxdeux$
            \end{MultiColonnes}
        \fi
    }{%
        $\Delta = \deltasimpl < 0$ donc pas de racine réelle.
    }%
        \end{tcbenumerate}
        \begin{MultiColonnes}{2}
            \tcbitem \begin{tcbenumerate}[2][4]\tcbitem[raster multicolumn=2]\textbf{Tableau de signe} de $f:x\mapsto ax^2+bx+c$ :

    On a $a=\coeffa$ \ifdim\coeffa pt<0pt{$<0$}\else{$>0$}\fi{} et %
    \ifdeltapositif{%
        \ifnum\numexpr\fpeval{round(\calcdelta,0)}=0
            $x_0 = \calcxzero$
        \else
            $x_1 = \xunaffichage$ et $x_2 = \xdeuxaffichage$ donc $\xpetitaffichage < \xgrandaffichage$
        \fi
    }{pas de racine}

    \ifdeltapositif{%
        \ifnum\numexpr\fpeval{round(\calcdelta,0)}=0
            % Delta = 0
            \begin{tikzpicture}
            \tkzTabInit[espcl=2,lgt=1.5]{$x$/0.8,$f(x)$/0.8}
            {$-\infty$,$\calcxzero$,$+\infty$}
            \tkzTabLine{,\signea,z,\signea}
            \end{tikzpicture}
        \else
            % Delta > 0
            \begin{tikzpicture}
            \tkzTabInit[espcl=2,lgt=1.5]{$x$/0.8,$f(x)$/0.8}
            {$-\infty$,$\xpetitaffichage$,$\xgrandaffichage$,$+\infty$}
            \tkzTabLine{,\signea,z,\signemoina,z,\signea}
            \end{tikzpicture}
        \fi
    }{%
        % Delta < 0
        \begin{tikzpicture}
        \tkzTabInit[espcl=3,lgt=1.5]{$x$/0.8,$f(x)$/0.8}
        {$-\infty$,$+\infty$}
        \tkzTabLine{,\signea}
        \end{tikzpicture}
    }%

    \tcbitem[raster multicolumn=2]\textbf{Solution de l'inéquation :}

    % Solution selon le type et le signe
    \def\tempgeq{geq}\def\templeq{leq}\def\tempg{g}\def\templ{l}%
    \ifx\typeineg\tempgeq
        On cherche où le trinôme est $\geq 0$, donc les signes $+$ et $0$ :
    \fi
    \ifx\typeineg\templeq
        On cherche où le trinôme est $\leq 0$, donc les signes $-$ et $0$ :
    \fi
    \ifx\typeineg\tempg
        On cherche où le trinôme est $> 0$, donc les signes $+$ (strictement) :
    \fi
    \ifx\typeineg\templ
        On cherche où le trinôme est $< 0$, donc les signes $-$ (strictement) :
    \fi

    % Déterminer la solution selon delta et type
    \ifdeltapositif{%
        \ifnum\numexpr\fpeval{round(\calcdelta,0)}=0
            % Delta = 0 : solution selon signe de a et type
            \ifx\typeineg\tempgeq
                \ifdim\coeffa pt>0pt
                    $S = \R$
                \else
                    $S = \ensembleDiscret{\calcxzerofrac \right\rbrace$}
                \fi
            \fi
            \ifx\typeineg\templeq
                \ifdim\coeffa pt<0pt
                    $S = \R$
                \else
                    $S = \ensembleDiscret{ \calcxzerofrac }$
                \fi
            \fi
            \ifx\typeineg\tempg
                \ifdim\coeffa pt>0pt
                    $S = \R \setminus \ensembleDiscret{\calcxzerofrac}$
                \else
                    $S = \emptyset$
                \fi
            \fi
            \ifx\typeineg\templ
                \ifdim\coeffa pt<0pt
                    $S = \R \setminus \ensembleDiscret{\calcxzerofrac}$
                \else
                    $S = \emptyset$
                \fi
            \fi
        \else
            % Delta > 0 : solution selon signe de a et type
            \ifx\typeineg\tempgeq
                \ifdim\coeffa pt>0pt
                    $S = \CrochetD-\infty;\xpetitaffichage\,\,\CrochetD \cup \CrochetG\xgrandaffichage;+\infty\,\,\CrochetG$
                \else
                    $S = \CrochetG\xpetitaffichage;\xgrandaffichage\,\,\CrochetD$
                \fi
            \fi
            \ifx\typeineg\templeq
                \ifdim\coeffa pt<0pt
                    $S = \CrochetD-\infty;\xpetitaffichage\CrochetD \cup \CrochetG\xgrandaffichage;+\infty\,\,\CrochetG$
                \else
                    $S = \CrochetG\xpetitaffichage;\xgrandaffichage\,\,\CrochetD$
                \fi
            \fi
            \ifx\typeineg\tempg
                \ifdim\coeffa pt>0pt
                    $S = \CrochetD-\infty;\xpetitaffichage\,\,\CrochetG \cup \CrochetD\xgrandaffichage;+\infty\,\,\CrochetG$
                \else
                    $S = \CrochetD\xpetitaffichage;\xgrandaffichage\,\,\CrochetG$
                \fi
            \fi
            \ifx\typeineg\templ
                \ifdim\coeffa pt<0pt
                    $S = \CrochetD-\infty;\xpetitaffichage\,\,\CrochetG \cup \CrochetD\xgrandaffichage;+\infty\,\,\CrochetG$
                \else
                    $S = \CrochetD\xpetitaffichage;\xgrandaffichage\,\,\CrochetG$
                \fi
            \fi
        \fi
    }{%
        % Delta < 0 : solution selon signe de a et type
        \ifx\typeineg\tempgeq
            \ifdim\coeffa pt>0pt
                $S = \R$
            \else
                $S = \emptyset$
            \fi
        \fi
        \ifx\typeineg\templeq
            \ifdim\coeffa pt<0pt
                $S = \R$
            \else
                $S = \emptyset$
            \fi
        \fi
        \ifx\typeineg\tempg
            \ifdim\coeffa pt>0pt
                $S = \R$
            \else
                $S = \emptyset$
            \fi
        \fi
        \ifx\typeineg\templ
            \ifdim\coeffa pt<0pt
                $S = \R$
            \else
                $S = \emptyset$
            \fi
        \fi
    }%
        \end{tcbenumerate}
            \tcbitem[halign=center,valign=center] \setrdcrep{seyes=false,correction color=black}\begin{crep}[colback=white,halign=center]
\begin{tikzpicture}[scale=0.7]
    % Calcul des bornes adaptatives
    \pgfmathsetmacro{\ecart}{abs(\xgrand - \xpetit)}
    \pgfmathsetmacro{\marge}{max(1.5, \ecart * 0.4)}
    \pgfmathsetmacro{\xmin}{\xpetit - \marge}
    \pgfmathsetmacro{\xmax}{\xgrand + \marge}

    % Axes avec bornes adaptatives
    \draw[->] (\xmin,0) -- (\xmax,0) node[right] {$x$};
    \draw[->] (0,-3) -- (0,4) node[above] {};
    \draw[thick,blue,domain=\xmin:\xmax,samples=100] plot (\x,{\coeffa*(\x)^2+\coeffb*(\x)+\coeffc});
    \draw (\xpetit,0) node {$+$} node[below=3pt] {$\xpetitaffichage$};
    \draw (\xgrand,0) node {$+$} node[below=3pt] {$\xgrandaffichage$};
    \node[blue,above right] at (0.5,3) {$f(x)=\coeffa x^2
  \ifdim\coeffb pt=1pt
      +x
  \else\ifdim\coeffb pt=-1pt
      -x
  \else\ifdim\coeffb pt<0pt
      \coeffb x
  \else
      +\coeffb x
  \fi\fi\fi
  \ifdim\coeffc pt<0pt
      \coeffc
  \else
      +\coeffc
  \fi$};
    % Variables pour la logique d'affichage
    \def\tempgeq{geq}\def\templeq{leq}\def\tempg{g}\def\templ{l}%

    % Dessiner ligne rouge selon type
    % Pour geq (≥0)
    \ifx\typeineg\tempgeq
        \ifdim\coeffa pt>0pt
            % a>0, f(x)≥0 : extérieur (]-∞,x1]∪[x2,+∞[)
            \draw[red,line width=0.8pt,<-] (\xpetit,0) -- (\xmin,0);
            \draw[red,line width=0.8pt,->] (\xgrand,0) -- (\xmax,0);
            % Crochets fermés
            \draw[red, line width=1.5pt] (\xpetit-0.1,-0.3) -- (\xpetit,-0.3) -- (\xpetit,0.3) -- (\xpetit-0.1,0.3);
            \draw[red, line width=1.5pt] (\xgrand+0.1,-0.3) -- (\xgrand,-0.3) -- (\xgrand,0.3) -- (\xgrand+0.1,0.3);
        \else
            % a<0, f(x)≥0 : intérieur ([x1,x2])
            \draw[red,line width=0.8pt] (\xpetit,0) -- (\xgrand,0);
            % Crochets fermés
            \draw[red, line width=1.5pt] (\xpetit-0.1,-0.3) -- (\xpetit,-0.3) -- (\xpetit,0.3) -- (\xpetit-0.1,0.3);
            \draw[red, line width=1.5pt] (\xgrand+0.1,-0.3) -- (\xgrand,-0.3) -- (\xgrand,0.3) -- (\xgrand+0.1,0.3);
        \fi
    \fi

    % Pour leq (≤0)
    \ifx\typeineg\templeq
        \ifdim\coeffa pt<0pt
            % a<0, f(x)≤0 : extérieur (]-∞,x1]∪[x2,+∞[)
            \draw[red,line width=0.8pt,<-] (\xpetit,0) -- (\xmin,0);
            \draw[red,line width=0.8pt,->] (\xgrand,0) -- (\xmax,0);
            % Crochets fermés
            \draw[red, line width=1.5pt] (\xpetit-0.1,-0.3) -- (\xpetit,-0.3) -- (\xpetit,0.3) -- (\xpetit-0.1,0.3);
            \draw[red, line width=1.5pt] (\xgrand+0.1,-0.3) -- (\xgrand,-0.3) -- (\xgrand,0.3) -- (\xgrand+0.1,0.3);
        \else
            % a>0, f(x)≤0 : intérieur ([x1,x2])
            \draw[red,line width=0.8pt] (\xpetit,0) -- (\xgrand,0);
            % Crochets fermés
            \draw[red, line width=1.5pt] (\xpetit-0.1,-0.3) -- (\xpetit,-0.3) -- (\xpetit,0.3) -- (\xpetit-0.1,0.3);
            \draw[red, line width=1.5pt] (\xgrand+0.1,-0.3) -- (\xgrand,-0.3) -- (\xgrand,0.3) -- (\xgrand+0.1,0.3);
        \fi
    \fi

    % Pour g (>0)
    \ifx\typeineg\tempg
        \ifdim\coeffa pt>0pt
            % a>0, f(x)>0 : extérieur (]-∞,x1[∪]x2,+∞[)
            \draw[red,line width=0.8pt,<-] (\xpetit,0) -- (\xmin,0);
            \draw[red,line width=0.8pt,->] (\xgrand,0) -- (\xmax,0);
            % Crochets ouverts
            \draw[red, line width=1.5pt] (\xpetit,-0.3) -- (\xpetit,0.3) -- (\xpetit+0.1,0.3) (\xpetit,-0.3) -- (\xpetit+0.1,-0.3);
            \draw[red, line width=1.5pt] (\xgrand,-0.3) -- (\xgrand,0.3) -- (\xgrand-0.1,0.3) (\xgrand,-0.3) -- (\xgrand-0.1,-0.3);
        \else
            % a<0, f(x)>0 : intérieur (]x1,x2[)
            \draw[red,line width=0.8pt] (\xpetit,0) -- (\xgrand,0);
            % Crochets ouverts
            \draw[red, line width=1.5pt] (\xpetit,-0.3) -- (\xpetit,0.3) -- (\xpetit+0.1,0.3) (\xpetit,-0.3) -- (\xpetit+0.1,-0.3);
            \draw[red, line width=1.5pt] (\xgrand,-0.3) -- (\xgrand,0.3) -- (\xgrand-0.1,0.3) (\xgrand,-0.3) -- (\xgrand-0.1,-0.3);
        \fi
    \fi

    % Pour l (<0)
    \ifx\typeineg\templ
        \ifdim\coeffa pt<0pt
            % a<0, f(x)<0 : extérieur (]-∞,x1[∪]x2,+∞[)
            \draw[red,line width=0.8pt,<-] (\xpetit,0) -- (\xmin,0);
            \draw[red,line width=0.8pt,->] (\xgrand,0) -- (\xmax,0);
            % Crochets ouverts
            \draw[red, line width=1.5pt] (\xpetit,-0.3) -- (\xpetit,0.3) -- (\xpetit+0.1,0.3) (\xpetit,-0.3) -- (\xpetit+0.1,-0.3);
            \draw[red, line width=1.5pt] (\xgrand,-0.3) -- (\xgrand,0.3) -- (\xgrand-0.1,0.3) (\xgrand,-0.3) -- (\xgrand-0.1,-0.3);
        \else
            % a>0, f(x)<0 : intérieur (]x1,x2[)
            \draw[red,line width=0.8pt] (\xpetit,0) -- (\xgrand,0);
            % Crochets ouverts
            \draw[red, line width=1.5pt] (\xpetit,-0.3) -- (\xpetit,0.3) -- (\xpetit+0.1,0.3) (\xpetit,-0.3) -- (\xpetit+0.1,-0.3);
            \draw[red, line width=1.5pt] (\xgrand,-0.3) -- (\xgrand,0.3) -- (\xgrand-0.1,0.3) (\xgrand,-0.3) -- (\xgrand-0.1,-0.3);
        \fi
    \fi
\end{tikzpicture}
\end{crep}
        \end{MultiColonnes}
    }
\begin{document}

\setcounter{pagecounter}{0}
\setcounter{ExoMA}{0}



\def\rdifficulty{1}
\chapitre[
    $\mathbf{1^{\text{ère}}}$% : $\mathbf{6^{\text{ème}}}$,$\mathbf{5^{\text{ème}}}$,$\mathbf{4^{\text{ème}}}$,$\mathbf{3^{\text{ème}}}$,$\mathbf{2^{\text{nde}}}$,$\mathbf{1^{\text{ère}}}$,$\mathbf{T^{\text{Le}}}$,
    ]{
    Factorisation des trinômes du second degré% : ,Equations
    }{
    Lycée% : Collège,Lycée
    }{
    Camille Claudel% : Amadis Jamyn,Eugène Belgrand
    }{
    % : ,\tableauPresenteEvalSixieme{}{10},\tableofcontents
    }{
    Cours :% : Cours :,Exercices
    }

    \tableofcontents
    
    \vfill
    \tableaucompetence{
        \competence{Calculer le discriminant associé à un polynôme de degré 2}
        \competence{Connaître le lien entre discriminant et signe d'un polynôme de degré 2}
        \competence{Dresser le tableau de signes d'une fonction polynôme de degré 2}
        \competence{Résoudre des équations et des inéquations du second degré}
    }
    \vfill
    \printvocindex
    \vfill
    \newpage
\begin{EXO}{Différentes formes des polynômes de degré 2}{}%8 points 
Soit $f$ la fonction définie sur $\R$ par $ f(x) = 2x^2+4x-16$.

\begin{tcbenumerate}
\tcbitem \tcbitempoint{2}Montrer que pour tout réel $x$, $f(x) = (2x-4)(x+4)$.
\tcbitem \tcbitempoint{2}Montrer que pour tout réel $x$, $f(x) = 2(x+1)^2-18$.
\tcbitem[boxrule=0.4pt,colframe=black] Choisir la forme la plus adaptée pour répondre aux questions suivantes, puis y répondre.
\begin{tcbenumerate}[2][1][alph]
\tcbitem \tcbitempoint{1}Dresser le tableau de variations de $f$
\tcbitem \tcbitempoint{1}Résoudre $f(x)=-16$
\tcbitem \tcbitempoint{1}Dresser le tableau de signes de $f$
\tcbitem \tcbitempoint{1}Résoudre $f(x)>0$
\end{tcbenumerate}
\end{tcbenumerate}

\exocorrection

\begin{tcbenumerate}[2]
\tcbitem Montrer que pour tout réel $x$, $f(x) = (2x-4)(x+4)$.
Développons $(2x-4)(x+4)$ :
\begin{align*}
(2x-4)(x+4) &= 2x \times x + 2x \times 4 - 4 \times x - 4 \times 4\\
&= 2x^2 + 8x - 4x - 16\\
&= 2x^2 + 4x - 16\\
&= f(x)
\end{align*}

\tcbitem Montrer que pour tout réel $x$, $f(x) = 2(x+1)^2-18$.

Développons $2(x+1)^2-18$ :
\begin{align*}
2(x+1)^2-18 &= 2(x^2 + 2x + 1) - 18\\
&= 2x^2 + 4x + 2 - 18\\
&= 2x^2 + 4x - 16\\
&= f(x)
\end{align*}

\tcbitem[raster multicolumn=2] Choisir la forme la plus adaptée pour répondre aux questions suivantes :

\begin{tcbenumerate}[3][1][alph]
\tcbitem[raster multicolumn=3] Dresser le tableau de variations de $f$

\begin{MultiColonnes}{3}
    \tcbitem \textbf{Forme canonique :} 
    
Le sommet est en $(-1; -18)$ et $a = 2 > 0$, donc la parabole est tournée vers le haut.
    \tcbitem[halign=center,valign=center,raster multicolumn=2] \begin{tikzpicture}
\tkzTabInit{$x$/1,$f$/1.8}
{$-\infty$,$-1$,$+\infty$}
\tkzTabVar{+/$+\infty$,-/$-18$,+/$+\infty$}
\end{tikzpicture}
\end{MultiColonnes}

\tcbitem[raster multicolumn=1] Résoudre $f(x)=-16$

\textbf{Forme développée :} 

$2x^2+4x-16 = -16$

$2x^2+4x = 0$

$2x(x+2) = 0$

Donc $x = 0$ ou $x = -2$.

$S = \{-2 ; 0\}$

\begin{tcbenumerate}[1][4][alph]
    \tcbitem[raster multicolumn=3] Résoudre $f(x)>0$

D'après le tableau de signes précédent :

$S = \CrochetD-\infty;-4\,\,\CrochetG \cup \CrochetD2;+\infty\,\,\CrochetG$

\end{tcbenumerate}

\tcbitem[raster multicolumn=2] Dresser le tableau de signes de $f$

\textbf{Forme factorisée :} $f(x) = (2x-4)(x+4) = 2(x-2)(x+4)$ \\

Les racines sont $x = 2$ et $x = -4$.\\

\begin{tikzpicture}
\tkzTabInit{$x$/1,$2$/1,$(x+4)$/1,$(x-2)$/1,$f(x)$/1}
{$-\infty$,$-4$,$2$,$+\infty$}
\tkzTabLine{,+,t,+,t,+,}
\tkzTabLine{,-,z,+,t,+,}
\tkzTabLine{,-,t,-,z,+,}
\tkzTabLine{,+,z,-,z,+,}
\end{tikzpicture}
\end{tcbenumerate}
\end{tcbenumerate}

\end{EXO}
\begin{EXO}{Lecture graphique}{}%6 points 

\begin{MultiColonnes}{3}
\tcbitem[raster multicolumn=2] Soit $f$ la fonction dont la représentation graphique est donnée ci-contre. 

\begin{tcbenumerate}
\tcbitem \tcbitempoint{1}Lire les coordonnées du sommet $S$ de la parabole.
\tcbitem \tcbitempoint{2}Déterminer la forme canonique de $f$.
\tcbitem \tcbitempoint{1}Déterminer les racines de la fonction $f$.
\tcbitem \tcbitempoint{2}En déduire la forme factorisée de la fonction $f$.
\end{tcbenumerate}

\tcbitem[halign=center,valign=center] \begin{tikzpicture}[line cap=round,line join=round,>=triangle 45,x=1.5cm,y=1.0cm,scale=0.8]
\begin{axis}[
x=1.5cm,y=1.0cm,
axis lines=middle,
ymajorgrids=true,
xmajorgrids=true,
xmin=-1.5,
xmax=3.5,
ymin=-2.5,
ymax=3.5,
xtick={-1.0,0.0,...,3.0},
ytick={-2.0,-1.0,...,3.0},]
\clip(-1.5,-2.5) rectangle (3.5,3.5);
\draw[line width=2.pt,color=blue,smooth,samples=100,domain=-1.5:3.5] plot(\x,{-2.0*((\x)-1.0)^(2.0)+2.0});
\begin{scriptsize}
\draw[color=blue] (2,2) node {$\mathcal{C}_f$};
%\draw [fill=black] (1.,2.) circle (3.0pt);
%\draw[color=black] (0.9,2.55) node {$A$};
%\draw [fill=black] (0.,0.) circle (3.0pt);
%\draw[color=black] (0.34,4.43) node {$B$};
\end{scriptsize}
\end{axis}
\end{tikzpicture}
\end{MultiColonnes}
\exocorrection

\begin{tcbenumerate}[2]
\tcbitem Coordonnées du sommet $S$ de la parabole.

En lisant le graphique, le sommet $S$ se trouve au point le plus haut de la parabole.

$S(1 ; 2)$

\tcbitem Forme canonique de $f$.

D'après la lecture graphique :
- Sommet : $S(1 ; 2)$
- La parabole est tournée vers le bas donc $a < 0$

Pour déterminer $a$, utilisons un autre point de la courbe.
La parabole passe par $(0 ; 0)$.

$f(x) = a(x-1)^2 + 2$

$f(0) = a(0-1)^2 + 2 = a + 2 = 0$

Donc $a = -2$.

$f(x) = -2(x-1)^2 + 2$

\tcbitem Racines de la fonction $f$.

Les racines correspondent aux points d'intersection avec l'axe des abscisses, c'est-à-dire quand $f(x) = 0$.

En lisant le graphique, la parabole coupe l'axe des abscisses en $x = 0$ et $x = 2$.

Les racines sont : $x_1 = 0$ et $x_2 = 2$.

\tcbitem Forme factorisée de la fonction $f$.

Connaissant les racines $x_1 = 0$ et $x_2 = 2$, la forme factorisée est :

$f(x) = a(x-x_1)(x-x_2) = a \cdot x(x-2)$

Pour déterminer $a$, utilisons le sommet $S(1 ; 2)$ :

$f(1) = a \cdot 1 \cdot (1-2) = a \cdot 1 \cdot (-1) = -a = 2$

Donc $a = -2$.

$f(x) = -2x(x-2) = -2x^2 + 4x$

\end{tcbenumerate}

\end{EXO}
\def\rdifficulty{2}
\begin{EXO}{Problème de géométrie}{}%6 points 


$ABCD$ est un carré de coté $4$cm. Soit $x\in \CrochetG 0;4\,\,\CrochetD$. 

\begin{MultiColonnes}{2}
\tcbitem $E$ est le point de $\CrochetG AB\,\,\CrochetD$ tel que $AE=x$.
\tcbitem $F$ est le point de $\CrochetG AD\,\,\CrochetD$ tel que $DF=x$.
\end{MultiColonnes}


\tcbitempoint{6}Déterminer la valeur de $x$ pour que l'aire du triangle $FEC$ soit \acc{minimale}.


\textit{Si vous bloquez sur cette exercice, une aide est disponible sur le bureau du professeur. }

\textit{Le barème de l'exercice sera adapté ( -1 point ) si vous choisissez d'utiliser cette aide.}
\exocorrection

\begin{MultiColonnes}{2}
\tcbitem[title=Configuration générale,halign=center]
\begin{tikzpicture}[scale=0.8]
% Carré ABCD
\draw[thick] (0,0) rectangle (4,4);

% Points du carré
\fill (0,0) circle (2pt) node[below left] {$A$};
\fill (4,0) circle (2pt) node[below right] {$B$};
\fill (4,4) circle (2pt) node[above right] {$C$};
\fill (0,4) circle (2pt) node[above left] {$D$};

% Point E sur AB avec x=1
\fill (1,0) circle (2pt) node[below,xshift=3pt] {$E$};
\draw[dashed] (1,0) -- (1,-0.3);
\draw[<->] (0,-0.6) -- (1,-0.6);
\node[yshift=-10pt] at (0.5,-0.5) {$x=1$};

% Point F sur AD avec DF=x=1
\fill (0,3) circle (2pt) node[left] {$F$};
\draw[dashed] (0,3) -- (-0.3,3);
\draw[<->] (-0.6,3) -- (-0.6,4);
\node[xshift=-15pt] at (-0.6,3.5) {$x=1$};

% Triangle FEC
\draw[red,thick] (0,3) -- (1,0) -- (4,4) -- cycle;
\fill[red,opacity=0.2] (0,3) -- (1,0) -- (4,4) -- cycle;

% Dimensions du carré
\draw[<->] (4.2,0) -- (4.2,4);
\node[xshift=10pt] at (4.5,2) {$4$ cm};
\draw[<->] (0,-1.2) -- (4,-1.2);
\node[yshift=-10pt] at (2,-1) {$4$ cm};

\node[red,yshift=15pt,xshift=-10pt] at (2,2) {$\mathcal{A}=\frac{15}{2}$ cm$^2$};
\end{tikzpicture}

\tcbitem[title=Configuration avec aire minimale,halign=center]
\begin{tikzpicture}[scale=0.8]
% Carré ABCD
\draw[thick] (0,0) rectangle (4,4);

% Points du carré
\fill (0,0) circle (2pt) node[below left] {$A$};
\fill (4,0) circle (2pt) node[below right] {$B$};
\fill (4,4) circle (2pt) node[above right] {$C$};
\fill (0,4) circle (2pt) node[above left] {$D$};

% Point E sur AB avec x=2
\fill (2,0) circle (2pt) node[below] {$E$};
\draw[dashed] (2,0) -- (2,-0.3);
\draw[<->] (0,-0.6) -- (2,-0.6);
\node[yshift=-10pt] at (1,-0.5) {$x=2$};

% Point F sur AD avec DF=x=2
\fill (0,2) circle (2pt) node[left] {$F$};
\draw[dashed] (0,2) -- (-0.3,2);
\draw[<->] (-0.3,2) -- (-0.3,4);
\node[xshift=-10pt] at (-0.6,3) {$x=2$};

% Triangle FEC optimal
\draw[blue,thick] (0,2) -- (2,0) -- (4,4) -- cycle;
\fill[blue,opacity=0.2] (0,2) -- (2,0) -- (4,4) -- cycle;

% Dimensions du carré
\draw[<->] (4.2,0) -- (4.2,4);
\node[xshift=10pt] at (4.5,2) {$4$ cm};
\draw[<->] (0,-1.2) -- (4,-1.2);
\node[yshift=-10pt] at (2,-1) {$4$ cm};

\node[blue,xshift=-15pt] at (2.2,2) {$\mathcal{A}_{\min}=6$ cm$^2$};
\end{tikzpicture}
\end{MultiColonnes}
\begin{tcbenumerate}[2]
\tcbitem[raster multicolumn=2] \textbf{Calcul de l'aire du triangle $FEC$ :} 

\begin{MultiColonnes}{2}
\tcbitem $\text{Aire}_{FEC} = \text{Aire du carré} - \text{Aire des autres triangles}$

L'aire du carré $ABCD$ est : $4 \times 4 = 16$ cm$^2$ ; de plus : 

\begin{MultiColonnes}{1}
    \tcbitem $\text{Aire}_{AFE} = \dfrac{1}{2} \times x \times (4-x) = \dfrac{1}{2}x(4-x) =  {\color{blue}\dfrac{1}{2}(4x-x^2)}$

    \tcbitem $\text{Aire}_{EBC} = \dfrac{1}{2} \times (4-x) \times 4 = 2(4-x) =  {\color{red}8-2x}$

    \tcbitem $\text{Aire}_{FDC} = \dfrac{1}{2} \times x \times 4 =  {\color{defi}2x}$

\end{MultiColonnes}

\tcbitem Donc :
$\text{Aire}_{FEC} = 16 - \text{Aire}_{AFE} - \text{Aire}_{EBC} - \text{Aire}_{FDC}$

$= 16 -  {\color{blue}\dfrac{1}{2}(4x-x^2)} - ( {\color{red}8-2x}) -  {\color{defi}2x}$

$= 16 - 2x + \dfrac{x^2}{2} - 8 + 2x - 2x$

$= 16 - 8 + \dfrac{x^2}{2} - 2x$

$= 8 + \dfrac{x^2}{2} - 2x = \dfrac{1}{2}(x^2 - 4x + 16)$
\end{MultiColonnes}



\tcbitem \textbf{Forme canonique et minimum :} 

$A_{FEC}(x) = \dfrac{1}{2}(x^2 - 4x + 16)$

Mettons sous forme canonique :
$x^2 - 4x + 16 = (x-2)^2 - 4 + 16 = (x-2)^2 + 12$

Donc : $A_{FEC}(x) = \dfrac{1}{2}((x-2)^2 + 12) = \dfrac{1}{2}(x-2)^2 + 6$


\tcbitem \textbf{Conclusion :} 

Le minimum de $A_{FEC}(x)$ est atteint quand $(x-2)^2 = 0$, c'est-à-dire pour $x = 2$.

La valeur minimale de l'aire est $A_{FEC}(2) = 6$ cm$^2$.

\textbf{Réponse :} L'aire du triangle $FEC$ est minimale pour $x = 2$ cm.
\end{tcbenumerate}
\end{EXO}

\newpage
\newcommand{\aideExoTrois}{
    \begin{bfbox}{Aide de l'exercice 3 :}
        \begin{tcbenumerate}[2]
            \tcbitem Faire une figure représentant la situation.
            \tcbitem Justifier l'égalité : $A_{EBC}(x)=8-2x$
            \tcbitem Montrer que l'aire du triangle $FEC$ vaut : $$A_{FEC}(x)=\dfrac{1}{2}\left(x^2-4x+16\right)$$
            \tcbitem En déduire la forme canonique de $A_{FEC}$
            \tcbitem[raster multicolumn=2] Déterminer la valeur de $x$ pour que l'aire du triangle $FEC$ soit minimale.
        \end{tcbenumerate}
    \end{bfbox}
}

\aideExoTrois
\vfill
\aideExoTrois
\vfill
\aideExoTrois
\vfill
\aideExoTrois
\vfill
\aideExoTrois
\vfill
\phantom{a}
% : \begin{EXO}{Différentes formes des polynômes de degré 2}{}%8 points 
Soit $f$ la fonction définie sur $\R$ par $ f(x) = 2x^2+4x-16$.

\begin{tcbenumerate}
\tcbitem \tcbitempoint{2}Montrer que pour tout réel $x$, $f(x) = (2x-4)(x+4)$.
\tcbitem \tcbitempoint{2}Montrer que pour tout réel $x$, $f(x) = 2(x+1)^2-18$.
\tcbitem[boxrule=0.4pt,colframe=black] Choisir la forme la plus adaptée pour répondre aux questions suivantes, puis y répondre.
\begin{tcbenumerate}[2][1][alph]
\tcbitem \tcbitempoint{1}Dresser le tableau de variations de $f$
\tcbitem \tcbitempoint{1}Résoudre $f(x)=-16$
\tcbitem \tcbitempoint{1}Dresser le tableau de signes de $f$
\tcbitem \tcbitempoint{1}Résoudre $f(x)>0$
\end{tcbenumerate}
\end{tcbenumerate}

\exocorrection

\begin{tcbenumerate}[2]
\tcbitem Montrer que pour tout réel $x$, $f(x) = (2x-4)(x+4)$.
Développons $(2x-4)(x+4)$ :
\begin{align*}
(2x-4)(x+4) &= 2x \times x + 2x \times 4 - 4 \times x - 4 \times 4\\
&= 2x^2 + 8x - 4x - 16\\
&= 2x^2 + 4x - 16\\
&= f(x)
\end{align*}

\tcbitem Montrer que pour tout réel $x$, $f(x) = 2(x+1)^2-18$.

Développons $2(x+1)^2-18$ :
\begin{align*}
2(x+1)^2-18 &= 2(x^2 + 2x + 1) - 18\\
&= 2x^2 + 4x + 2 - 18\\
&= 2x^2 + 4x - 16\\
&= f(x)
\end{align*}

\tcbitem[raster multicolumn=2] Choisir la forme la plus adaptée pour répondre aux questions suivantes :

\begin{tcbenumerate}[3][1][alph]
\tcbitem[raster multicolumn=3] Dresser le tableau de variations de $f$

\begin{MultiColonnes}{3}
    \tcbitem \textbf{Forme canonique :} 
    
Le sommet est en $(-1; -18)$ et $a = 2 > 0$, donc la parabole est tournée vers le haut.
    \tcbitem[halign=center,valign=center,raster multicolumn=2] \begin{tikzpicture}
\tkzTabInit{$x$/1,$f$/1.8}
{$-\infty$,$-1$,$+\infty$}
\tkzTabVar{+/$+\infty$,-/$-18$,+/$+\infty$}
\end{tikzpicture}
\end{MultiColonnes}

\tcbitem[raster multicolumn=1] Résoudre $f(x)=-16$

\textbf{Forme développée :} 

$2x^2+4x-16 = -16$

$2x^2+4x = 0$

$2x(x+2) = 0$

Donc $x = 0$ ou $x = -2$.

$S = \{-2 ; 0\}$

\begin{tcbenumerate}[1][4][alph]
    \tcbitem[raster multicolumn=3] Résoudre $f(x)>0$

D'après le tableau de signes précédent :

$S = \CrochetD-\infty;-4\,\,\CrochetG \cup \CrochetD2;+\infty\,\,\CrochetG$

\end{tcbenumerate}

\tcbitem[raster multicolumn=2] Dresser le tableau de signes de $f$

\textbf{Forme factorisée :} $f(x) = (2x-4)(x+4) = 2(x-2)(x+4)$ \\

Les racines sont $x = 2$ et $x = -4$.\\

\begin{tikzpicture}
\tkzTabInit{$x$/1,$2$/1,$(x+4)$/1,$(x-2)$/1,$f(x)$/1}
{$-\infty$,$-4$,$2$,$+\infty$}
\tkzTabLine{,+,t,+,t,+,}
\tkzTabLine{,-,z,+,t,+,}
\tkzTabLine{,-,t,-,z,+,}
\tkzTabLine{,+,z,-,z,+,}
\end{tikzpicture}
\end{tcbenumerate}
\end{tcbenumerate}

\end{EXO}
\begin{EXO}{Lecture graphique}{}%6 points 

\begin{MultiColonnes}{3}
\tcbitem[raster multicolumn=2] Soit $f$ la fonction dont la représentation graphique est donnée ci-contre. 

\begin{tcbenumerate}
\tcbitem \tcbitempoint{1}Lire les coordonnées du sommet $S$ de la parabole.
\tcbitem \tcbitempoint{2}Déterminer la forme canonique de $f$.
\tcbitem \tcbitempoint{1}Déterminer les racines de la fonction $f$.
\tcbitem \tcbitempoint{2}En déduire la forme factorisée de la fonction $f$.
\end{tcbenumerate}

\tcbitem[halign=center,valign=center] \begin{tikzpicture}[line cap=round,line join=round,>=triangle 45,x=1.5cm,y=1.0cm,scale=0.8]
\begin{axis}[
x=1.5cm,y=1.0cm,
axis lines=middle,
ymajorgrids=true,
xmajorgrids=true,
xmin=-1.5,
xmax=3.5,
ymin=-2.5,
ymax=3.5,
xtick={-1.0,0.0,...,3.0},
ytick={-2.0,-1.0,...,3.0},]
\clip(-1.5,-2.5) rectangle (3.5,3.5);
\draw[line width=2.pt,color=blue,smooth,samples=100,domain=-1.5:3.5] plot(\x,{-2.0*((\x)-1.0)^(2.0)+2.0});
\begin{scriptsize}
\draw[color=blue] (2,2) node {$\mathcal{C}_f$};
%\draw [fill=black] (1.,2.) circle (3.0pt);
%\draw[color=black] (0.9,2.55) node {$A$};
%\draw [fill=black] (0.,0.) circle (3.0pt);
%\draw[color=black] (0.34,4.43) node {$B$};
\end{scriptsize}
\end{axis}
\end{tikzpicture}
\end{MultiColonnes}
\exocorrection

\begin{tcbenumerate}[2]
\tcbitem Coordonnées du sommet $S$ de la parabole.

En lisant le graphique, le sommet $S$ se trouve au point le plus haut de la parabole.

$S(1 ; 2)$

\tcbitem Forme canonique de $f$.

D'après la lecture graphique :
- Sommet : $S(1 ; 2)$
- La parabole est tournée vers le bas donc $a < 0$

Pour déterminer $a$, utilisons un autre point de la courbe.
La parabole passe par $(0 ; 0)$.

$f(x) = a(x-1)^2 + 2$

$f(0) = a(0-1)^2 + 2 = a + 2 = 0$

Donc $a = -2$.

$f(x) = -2(x-1)^2 + 2$

\tcbitem Racines de la fonction $f$.

Les racines correspondent aux points d'intersection avec l'axe des abscisses, c'est-à-dire quand $f(x) = 0$.

En lisant le graphique, la parabole coupe l'axe des abscisses en $x = 0$ et $x = 2$.

Les racines sont : $x_1 = 0$ et $x_2 = 2$.

\tcbitem Forme factorisée de la fonction $f$.

Connaissant les racines $x_1 = 0$ et $x_2 = 2$, la forme factorisée est :

$f(x) = a(x-x_1)(x-x_2) = a \cdot x(x-2)$

Pour déterminer $a$, utilisons le sommet $S(1 ; 2)$ :

$f(1) = a \cdot 1 \cdot (1-2) = a \cdot 1 \cdot (-1) = -a = 2$

Donc $a = -2$.

$f(x) = -2x(x-2) = -2x^2 + 4x$

\end{tcbenumerate}

\end{EXO}
\def\rdifficulty{2}
\begin{EXO}{Problème de géométrie}{}%6 points 


$ABCD$ est un carré de coté $4$cm. Soit $x\in \CrochetG 0;4\,\,\CrochetD$. 

\begin{MultiColonnes}{2}
\tcbitem $E$ est le point de $\CrochetG AB\,\,\CrochetD$ tel que $AE=x$.
\tcbitem $F$ est le point de $\CrochetG AD\,\,\CrochetD$ tel que $DF=x$.
\end{MultiColonnes}


\tcbitempoint{6}Déterminer la valeur de $x$ pour que l'aire du triangle $FEC$ soit \acc{minimale}.


\textit{Si vous bloquez sur cette exercice, une aide est disponible sur le bureau du professeur. }

\textit{Le barème de l'exercice sera adapté ( -1 point ) si vous choisissez d'utiliser cette aide.}
\exocorrection

\begin{MultiColonnes}{2}
\tcbitem[title=Configuration générale,halign=center]
\begin{tikzpicture}[scale=0.8]
% Carré ABCD
\draw[thick] (0,0) rectangle (4,4);

% Points du carré
\fill (0,0) circle (2pt) node[below left] {$A$};
\fill (4,0) circle (2pt) node[below right] {$B$};
\fill (4,4) circle (2pt) node[above right] {$C$};
\fill (0,4) circle (2pt) node[above left] {$D$};

% Point E sur AB avec x=1
\fill (1,0) circle (2pt) node[below,xshift=3pt] {$E$};
\draw[dashed] (1,0) -- (1,-0.3);
\draw[<->] (0,-0.6) -- (1,-0.6);
\node[yshift=-10pt] at (0.5,-0.5) {$x=1$};

% Point F sur AD avec DF=x=1
\fill (0,3) circle (2pt) node[left] {$F$};
\draw[dashed] (0,3) -- (-0.3,3);
\draw[<->] (-0.6,3) -- (-0.6,4);
\node[xshift=-15pt] at (-0.6,3.5) {$x=1$};

% Triangle FEC
\draw[red,thick] (0,3) -- (1,0) -- (4,4) -- cycle;
\fill[red,opacity=0.2] (0,3) -- (1,0) -- (4,4) -- cycle;

% Dimensions du carré
\draw[<->] (4.2,0) -- (4.2,4);
\node[xshift=10pt] at (4.5,2) {$4$ cm};
\draw[<->] (0,-1.2) -- (4,-1.2);
\node[yshift=-10pt] at (2,-1) {$4$ cm};

\node[red,yshift=15pt,xshift=-10pt] at (2,2) {$\mathcal{A}=\frac{15}{2}$ cm$^2$};
\end{tikzpicture}

\tcbitem[title=Configuration avec aire minimale,halign=center]
\begin{tikzpicture}[scale=0.8]
% Carré ABCD
\draw[thick] (0,0) rectangle (4,4);

% Points du carré
\fill (0,0) circle (2pt) node[below left] {$A$};
\fill (4,0) circle (2pt) node[below right] {$B$};
\fill (4,4) circle (2pt) node[above right] {$C$};
\fill (0,4) circle (2pt) node[above left] {$D$};

% Point E sur AB avec x=2
\fill (2,0) circle (2pt) node[below] {$E$};
\draw[dashed] (2,0) -- (2,-0.3);
\draw[<->] (0,-0.6) -- (2,-0.6);
\node[yshift=-10pt] at (1,-0.5) {$x=2$};

% Point F sur AD avec DF=x=2
\fill (0,2) circle (2pt) node[left] {$F$};
\draw[dashed] (0,2) -- (-0.3,2);
\draw[<->] (-0.3,2) -- (-0.3,4);
\node[xshift=-10pt] at (-0.6,3) {$x=2$};

% Triangle FEC optimal
\draw[blue,thick] (0,2) -- (2,0) -- (4,4) -- cycle;
\fill[blue,opacity=0.2] (0,2) -- (2,0) -- (4,4) -- cycle;

% Dimensions du carré
\draw[<->] (4.2,0) -- (4.2,4);
\node[xshift=10pt] at (4.5,2) {$4$ cm};
\draw[<->] (0,-1.2) -- (4,-1.2);
\node[yshift=-10pt] at (2,-1) {$4$ cm};

\node[blue,xshift=-15pt] at (2.2,2) {$\mathcal{A}_{\min}=6$ cm$^2$};
\end{tikzpicture}
\end{MultiColonnes}
\begin{tcbenumerate}[2]
\tcbitem[raster multicolumn=2] \textbf{Calcul de l'aire du triangle $FEC$ :} 

\begin{MultiColonnes}{2}
\tcbitem $\text{Aire}_{FEC} = \text{Aire du carré} - \text{Aire des autres triangles}$

L'aire du carré $ABCD$ est : $4 \times 4 = 16$ cm$^2$ ; de plus : 

\begin{MultiColonnes}{1}
    \tcbitem $\text{Aire}_{AFE} = \dfrac{1}{2} \times x \times (4-x) = \dfrac{1}{2}x(4-x) =  {\color{blue}\dfrac{1}{2}(4x-x^2)}$

    \tcbitem $\text{Aire}_{EBC} = \dfrac{1}{2} \times (4-x) \times 4 = 2(4-x) =  {\color{red}8-2x}$

    \tcbitem $\text{Aire}_{FDC} = \dfrac{1}{2} \times x \times 4 =  {\color{defi}2x}$

\end{MultiColonnes}

\tcbitem Donc :
$\text{Aire}_{FEC} = 16 - \text{Aire}_{AFE} - \text{Aire}_{EBC} - \text{Aire}_{FDC}$

$= 16 -  {\color{blue}\dfrac{1}{2}(4x-x^2)} - ( {\color{red}8-2x}) -  {\color{defi}2x}$

$= 16 - 2x + \dfrac{x^2}{2} - 8 + 2x - 2x$

$= 16 - 8 + \dfrac{x^2}{2} - 2x$

$= 8 + \dfrac{x^2}{2} - 2x = \dfrac{1}{2}(x^2 - 4x + 16)$
\end{MultiColonnes}



\tcbitem \textbf{Forme canonique et minimum :} 

$A_{FEC}(x) = \dfrac{1}{2}(x^2 - 4x + 16)$

Mettons sous forme canonique :
$x^2 - 4x + 16 = (x-2)^2 - 4 + 16 = (x-2)^2 + 12$

Donc : $A_{FEC}(x) = \dfrac{1}{2}((x-2)^2 + 12) = \dfrac{1}{2}(x-2)^2 + 6$


\tcbitem \textbf{Conclusion :} 

Le minimum de $A_{FEC}(x)$ est atteint quand $(x-2)^2 = 0$, c'est-à-dire pour $x = 2$.

La valeur minimale de l'aire est $A_{FEC}(2) = 6$ cm$^2$.

\textbf{Réponse :} L'aire du triangle $FEC$ est minimale pour $x = 2$ cm.
\end{tcbenumerate}
\end{EXO}

\newpage
\newcommand{\aideExoTrois}{
    \begin{bfbox}{Aide de l'exercice 3 :}
        \begin{tcbenumerate}[2]
            \tcbitem Faire une figure représentant la situation.
            \tcbitem Justifier l'égalité : $A_{EBC}(x)=8-2x$
            \tcbitem Montrer que l'aire du triangle $FEC$ vaut : $$A_{FEC}(x)=\dfrac{1}{2}\left(x^2-4x+16\right)$$
            \tcbitem En déduire la forme canonique de $A_{FEC}$
            \tcbitem[raster multicolumn=2] Déterminer la valeur de $x$ pour que l'aire du triangle $FEC$ soit minimale.
        \end{tcbenumerate}
    \end{bfbox}
}

\aideExoTrois
\vfill
\aideExoTrois
\vfill
\aideExoTrois
\vfill
\aideExoTrois
\vfill
\aideExoTrois
\vfill
\phantom{a}
,

\newpage
\section{Correction des exercices}

\rdexocorrection{0}

\end{document}