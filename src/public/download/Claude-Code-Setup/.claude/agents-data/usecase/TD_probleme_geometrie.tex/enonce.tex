\vfill
\begin{Activite}[Problème de géométrie]
\tcbitempoint{6}
$ABCD$ est un rectangle tel que $AB=6$cm et $AD=3$cm. Soit $x\in \CrochetG 0;3\,\,\CrochetD$. \\[0.4em]

\begin{MultiColonnes}{2}
\tcbitem $E$ est le point de $\CrochetG AB\,\,\CrochetD$ tel que $AE=2x$.
\tcbitem $F$ est le point de $\CrochetG AD\,\,\CrochetD$ tel que $DF=x$.
\end{MultiColonnes}

%\tcbitempoint{6}Déterminer la valeur de $x$ pour que l'aire du triangle $FEC$ soit \acc{minimale}.
\begin{tcbenumerate}[2]
    \tcbitem Faire une figure représentant la situation.
    \tcbitem Justifier l'égalité : $A_{EBC}(x)=9-3x$
    \tcbitem Montrer que l'aire du triangle $FEC$ vaut : $$A_{FEC}(x)=x^2-3x+9$$
    \tcbitem En déduire la forme canonique de $A_{FEC}$
    \tcbitem[raster multicolumn=2] Déterminer la valeur de $x$ pour que l'aire du triangle $FEC$ soit minimale.
\end{tcbenumerate}
\end{Activite}

\vfill

\begin{Activite}[Problème de géométrie]
\tcbitempoint{6}
$ABCD$ est un rectangle tel que $AB=6$cm et $AD=3$cm. Soit $x\in \CrochetG 0;3\,\,\CrochetD$. \\[0.4em]

\begin{MultiColonnes}{2}
\tcbitem $E$ est le point de $\CrochetG AB\,\,\CrochetD$ tel que $AE=2x$.
\tcbitem $F$ est le point de $\CrochetG AD\,\,\CrochetD$ tel que $DF=x$.
\end{MultiColonnes}

%\tcbitempoint{6}Déterminer la valeur de $x$ pour que l'aire du triangle $FEC$ soit \acc{minimale}.
\begin{tcbenumerate}[2]
    \tcbitem Faire une figure représentant la situation.
    \tcbitem Justifier l'égalité : $A_{EBC}(x)=9-3x$
    \tcbitem Montrer que l'aire du triangle $FEC$ vaut : $$A_{FEC}(x)=x^2-3x+9$$
    \tcbitem En déduire la forme canonique de $A_{FEC}$
    \tcbitem[raster multicolumn=2] Déterminer la valeur de $x$ pour que l'aire du triangle $FEC$ soit minimale.
\end{tcbenumerate}
\end{Activite}

\vfill

\begin{Activite}[Problème de géométrie]
\tcbitempoint{6}
$ABCD$ est un rectangle tel que $AB=6$cm et $AD=3$cm. Soit $x\in \CrochetG 0;3\,\,\CrochetD$. \\[0.4em]

\begin{MultiColonnes}{2}
\tcbitem $E$ est le point de $\CrochetG AB\,\,\CrochetD$ tel que $AE=2x$.
\tcbitem $F$ est le point de $\CrochetG AD\,\,\CrochetD$ tel que $DF=x$.
\end{MultiColonnes}

%\tcbitempoint{6}Déterminer la valeur de $x$ pour que l'aire du triangle $FEC$ soit \acc{minimale}.
\begin{tcbenumerate}[2]
    \tcbitem Faire une figure représentant la situation.
    \tcbitem Justifier l'égalité : $A_{EBC}(x)=9-3x$
    \tcbitem Montrer que l'aire du triangle $FEC$ vaut : $$A_{FEC}(x)=x^2-3x+9$$
    \tcbitem En déduire la forme canonique de $A_{FEC}$
    \tcbitem[raster multicolumn=2] Déterminer la valeur de $x$ pour que l'aire du triangle $FEC$ soit minimale.
\end{tcbenumerate}
\end{Activite}
\vfill
\phantom{a}