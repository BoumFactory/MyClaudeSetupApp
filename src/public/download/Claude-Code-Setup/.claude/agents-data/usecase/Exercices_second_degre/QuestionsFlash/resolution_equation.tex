\begin{EXO}{Résolution d'équation factorisée}{}
Soit $f$ une fonction définie sur $\R$ par $f(x) = 5(x+1)(x-6)$. 

\tcbitempoint{1} Résoudre l'équation $f(x)=0$.

\begin{crep}
Un produit est nul si et seulement si l'un de ses facteurs est nul.

\begin{tcbenumerate}[3]
\tcbitem $5 = 0$ impossible
\tcbitem $x+1 = 0 \Leftrightarrow x = -1$
\tcbitem $x-6 = 0 \Leftrightarrow x = 6$
\end{tcbenumerate}
$S = \{-1 ; 6\}$
\end{crep}

\exocorrection

On a $f(x) = 5(x+1)(x-6) = 0$.

Un produit est nul si et seulement si l'un de ses facteurs est nul.

\begin{tcbenumerate}[3]
\tcbitem $5 = 0$ impossible
\tcbitem $x+1 = 0 \Leftrightarrow x = -1$
\tcbitem $x-6 = 0 \Leftrightarrow x = 6$
\end{tcbenumerate}

Donc $S = \{-1 ; 6\}$.
\end{EXO}