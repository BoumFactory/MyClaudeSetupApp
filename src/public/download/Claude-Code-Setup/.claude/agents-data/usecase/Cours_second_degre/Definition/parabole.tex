\begin{Definition}[Parabole]
    Soit $f$ une fonction polynôme de degré 2 définie sur $\R$ par $f(x)=a x^2+b x + c$, avec $a\in \R^*$ et $b,c \in \R$.
    
    \begin{itemize}
        \item Si $a>0$ alors la représentation graphique de $f$ est une\voc{parabole} $\mathcal{P}$ \frquote{tournée vers le haut}.
        \item Si $a<0$ alors la représentation graphique de $f$ est une \acc{parabole} $\mathcal{P}$ \frquote{tournée vers le bas}.
    \end{itemize}
    
    \begin{center} 
    \begin{tabular}{cc}
    \begin{tikzpicture}[scale=0.8]
        \draw[line width=1.2pt,color=monrose,smooth,samples=100,domain=-0.1:3.1] plot(\x,{(\x-1.5)^(2.0)+0.2});    
        % Axis
        \draw[thick,->] (-0.5,0) -- (3.5,0);
        \draw[thick,->] (0,0) -- (0,2.8);
        \node at (-0.2,2.4) {$c$};
    \end{tikzpicture}
    &
    \begin{tikzpicture}[scale=0.8]
        \draw[line width=1.2pt,color=monrose,smooth,samples=100,domain=-0.1:3.1] plot(\x,{-(\x-1.5)^(2.0)+2.7});    
        % Axis
        \draw[thick,->] (-0.5,0) -- (3.5,0);
        \draw[thick,->] (0,0) -- (0,2.8);
        \node at (-0.2,0.5) {$c$};
    \end{tikzpicture}
    \\ 
    $a>0$ & $a<0$ \\ 
    \end{tabular}  
    \end{center}
\end{Definition}