\begin{Propriete}[Racines et forme factorisée]
    Si un polynôme est écrit sous \acc{forme factorieée} $a(x-x_1)(x-x_2)$ avec $a \neq 0$, alors ses \acc{racines} sont $x_1$ et $x_2$.\\

    De plus un polynôme de degré 2 est écrit sous forme développée $ax^2+bx+c$ et possède deux racines $x_1$ et $x_2$, alors :
    \begin{tcbenumerate}[2]
        \tcbitem La somme des racines est égale à $-\dfrac{b}{a}$
        \tcbitem Le produit des racines est égale à $\dfrac{c}{a}$
    \end{tcbenumerate}
\end{Propriete}