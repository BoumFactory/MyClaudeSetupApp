\begin{Propriete}[Forme canonique]
    Soit $f$ une fonction définie sur $\mathbb{R}$ par $f(x) = ax^{2} + bx + c$ une fonction polynôme du second degré ($a \neq 0$). 
    
    Il existe deux nombres réels $\alpha$ et $\beta$ tels que pour tout nombre réel $x$, on a :
    \begin{center}$ax^{2} + bx + c = a(x-\alpha)^{2} + \beta$\end{center}
    
    L'écriture $a(x-\alpha)^{2} +\beta$ est la\voc{forme canonique} de la fonction $f$.
    
    \begin{tcbenumerate}[2]
        \tcbitem[halign=center] $\alpha = -\dfrac{b}{2a}$
        \tcbitem[halign=center] $\beta = -\dfrac{b^2-4ac}{4a}$
    \end{tcbenumerate}
\end{Propriete}