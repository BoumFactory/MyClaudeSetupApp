\begin{Definition}[Polynôme]
    Un\voc{polynôme} est une somme (finie) de monômes.
    
    \begin{itemize}
        \item Un polynôme est dit\voc{réduit} lorsque tous ses monômes sont de degrés distincts.
        \item Le\voc{degré d'un polynôme réduit} est le plus grand degré de ses monômes.
        
        \item Un polynôme réduit est dit\voc{ordonné} lorsque ses monômes sont rangés suivant les puissances décroissantes de l'indéterminée $X$.
        
    \end{itemize}
\end{Definition}

\begin{Exemple}[Polynômes]
    \begin{tcbenumerate}
        \tcbitem $-3+8X^2+4X-7X^3-7X+10$ est un \acc{polynôme}.
        \tcbitem $3X^2-12X^5 +2$ est sous forme réduite mais $5+7X-14X^2+8X$ n'est pas sous forme \acc{réduite} car les monômes $7X$ et $8X$ sont semblables (même degré).
        \tcbitem $-7X^3+X-3$ est \acc{ordonné} alors que $X-7X^2+2$ ne l'est pas.
    \end{tcbenumerate}
\end{Exemple}
