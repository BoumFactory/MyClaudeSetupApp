\begin{EXO}{Forme factorisée à partir des racines}{}
Soit $f$ la fonction définie sur $\R$ par $f(x)=4x^2-20x-56$. On admet que les racines de $f$ sont $7$ et $-2$. \tcbitempoint{2} Déterminer la forme factorisée de $f$.

\begin{crep}
$f(x) = 4(x-7)(x+2)$
\end{crep}

\exocorrection

Les racines de $f$ sont $x_1 = 7$ et $x_2 = -2$.

La forme factorisée s'écrit $f(x) = a(x-x_1)(x-x_2)$ avec $a$ le coefficient dominant.

Dans la forme développée $f(x) = 4x^2-20x-56$, on lit $a = 4$.

Donc $f(x) = 4(x-7)(x-(-2)) = 4(x-7)(x+2)$.

\textbf{Vérification :} $4(x-7)(x+2) = 4(x^2+2x-7x-14) = 4(x^2-5x-14) = 4x^2-20x-56$ \checkmark
\end{EXO}