\begin{Definition}[Monômes]
    Les\voc{monômes} sont des expressions algébriques formées du produit d'un coefficient $a$ réel par une puissance (entière) d'une indéterminée $X$ : $aX^n$. 
    
    L'exposant de $X$ est appelé le\voc{degré du monôme}.

\end{Definition}

\begin{Exemple}[Exemples de monômes]
    \begin{tcbenumerate}[3]
        \tcbitem $4X^0=4$ est de degré $0$
        \tcbitem $-3X^1=-3X$ est de degré $1$
        \tcbitem $\pi X^2$ est de degré $2$
        \tcbitem $12{,}5X^7$ est de degré $7$
        \tcbitem $0X=0$ est de degré $-\infty$
    \end{tcbenumerate}
\end{Exemple}