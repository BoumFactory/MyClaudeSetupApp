\begin{EXO}{Identification des coefficients}{}
\tcbitempoint{6} Parmi les fonctions ci-dessous, indiquer les fonctions polynômes de degré $2$, en précisant ses coefficients.
\begin{tcbenumerate}[3]
\tcbitem $f(x) = (x+3)^2$

\begin{crep}
$f(x) = x^2+6x+9$ donc oui, $a=1$, $b=6$, $c=9$
\end{crep}

\tcbitem $g(x)= (x+3)(x-3)$

\begin{crep}
$g(x) = x^2-9$ donc oui, $a=1$, $b=0$, $c=-9$
\end{crep}

\tcbitem $h(x)= (x+1)^2-(x-1)^2$

\begin{crep}
$h(x) = 4x$ donc non, degré 1
\end{crep}
\end{tcbenumerate}

\exocorrection

\begin{tcbenumerate}[3]
\tcbitem $f(x) = (x+3)^2 = x^2+6x+9$

C'est une fonction polynôme de degré 2 avec $a=1$, $b=6$, $c=9$.

\tcbitem $g(x)= (x+3)(x-3) = x^2-9$

C'est une fonction polynôme de degré 2 avec $a=1$, $b=0$, $c=-9$.

\tcbitem $h(x)= (x+1)^2-(x-1)^2$

Développons : $(x+1)^2 = x^2+2x+1$ et $(x-1)^2 = x^2-2x+1$

Donc $h(x) = (x^2+2x+1)-(x^2-2x+1) = x^2+2x+1-x^2+2x-1 = 4x$

$h$ est une fonction affine (degré 1), pas une fonction polynôme de degré 2.
\end{tcbenumerate}
\end{EXO}