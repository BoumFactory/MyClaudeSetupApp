\begin{EXO}{Forme canonique par lecture graphique}{}
Soit $f$ la fonction dont la représentation graphique est donnée ci-contre. 

\begin{MultiColonnes}{3}
    \tcbitem[raster multicolumn=2] \tcbitempoint{3} Déterminer la forme canonique de $f$.

\begin{crep}
- Sommet : $A(1;2)$ donc $\alpha = 1$ et $\beta = 2$

- Point $B(0;4)$ appartient à la courbe

Forme canonique : $f(x) = a(x-1)^2 + 2$

Puisque $f(0) = 4$, $a(0-1)^2 + 2 = 4$

D'où $a + 2 = 4 \Longrightarrow a = 2$ et $f(x) = 2(x-1)^2 + 2$
\end{crep}

    \tcbitem \begin{tikzpicture}[line cap=round,line join=round,>=triangle 45,x=1.0cm,y=1.0cm,scale=0.8]
\begin{axis}[
x=1.0cm,y=1.0cm,
axis lines=middle,
ymajorgrids=true,
xmajorgrids=true,
xmin=-1.5,
xmax=3.5,
ymin=-0.5,
ymax=5.5,
xtick={-1.0,0.0,...,3.0},
ytick={-0.0,1.0,...,5.0},]
\clip(-1.5,-0.5) rectangle (3.5,5.5);
\addplot[thick,monrose,samples=200] {2*(x-1)^2+2};
\begin{scriptsize}
\draw[color=monrose] (0.24,5) node {$\mathcal{C}_f$};
\draw [fill=black] (1.,2.) circle (3.0pt);
\draw[color=black] (0.9,2.55) node {$A$};
\draw [fill=black] (0.,4.) circle (3.0pt);
\draw[color=black] (0.34,4.43) node {$B$};
\end{scriptsize}
\end{axis}
\end{tikzpicture}
    
\end{MultiColonnes}

\exocorrection

En observant la représentation graphique, je peux identifier :

\textbf{1) Le sommet de la parabole :} 

Le point le plus bas de la courbe est $A(1;2)$, donc :

- $\alpha = 1$ (abscisse du sommet)

- $\beta = 2$ (ordonnée du sommet)

\textbf{2) La forme canonique partielle :}

$f(x) = a(x-\alpha)^2 + \beta = a(x-1)^2 + 2$

\textbf{3) Détermination du coefficient $a$ :}

La courbe passe par le point $B(0;4)$, donc $f(0) = 4$ :

$f(0) = a(0-1)^2 + 2 = a \times 1 + 2 = a + 2$

Puisque $f(0) = 4$ :
\begin{center}$a + 2 = 4 \Leftrightarrow a = 2$\end{center}

\textbf{4) Forme canonique finale :}

\begin{center}$f(x) = 2(x-1)^2 + 2$\end{center}

Vérification : $f(0) = 2(0-1)^2 + 2 = 2 \times 1 + 2 = 4$ ✓
\end{EXO}