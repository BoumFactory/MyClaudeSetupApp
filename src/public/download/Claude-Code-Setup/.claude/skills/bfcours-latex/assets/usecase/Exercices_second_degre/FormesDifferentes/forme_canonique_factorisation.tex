\begin{EXO}{Forme canonique par factorisation}{}
Soit $f$ la fonction définie sur $\R$ par $ f(x) = 2x^2+4x+8$.

\tcbitempoint{2} Déterminer la forme canonique de $f$.
\begin{crep}
$f(x) = 2(x+1)^2 + 6$
\end{crep}

\exocorrection

D'après le cours : 
L'écriture $a(x-\alpha)^{2} +\beta$ est la\voc{forme canonique} de la fonction $f:x\mapsto ax^{2} + bx + c$.

Pour $f(x) = 2x^2+4x+8$, on a $a=2$, $b=4$, $c=8$.

\begin{tcbenumerate}[2][1][alph]
    \tcbitem[halign=center] $\alpha = -\dfrac{b}{2a} = -\dfrac{4}{2 \times 2} = -1$
    \tcbitem[halign=center] $\beta = -\dfrac{b^2-4ac}{4a} = -\dfrac{4^2-4 \times 2 \times 8}{4 \times 2}$
    
    $\phantom{\beta}= -\dfrac{16-64}{8} = -\dfrac{-48}{8} = 6$
\end{tcbenumerate}

En remplaçant $a$, $\alpha$ et $\beta$  dans l'expression de la forme canonique, on obtient: 
$$f(x) = 2(x+1)^2 + 6$$
\end{EXO}