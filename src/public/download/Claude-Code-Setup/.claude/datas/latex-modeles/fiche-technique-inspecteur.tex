% Template de Fiche Technique - Format Inspecteur (Officiel)
% Utilisation : Document administratif destiné aux inspecteurs
% Style : Sobre, académique, professionnel avec en-tête officiel

\documentclass[11pt,a4paper]{article}

% ============================================
% PACKAGES
% ============================================
\usepackage[utf8]{inputenc}
\usepackage[T1]{fontenc}
\usepackage[french]{babel}
\usepackage{geometry}
\usepackage{graphicx}
\usepackage{xcolor}
\usepackage{fancyhdr}
\usepackage{titlesec}
\usepackage{enumitem}
\usepackage{hyperref}
\usepackage{tabularx}
\usepackage{array}
\usepackage{booktabs}

% ============================================
% CONFIGURATION DE LA PAGE
% ============================================
\geometry{
    a4paper,
    top=3.5cm,
    bottom=2.5cm,
    left=2.5cm,
    right=2.5cm,
    headheight=2.5cm,
    headsep=0.8cm
}

% ============================================
% COULEURS OFFICIELLES
% ============================================
\definecolor{bleu-rf}{RGB}{0,85,164}      % Bleu République Française
\definecolor{rouge-rf}{RGB}{239,65,53}    % Rouge République Française
\definecolor{gris-titre}{RGB}{50,50,50}   % Gris pour les titres
\definecolor{gris-clair}{RGB}{240,240,240} % Gris clair pour les encadrés

% ============================================
% EN-TÊTE ET PIED DE PAGE
% ============================================
\pagestyle{fancy}
\fancyhf{}

% En-tête avec logo République Française
\fancyhead[L]{%
    \begin{minipage}[c]{0.15\textwidth}
        \includegraphics[height=1.8cm]{C:/Users/Utilisateur/Documents/Professionnel/1. Reims 2025 - 2026/.claude/datas/latex-modeles/logo-republique-francaise.svg}
    \end{minipage}%
    \hspace{0.02\textwidth}%
    \begin{minipage}[c]{0.75\textwidth}
        \small
        \textbf{Académie de %ACADEMIE%}\\
        %TYPE_ETABLISSEMENT% %NOM_ETABLISSEMENT%\\
        \textit{Année scolaire %ANNEE_SCOLAIRE%}
    \end{minipage}
}

\fancyhead[R]{}

% Pied de page
\fancyfoot[C]{\small Page \thepage}
\fancyfoot[R]{\small \textit{Document confidentiel}}

% Ligne de séparation en-tête
\renewcommand{\headrulewidth}{0.5pt}
\renewcommand{\footrulewidth}{0pt}

% ============================================
% STYLE DES TITRES
% ============================================
\titleformat{\section}
    {\Large\bfseries\color{bleu-rf}}
    {\thesection}{1em}{}
    [\vspace{-0.5em}\titlerule]

\titleformat{\subsection}
    {\large\bfseries\color{gris-titre}}
    {\thesubsection}{1em}{}

\titleformat{\subsubsection}
    {\normalsize\bfseries\color{gris-titre}}
    {\thesubsubsection}{1em}{}

% Espacement des titres
\titlespacing*{\section}{0pt}{1.5ex plus 1ex minus .2ex}{1ex plus .2ex}
\titlespacing*{\subsection}{0pt}{1.2ex plus 1ex minus .2ex}{0.8ex plus .2ex}

% ============================================
% STYLE DES LISTES
% ============================================
\setlist[itemize,1]{label=\textcolor{bleu-rf}{$\bullet$}, leftmargin=1.5em}
\setlist[itemize,2]{label=\textcolor{gris-titre}{--}, leftmargin=2em}
\setlist[enumerate]{leftmargin=1.5em}

% ============================================
% ENVIRONNEMENTS PERSONNALISÉS
% ============================================

% Encadré pour informations importantes
\newenvironment{encadre}{%
    \begin{center}
    \begin{tabular}{|p{0.95\textwidth}|}
    \hline
    \rowcolor{gris-clair}
}{%
    \\
    \hline
    \end{tabular}
    \end{center}
}

% Tableau d'informations clés
\newenvironment{infoscles}{%
    \begin{center}
    \begin{tabularx}{0.95\textwidth}{>{\bfseries}l X}
    \toprule
}{%
    \bottomrule
    \end{tabularx}
    \end{center}
}

% ============================================
% CONFIGURATION HYPERREF
% ============================================
\hypersetup{
    colorlinks=true,
    linkcolor=bleu-rf,
    urlcolor=bleu-rf,
    citecolor=bleu-rf,
    pdfborder={0 0 0}
}

% ============================================
% DÉBUT DU DOCUMENT
% ============================================
\begin{document}

% ============================================
% PAGE DE TITRE
% ============================================
\begin{center}
    \vspace*{1cm}

    {\LARGE\bfseries\color{bleu-rf} FICHE TECHNIQUE PÉDAGOGIQUE}

    \vspace{0.5cm}
    \rule{0.8\textwidth}{1pt}
    \vspace{0.5cm}

    {\Large %TITRE_RESSOURCE%}

    \vspace{0.3cm}
    {\large %NIVEAU%}

    \vspace{1cm}
\end{center}

% Informations clés en encadré
\begin{encadre}
    \begin{infoscles}
        Type de ressource & %TYPE_RESSOURCE% \\
        \midrule
        Niveau & %NIVEAU_DETAILLE% \\
        \midrule
        Thème & %THEME% \\
        \midrule
        Durée estimée & %DUREE% \\
        \midrule
        Auteur & %AUTEUR% \\
        \midrule
        Date & %DATE% \\
    \end{infoscles}
\end{encadre}

\vspace{1cm}

% ============================================
% SECTION 1 : PRÉSENTATION DE LA RESSOURCE
% ============================================
\section{Présentation de la ressource}

%PRESENTATION%

% ============================================
% SECTION 2 : OBJECTIFS PÉDAGOGIQUES
% ============================================
\section{Objectifs pédagogiques}

\subsection{Connaissances}
%CONNAISSANCES%

\subsection{Compétences}
%COMPETENCES%

\subsection{Capacités travaillées}
%CAPACITES%

% ============================================
% SECTION 3 : LIEN AVEC LES PROGRAMMES OFFICIELS
% ============================================
\section{Inscription dans les programmes officiels}

\subsection{Références au Bulletin Officiel}
%REFERENCES_BO%

\subsection{Compétences du socle commun}
\textit{(Pour le collège uniquement)}

%COMPETENCES_SOCLE%

\subsection{Compétences mathématiques travaillées}
%COMPETENCES_MATHS%

% ============================================
% SECTION 4 : ACTIVITÉS PRÉVUES DES ÉLÈVES
% ============================================
\section{Activités prévues des élèves}

%ACTIVITES_ELEVES%

% ============================================
% SECTION 5 : PRÉREQUIS
% ============================================
\section{Prérequis}

\subsection{Prérequis mathématiques}
%PREREQUIS_MATHS%

\subsection{Prérequis méthodologiques}
%PREREQUIS_METHODO%

% ============================================
% SECTION 6 : MODALITÉS DE MISE EN ŒUVRE
% ============================================
\section{Modalités de mise en œuvre}

\begin{infoscles}
    Durée totale & %DUREE_TOTALE% \\
    \midrule
    Organisation & %ORGANISATION% \\
    \midrule
    Matériel nécessaire & %MATERIEL% \\
    \midrule
    Supports & %SUPPORTS% \\
\end{infoscles}

\subsection{Déroulement proposé}
%DEROULEMENT%

% ============================================
% SECTION 7 : CHOIX DIDACTIQUES ET PÉDAGOGIQUES
% ============================================
\section{Choix didactiques et pédagogiques}

\subsection{Justification des choix pédagogiques}
%JUSTIFICATION_PEDAGOGIQUE%

\subsection{Différenciation}
%DIFFERENCIATION%

\subsection{Difficultés anticipées}
%DIFFICULTES%

% ============================================
% SECTION 8 : ÉVALUATION
% ============================================
\section{Évaluation des acquis}

\subsection{Modalités d'évaluation}
%MODALITES_EVALUATION%

\subsection{Critères de réussite}
%CRITERES_REUSSITE%

% ============================================
% SECTION 9 : PROLONGEMENTS
% ============================================
\section{Prolongements possibles}

%PROLONGEMENTS%

% ============================================
% SECTION 10 : BIBLIOGRAPHIE ET RESSOURCES
% ============================================
\section{Bibliographie et ressources}

%BIBLIOGRAPHIE%

\end{document}
