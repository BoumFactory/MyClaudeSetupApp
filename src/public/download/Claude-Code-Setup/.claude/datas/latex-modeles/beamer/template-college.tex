%% Template Beamer pour collège (6e - 3e)
%% Style : Couleurs vives, police grande, animations fréquentes
%% Densité maximale : 60%

\documentclass[14pt, xcolor={svgnames}]{beamer}

% === THÈME ET COULEURS ===
\usetheme{Boadilla}
\usecolortheme{whale}

% Couleurs vives adaptées au collège
\definecolor{college-main}{RGB}{41, 128, 185}      % Bleu vif
\definecolor{college-accent}{RGB}{243, 156, 18}    % Orange
\definecolor{college-alert}{RGB}{231, 76, 60}      % Rouge vif
\definecolor{college-example}{RGB}{46, 204, 113}   % Vert

\setbeamercolor{structure}{fg=college-main}
\setbeamercolor{block title}{bg=college-main, fg=white}
\setbeamercolor{block body}{bg=college-main!10}
\setbeamercolor{block title example}{bg=college-example, fg=white}
\setbeamercolor{block body example}{bg=college-example!10}
\setbeamercolor{block title alerted}{bg=college-alert, fg=white}
\setbeamercolor{block body alerted}{bg=college-alert!10}

% === PACKAGES ESSENTIELS ===
\usepackage[french]{babel}
\usepackage[utf8]{inputenc}
\usepackage[T1]{fontenc}
\usepackage{amsmath, amssymb}
\usepackage{tikz}
\usepackage{graphicx}

% Packages TikZ pour graphiques
\usetikzlibrary{calc, positioning, shapes, arrows, decorations.pathreplacing}

% Package pour les encadrés d'exercices
\usepackage{tcolorbox}
\tcbuselibrary{skins, breakable}

% === ENVIRONNEMENT EXERCICE ===
\newtcolorbox{exobeamer}[1][]{
  enhanced,
  breakable,
  colback=college-accent!10,
  colframe=college-accent,
  coltitle=white,
  fonttitle=\bfseries\large,
  title={🎯 Exercice},
  attach boxed title to top left={yshift=-3mm, xshift=5mm},
  boxed title style={
    colback=college-accent,
    sharp corners,
    boxrule=0pt,
  },
  top=6mm,
  bottom=4mm,
  left=4mm,
  right=4mm,
  overlay unbroken and first={
    % Estimation (en haut à gauche sous le titre)
    \node[anchor=north west, font=\small\itshape, text=college-accent!80!black]
      at ([xshift=5mm, yshift=-12mm]frame.north west) {#1};
    % Zone modifiable (en haut à droite)
    \node[anchor=north east, font=\small, text=red!70!black, draw=red!70!black,
          fill=white, inner sep=2pt, minimum width=3.5cm]
      at ([xshift=-5mm, yshift=-12mm]frame.north east) {Temps réel : \underline{\hspace{2cm}}};
  }
}

% === MISE EN PAGE ===
% Espacement des listes
\setlength{\leftmargini}{1.5em}
\setlength{\parskip}{0.5em}

% Numérotation des slides
\setbeamertemplate{footline}[frame number]

% Pas de symboles de navigation
\setbeamertemplate{navigation symbols}{}

% === MÉTADONNÉES (À REMPLIR) ===
\title{{{TITRE_PRESENTATION}}}
\subtitle{{{SOUS_TITRE}}}
\author{{{AUTEUR}}}
\date{{{DATE}}}
\institute{{{ETABLISSEMENT}}}

% === DÉBUT DU DOCUMENT ===
\begin{document}

% === PAGE DE TITRE ===
\begin{frame}
  \titlepage
\end{frame}

% === PLAN (optionnel, pour présentations > 12 slides) ===
% \begin{frame}{Plan}
%   \tableofcontents
% \end{frame}

% === SECTION 1 : EXEMPLE ===
\section{Introduction}

\begin{frame}{Titre de la première slide}
  \begin{itemize}[<+->]  % Animation item par item
    \item Premier point important
    \item Deuxième point
    \item Troisième point
  \end{itemize}

  \pause

  \begin{exampleblock}{Exemple}
    Voici un exemple concret...
  \end{exampleblock}
\end{frame}

\begin{frame}{Définition}
  \begin{block}{À retenir}
    Une définition importante à mémoriser.
  \end{block}

  \pause
  \vspace{1em}

  \textbf{Illustration :}

  \begin{center}
    \begin{tikzpicture}[scale=1.2]
      % Exemple de graphique simple
      \draw[<->, thick] (-3,0) -- (3,0) node[right] {$x$};
      \foreach \x in {-2,-1,0,1,2}
        \draw (\x, -0.1) -- (\x, 0.1) node[above] {\x};
    \end{tikzpicture}
  \end{center}
\end{frame}

% === SECTION 2 : EXERCICE ===
\section{Application}

\begin{frame}{À toi de jouer !}
  \begin{exobeamer}[Estimation : 5 min | Difficulté : ★☆☆]
    \textbf{Énoncé :}

    Calculer : $3 + 7 = ?$

    \pause
    \vspace{1em}

    \textbf{Solution :}

    \uncover<2->{
      $3 + 7 = 10$
    }
  \end{exobeamer}
\end{frame}

% === QUESTION INTERACTIVE ===
\begin{frame}{Vérifions ensemble}
  \textbf{Question :} Que vaut $2 \times 5$ ?

  \pause
  \vspace{2em}

  \begin{center}
    {\Huge \textcolor{college-main}{$2 \times 5 = 10$}}
  \end{center}

  \pause

  \textbf{Bravo !} Vous avez bien compris.
\end{frame}

% === SYNTHÈSE ===
\section{Conclusion}

\begin{frame}{Ce qu'il faut retenir}
  \begin{alertblock}{Points clés}
    \begin{itemize}
      \item Point important n°1
      \item Point important n°2
      \item Point important n°3
    \end{itemize}
  \end{alertblock}

  \vfill

  \begin{center}
    \textcolor{college-accent}{\textbf{Bon travail !}}
  \end{center}
\end{frame}

% === FIN ===
\begin{frame}
  \begin{center}
    {\Huge Merci pour votre attention !}

    \vspace{2em}

    {\large Des questions ?}
  \end{center}
\end{frame}

\end{document}
