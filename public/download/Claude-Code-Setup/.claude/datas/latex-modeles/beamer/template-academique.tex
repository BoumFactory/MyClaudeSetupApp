%% Template Beamer pour présentations académiques
%% Style : Sobre, professionnel, haute densité d'information maîtrisée
%% Densité maximale : 70%

\documentclass[11pt, aspectratio=169, xcolor={svgnames}]{beamer}

% === THÈME ET COULEURS ===
\usetheme{default}
\usecolortheme{dove}

% Couleurs sobres académiques
\definecolor{acad-main}{RGB}{0, 0, 0}             % Noir
\definecolor{acad-accent}{RGB}{51, 51, 51}        % Gris foncé
\definecolor{acad-light}{RGB}{245, 245, 245}      % Gris très clair
\definecolor{acad-blue}{RGB}{0, 51, 102}          % Bleu nuit (optionnel)

\setbeamercolor{structure}{fg=acad-main}
\setbeamercolor{block title}{bg=acad-accent, fg=white}
\setbeamercolor{block body}{bg=acad-light}
\setbeamercolor{alerted text}{fg=acad-accent}

% === PACKAGES ESSENTIELS ===
\usepackage[french]{babel}
\usepackage[utf8]{inputenc}
\usepackage[T1]{fontenc}
\usepackage{amsmath, amssymb, amsthm}
\usepackage{mathtools}
\usepackage{tikz}
\usepackage{graphicx}
\usepackage{pgfplots}
\pgfplotsset{compat=1.18}
\usepackage{booktabs}  % Pour des tableaux professionnels

% Packages TikZ pour graphiques techniques
\usetikzlibrary{calc, positioning, shapes, arrows, decorations.pathreplacing, patterns, 3d}

% Package pour les encadrés
\usepackage{tcolorbox}
\tcbuselibrary{skins, breakable, theorems}

% Bibliographie (optionnel)
% \usepackage[style=authoryear, backend=biber]{biblatex}
% \addbibresource{references.bib}

% === ENVIRONNEMENT EXERCICE (optionnel pour académique) ===
\newtcolorbox{exobeamer}[1][]{
  enhanced,
  breakable,
  colback=acad-light,
  colframe=acad-accent,
  coltitle=white,
  fonttitle=\bfseries,
  title={Application},
  attach boxed title to top left={yshift=-2mm, xshift=5mm},
  boxed title style={
    colback=acad-accent,
    sharp corners,
    boxrule=0pt,
  },
  top=4mm,
  bottom=2mm,
  left=2mm,
  right=2mm,
  overlay unbroken and first={
    \node[anchor=north west, font=\scriptsize\itshape, text=acad-accent!80!black]
      at ([xshift=5mm, yshift=-8mm]frame.north west) {#1};
  }
}

% === THÉORÈMES ET DÉFINITIONS ===
\newtheorem{theoreme}{Théorème}
\newtheorem{proposition}{Proposition}
\newtheorem{lemme}{Lemme}
\newtheorem{corollaire}{Corollaire}

\theoremstyle{definition}
\newtheorem{definition}{Définition}
\newtheorem{remarque}{Remarque}

% === MISE EN PAGE ===
% Espacement des listes (compact)
\setlength{\leftmargini}{1.2em}
\setlength{\parskip}{0.2em}

% En-tête et pied de page professionnel
\setbeamertemplate{footline}{
  \hbox{%
    \begin{beamercolorbox}[wd=.33\paperwidth,ht=2.25ex,dp=1ex,left]{author in head/foot}%
      \hspace*{1em}\insertshortauthor
    \end{beamercolorbox}%
    \begin{beamercolorbox}[wd=.34\paperwidth,ht=2.25ex,dp=1ex,center]{title in head/foot}%
      \insertshorttitle
    \end{beamercolorbox}%
    \begin{beamercolorbox}[wd=.33\paperwidth,ht=2.25ex,dp=1ex,right]{date in head/foot}%
      \insertshortdate{}\hspace*{2em}
      \insertframenumber{} / \inserttotalframenumber\hspace*{2ex}
    \end{beamercolorbox}%
  }
  \vskip0pt%
}

% Pas de symboles de navigation
\setbeamertemplate{navigation symbols}{}

% Numérotation des équations
\setbeamertemplate{theorem}[numbered]

% === MÉTADONNÉES (À REMPLIR) ===
\title{{{TITRE_PRESENTATION}}}
\subtitle{{{SOUS_TITRE}}}
\author{{{AUTEUR}}}
\date{{{DATE}}}
\institute{%
  {{ETABLISSEMENT}} \\
  {{LABORATOIRE}} \\
  {{VILLE}}, {{PAYS}}
}

% Versions courtes
\def\insertshortauthor{{{AUTEUR_COURT}}}
\def\insertshorttitle{{{TITRE_COURT}}}
\def\insertshortdate{{{DATE_COURTE}}}

% === DÉBUT DU DOCUMENT ===
\begin{document}

% === PAGE DE TITRE ===
\begin{frame}[plain]
  \titlepage
\end{frame}

% === PLAN ===
\begin{frame}{Plan}
  \tableofcontents
\end{frame}

% === SECTION 1 : CONTEXTE ===
\section{Contexte et motivation}

\begin{frame}{Contexte}
  \begin{block}{Problématique}
    Présentation du problème scientifique ou mathématique étudié.
  \end{block}

  \vspace{0.5em}

  \textbf{État de l'art :}
  \begin{itemize}
    \item Travaux de \cite{reference1} en 2020
    \item Approche de \cite{reference2} limitée par...
    \item Notre contribution : extension à...
  \end{itemize}
\end{frame}

\begin{frame}{Objectifs}
  \textbf{Objectifs de cette présentation :}

  \begin{enumerate}
    \item Présenter le cadre théorique
    \item Démontrer les résultats principaux
    \item Analyser les applications
    \item Discuter des perspectives
  \end{enumerate}
\end{frame}

% === SECTION 2 : CADRE THÉORIQUE ===
\section{Cadre théorique}

\begin{frame}{Définitions}
  \begin{definition}[Espace de Hilbert]
    Un espace de Hilbert $\mathcal{H}$ est un espace vectoriel muni d'un produit scalaire $\langle \cdot, \cdot \rangle$ complet pour la norme induite.
  \end{definition}

  \vspace{0.5em}

  \begin{definition}[Opérateur compact]
    Un opérateur $T : \mathcal{H} \to \mathcal{H}$ est compact si l'image de toute boule bornée est relativement compacte.
  \end{definition}
\end{frame}

\begin{frame}{Théorème principal}
  \begin{theoreme}[Spectral pour opérateurs compacts]
    Soit $T : \mathcal{H} \to \mathcal{H}$ un opérateur compact auto-adjoint. Alors :
    \begin{enumerate}
      \item Le spectre de $T$ est discret
      \item Il existe une base orthonormée de vecteurs propres
      \item Les valeurs propres tendent vers zéro
    \end{enumerate}
  \end{theoreme}

  \vspace{0.5em}

  \begin{proof}[Esquisse]
    Par analyse spectrale et compacité... (détails dans \cite{reference3})
  \end{proof}
\end{frame}

% === SECTION 3 : RÉSULTATS ===
\section{Résultats principaux}

\begin{frame}{Résultat 1}
  \begin{proposition}
    Sous les hypothèses (H1)-(H3), le problème admet une solution unique.
  \end{proposition}

  \textbf{Démonstration :}

  On procède en trois étapes :
  \begin{enumerate}
    \item Existence par méthode variationnelle
    \item Unicité par stricte convexité
    \item Régularité par bootstrap
  \end{enumerate}
\end{frame}

\begin{frame}{Illustration numérique}
  \begin{columns}[T]
    \begin{column}{0.48\textwidth}
      \textbf{Configuration :}
      \begin{itemize}
        \item Maillage : $N = 10^4$
        \item Schéma : EF P1
        \item Solver : GMRES
      \end{itemize}
    \end{column}

    \begin{column}{0.48\textwidth}
      \begin{center}
        \includegraphics[width=\textwidth]{graphique_resultats.pdf}
      \end{center}
      \scriptsize Figure 1 : Convergence du schéma
    \end{column}
  \end{columns}
\end{frame}

\begin{frame}{Tableau de résultats}
  \begin{table}
    \centering
    \caption{Erreurs numériques en norme $L^2$}
    \begin{tabular}{@{}ccccc@{}}
      \toprule
      $N$ & Erreur & Ordre & Temps (s) & Mémoire (MB) \\
      \midrule
      $10^2$ & $2.3 \times 10^{-2}$ & -- & 0.12 & 5 \\
      $10^3$ & $2.5 \times 10^{-3}$ & 1.96 & 1.45 & 48 \\
      $10^4$ & $2.6 \times 10^{-4}$ & 1.98 & 18.3 & 512 \\
      \bottomrule
    \end{tabular}
  \end{table}
\end{frame}

% === SECTION 4 : DISCUSSION ===
\section{Discussion}

\begin{frame}{Comparaison avec l'existant}
  \begin{table}
    \centering
    \small
    \begin{tabular}{lccc}
      \toprule
      Méthode & Complexité & Précision & Stabilité \\
      \midrule
      \cite{reference1} & $O(N^2)$ & $O(h^2)$ & Conditionnelle \\
      \cite{reference2} & $O(N \log N)$ & $O(h)$ & Inconditionnelle \\
      \textbf{Notre méthode} & $\mathbf{O(N)}$ & $\mathbf{O(h^2)}$ & \textbf{Inconditionnelle} \\
      \bottomrule
    \end{tabular}
  \end{table}

  \vspace{0.5em}

  $\Rightarrow$ Amélioration significative en complexité tout en conservant l'ordre optimal.
\end{frame}

% === CONCLUSION ===
\section{Conclusion}

\begin{frame}{Conclusions}
  \textbf{Résumé des contributions :}

  \begin{itemize}
    \item Nouveau cadre théorique unifié
    \item Démonstration de convergence optimale
    \item Implémentation efficace et validation
  \end{itemize}

  \vspace{1em}

  \textbf{Perspectives :}

  \begin{itemize}
    \item Extension au cas non-linéaire
    \item Analyse en dimension supérieure
    \item Applications industrielles
  \end{itemize}
\end{frame}

% === RÉFÉRENCES ===
\begin{frame}[allowframebreaks]{Références}
  \footnotesize
  \begin{thebibliography}{99}
    \bibitem{reference1} A. Auteur, B. Collaborateur,
      \emph{Titre de l'article},
      Journal Name, vol. 42, pp. 123-145, 2020.

    \bibitem{reference2} C. Chercheur,
      \emph{Monographie importante},
      Éditeur, 2021.

    \bibitem{reference3} D. Expert, E. Spécialiste,
      \emph{Another relevant work},
      Conference Proceedings, pp. 567-589, 2022.
  \end{thebibliography}
\end{frame}

% === FIN ===
\begin{frame}[plain]
  \begin{center}
    \vspace{2em}
    {\Large Merci pour votre attention}

    \vspace{2em}

    {\normalsize Questions et discussion}

    \vspace{3em}

    \scriptsize
    Contact : {{EMAIL}} \\
    Slides disponibles sur : {{URL}}
  \end{center}
\end{frame}

\end{document}
