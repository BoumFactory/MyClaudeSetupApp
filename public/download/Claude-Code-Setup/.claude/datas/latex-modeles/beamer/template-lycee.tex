%% Template Beamer pour lycée (2nde - Terminale)
%% Style : Équilibre entre rigueur et accessibilité
%% Densité maximale : 70%

\documentclass[12pt, xcolor={svgnames}]{beamer}

% === THÈME ET COULEURS ===
\usetheme{Madrid}
\usecolortheme{default}

% Couleurs équilibrées pour le lycée
\definecolor{lycee-main}{RGB}{0, 51, 102}          % Bleu marine
\definecolor{lycee-accent}{RGB}{153, 0, 0}         % Bordeaux
\definecolor{lycee-light}{RGB}{230, 230, 250}      % Lavande
\definecolor{lycee-example}{RGB}{0, 102, 51}       % Vert foncé

\setbeamercolor{structure}{fg=lycee-main}
\setbeamercolor{block title}{bg=lycee-main, fg=white}
\setbeamercolor{block body}{bg=lycee-light}
\setbeamercolor{block title example}{bg=lycee-example, fg=white}
\setbeamercolor{block body example}{bg=lycee-example!10}
\setbeamercolor{block title alerted}{bg=lycee-accent, fg=white}
\setbeamercolor{block body alerted}{bg=lycee-accent!10}

% === PACKAGES ESSENTIELS ===
\usepackage[french]{babel}
\usepackage[utf8]{inputenc}
\usepackage[T1]{fontenc}
\usepackage{amsmath, amssymb, amsthm}
\usepackage{mathtools}
\usepackage{tikz}
\usepackage{graphicx}
\usepackage{pgfplots}
\pgfplotsset{compat=1.18}

% Packages TikZ pour graphiques avancés
\usetikzlibrary{calc, positioning, shapes, arrows, decorations.pathreplacing, patterns}

% Package pour les encadrés d'exercices
\usepackage{tcolorbox}
\tcbuselibrary{skins, breakable, theorems}

% === ENVIRONNEMENT EXERCICE ===
\newtcolorbox{exobeamer}[1][]{
  enhanced,
  breakable,
  colback=lycee-light,
  colframe=lycee-main,
  coltitle=white,
  fonttitle=\bfseries,
  title={Exercice},
  attach boxed title to top left={yshift=-2.5mm, xshift=5mm},
  boxed title style={
    colback=lycee-main,
    sharp corners,
    boxrule=0pt,
  },
  top=5mm,
  bottom=3mm,
  left=3mm,
  right=3mm,
  overlay unbroken and first={
    % Estimation
    \node[anchor=north west, font=\small\itshape, text=lycee-main!80!black]
      at ([xshift=5mm, yshift=-10mm]frame.north west) {#1};
    % Zone modifiable
    \node[anchor=north east, font=\small, text=red!70!black, draw=red!70!black,
          fill=white, inner sep=2pt, minimum width=3cm]
      at ([xshift=-5mm, yshift=-10mm]frame.north east) {Temps réel : \underline{\hspace{1.5cm}}};
  }
}

% === THÉORÈMES ET DÉFINITIONS ===
\newtheorem{theoreme}{Théorème}
\newtheorem{propriete}{Propriété}
\newtheorem{lemme}{Lemme}
\newtheorem{corollaire}{Corollaire}

\theoremstyle{definition}
\newtheorem{definition}{Définition}
\newtheorem{exemple}{Exemple}

% === MISE EN PAGE ===
% Espacement des listes
\setlength{\leftmargini}{1.5em}
\setlength{\parskip}{0.3em}

% En-tête et pied de page
\setbeamertemplate{footline}{
  \hbox{%
    \begin{beamercolorbox}[wd=.5\paperwidth,ht=2.5ex,dp=1ex,left]{author in head/foot}%
      \hspace*{1em}\insertshortauthor
    \end{beamercolorbox}%
    \begin{beamercolorbox}[wd=.5\paperwidth,ht=2.5ex,dp=1ex,right]{title in head/foot}%
      \insertshorttitle\hspace*{1em}
      \insertframenumber{} / \inserttotalframenumber\hspace*{1em}
    \end{beamercolorbox}%
  }
  \vskip0pt%
}

% Pas de symboles de navigation
\setbeamertemplate{navigation symbols}{}

% === MÉTADONNÉES (À REMPLIR) ===
\title{{{TITRE_PRESENTATION}}}
\subtitle{{{SOUS_TITRE}}}
\author{{{AUTEUR}}}
\date{{{DATE}}}
\institute{{{ETABLISSEMENT}}}

% Versions courtes pour l'en-tête
\def\insertshortauthor{{{AUTEUR_COURT}}}
\def\insertshorttitle{{{TITRE_COURT}}}

% === DÉBUT DU DOCUMENT ===
\begin{document}

% === PAGE DE TITRE ===
\begin{frame}
  \titlepage
\end{frame}

% === PLAN ===
\begin{frame}{Plan}
  \tableofcontents
\end{frame}

% === SECTION 1 : RAPPELS ===
\section{Rappels et prérequis}

\begin{frame}{Rappels}
  \begin{block}{Notions à connaître}
    \begin{itemize}
      \item Notion 1 vue en classe précédente
      \item Notion 2 essentielle pour aujourd'hui
      \item Notion 3 à ne pas oublier
    \end{itemize}
  \end{block}

  \pause

  \begin{exampleblock}{Exemple de rappel}
    Si $f(x) = x^2$, alors $f'(x) = 2x$
  \end{exampleblock}
\end{frame}

% === SECTION 2 : COURS ===
\section{Nouvelle notion}

\begin{frame}{Définition}
  \begin{definition}
    On appelle \emph{fonction dérivée} de $f$ la fonction $f'$ définie par :
    \[
      f'(x) = \lim_{h \to 0} \frac{f(x+h) - f(x)}{h}
    \]
  \end{definition}

  \pause

  \begin{alertblock}{Condition d'existence}
    Cette limite doit exister et être finie.
  \end{alertblock}
\end{frame}

\begin{frame}{Théorème principal}
  \begin{theoreme}[Dérivée des fonctions puissances]
    Soit $n \in \mathbb{N}^*$. La fonction $f : x \mapsto x^n$ est dérivable sur $\mathbb{R}$ et :
    \[
      \forall x \in \mathbb{R}, \quad f'(x) = nx^{n-1}
    \]
  \end{theoreme}

  \pause

  \begin{proof}[Esquisse de démonstration]
    Par définition de la dérivée et développement du binôme...
  \end{proof}
\end{frame}

\begin{frame}{Illustration graphique}
  \begin{center}
    \begin{tikzpicture}
      \begin{axis}[
        axis lines=middle,
        xlabel=$x$,
        ylabel=$y$,
        domain=-2:2,
        samples=100,
        grid=major,
        width=9cm,
        height=6cm,
        legend pos=north west
      ]
        \addplot[blue, thick] {x^2};
        \addlegendentry{$f(x) = x^2$}

        \addplot[red, thick, dashed] {2*x};
        \addlegendentry{$f'(x) = 2x$}
      \end{axis}
    \end{tikzpicture}
  \end{center}
\end{frame}

% === SECTION 3 : MÉTHODES ===
\section{Méthodes et exemples}

\begin{frame}{Méthode : Calculer une dérivée}
  \textbf{Étapes :}

  \begin{enumerate}
    \item<2-> Identifier la forme de la fonction
    \item<3-> Appliquer la formule de dérivation
    \item<4-> Simplifier l'expression obtenue
    \item<5-> Vérifier le résultat
  \end{enumerate}

  \vspace{1em}

  \uncover<6->{
    \begin{exampleblock}{Application}
      Calculons la dérivée de $f(x) = 3x^4 - 2x^2 + 5$
    \end{exampleblock}
  }
\end{frame}

\begin{frame}{Exemple détaillé}
  \textbf{Calculer la dérivée de} $f(x) = 3x^4 - 2x^2 + 5$

  \vspace{1em}

  \begin{align*}
    f'(x) &= \uncover<2->{(3x^4)' - (2x^2)' + (5)'} \\[0.5em]
          &= \uncover<3->{3 \times 4x^3 - 2 \times 2x + 0} \\[0.5em]
          &= \uncover<4->{12x^3 - 4x}
  \end{align*}

  \vspace{1em}

  \uncover<5->{
    \textbf{Conclusion :} $f'(x) = 12x^3 - 4x$
  }
\end{frame}

% === SECTION 4 : EXERCICES ===
\section{Exercices d'application}

\begin{frame}{Exercice guidé}
  \begin{exobeamer}[Estimation : 8 min | Difficulté : ★★☆]
    \textbf{Énoncé :}

    Étudier les variations de $f(x) = x^3 - 3x + 1$ sur $\mathbb{R}$.

    \pause

    \textbf{Méthode :}
    \begin{enumerate}
      \item<3-> Calculer $f'(x)$ : $f'(x) = 3x^2 - 3 = 3(x^2 - 1)$
      \item<4-> Factoriser : $f'(x) = 3(x-1)(x+1)$
      \item<5-> Tableau de signes et variations...
    \end{enumerate}
  \end{exobeamer}
\end{frame}

\begin{frame}{Exercice d'application}
  \begin{exobeamer}[Estimation : 5 min | Difficulté : ★☆☆]
    \textbf{À vous de jouer :}

    Calculer la dérivée de :
    \begin{enumerate}
      \item $g(x) = 5x^3 + 2x$
      \item $h(x) = x^4 - 7x^2 + 3$
    \end{enumerate}

    \pause
    \vspace{0.5em}

    \textbf{Solutions :}
    \begin{enumerate}
      \item<3-> $g'(x) = 15x^2 + 2$
      \item<4-> $h'(x) = 4x^3 - 14x$
    \end{enumerate}
  \end{exobeamer}
\end{frame}

% === SYNTHÈSE ===
\section{Conclusion}

\begin{frame}{Ce qu'il faut retenir}
  \begin{block}{Points clés du chapitre}
    \begin{itemize}
      \item Définition de la dérivée : limite du taux d'accroissement
      \item Formule : $(x^n)' = nx^{n-1}$
      \item Application : étude de variations
    \end{itemize}
  \end{block}

  \pause

  \begin{alertblock}{À savoir par cœur}
    \begin{center}
      \begin{tabular}{c|c}
        Fonction & Dérivée \\
        \hline
        $x^n$ & $nx^{n-1}$ \\
        $k$ & $0$ \\
      \end{tabular}
    \end{center}
  \end{alertblock}
\end{frame}

% === FIN ===
\begin{frame}
  \begin{center}
    {\Large Merci pour votre attention}

    \vspace{2em}

    {\normalsize Questions ?}
  \end{center}
\end{frame}

\end{document}
