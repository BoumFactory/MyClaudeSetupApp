\documentclass[a4paper,11pt,fleqn]{article}

\usepackage[left=1cm,right=1cm,top=0.5cm,bottom=2cm]{geometry}

\usepackage{bfcours}
\usepackage{bfcours-fonts}
%\usepackage{bfcours-fonts-dys}

\def\rdifficulty{1}
\setrdexo{%left skip=1cm,
display exotitle,
exo header = tcolorbox,
%display tags,
skin = bouyachakka,
lower ={box=crep},
display score,
display level,
save lower,
score=\points,
level=\rdifficulty,
overlay={\node[inner sep=0pt,
anchor=west,rotate=90, yshift=0.3cm]%,xshift=-3em], yshift=0.45cm
at (frame.south west) {\thetags[0]} ;}
]%obligatoire
}

\setrdcrep{correction=false, correction color=prop, correction font = \large\bfseries}
\edef\currentseyesoption{\ifcorrection false\else true\fi}
\setrdcrep{seyes=\currentseyesoption}

\newcommand{\tikzinclude}[1]{%
    \stepcounter{tikzfigcounter}%
    \csname tikzfig#1\endcsname
}
\newcommand{\tikzfigTnVi}{
\begin{tikzpicture}[line cap=round,line join=round,>=triangle 45,x=1.0cm,y=1.0cm,scale=0.6]
\begin{axis}[
x=1.0cm,y=1.0cm,
axis lines=middle,
ymajorgrids=true,
xmajorgrids=true,
xmin=-1.5,
xmax=3.5,
ymin=-3.2,
ymax=3.2,
xtick={-1.0,0.0,...,3.0},
ytick={-3.0,-2.0,...,3.0},]
\clip(-1.5,-3.2) rectangle (3.5,3.2);
\draw[line width=2.pt,color=blue,smooth,samples=100,domain=-1.5:3.5] plot(\x,{-0.6*((\x)-1.0)^(2.0)+2.0});
\begin{scriptsize}
\draw[color=blue] (-1,1) node {\large{$\mathcal{C}_f$}};
\end{scriptsize}
\end{axis}
\end{tikzpicture}
}

\newcommand{\tikzfigXWyE}{
\begin{tikzpicture}[line cap=round,line join=round,>=triangle 45,x=1.0cm,y=1.0cm,scale=0.6]
\begin{axis}[
x=1.0cm,y=1.0cm,
axis lines=middle,
ymajorgrids=true,
xmajorgrids=true,
xmin=-1.5,
xmax=3.5,
ymin=-3.2,
ymax=3.2,
xtick={-1.0,0.0,...,3.0},
ytick={-3.0,-2.0,...,3.0},]
\clip(-1.5,-3.2) rectangle (3.5,3.2);
\draw[line width=2.pt,color=blue,smooth,samples=100,domain=-1.5:3.5] plot(\x,{0.5*((\x)-2.0)^(2.0)-3.0});
\begin{scriptsize}
\draw[color=blue] (-1,2.5) node {\large{$\mathcal{C}_g$}};
\end{scriptsize}
\end{axis}
\end{tikzpicture}
}

\newcommand{\tikzfigbxUA}{
\begin{tikzpicture}[line cap=round,line join=round,>=triangle 45,x=1.0cm,y=1.0cm,scale=0.6]
\begin{axis}[
x=1.0cm,y=1.0cm,
axis lines=middle,
ymajorgrids=true,
xmajorgrids=true,
xmin=-1.5,
xmax=3.5,
ymin=-3.2,
ymax=3.2,
xtick={-1.0,0.0,...,3.0},
ytick={-3.0,-2.0,...,3.0},]
\clip(-1.5,-3.2) rectangle (3.5,3.2);
\draw[line width=2.pt,color=blue,smooth,samples=100,domain=-1.5:3.5] plot(\x,{0-0.3*((\x)-1.0)^(2.0)-1.0});
\begin{scriptsize}
\draw[color=blue] (2.5,-1) node {\large{$\mathcal{C}_h$}};
\end{scriptsize}
\end{axis}
\end{tikzpicture}
}

\newcommand{\tikzfigzaPA}{
\begin{tikzpicture}[line cap=round,line join=round,>=triangle 45,x=1.0cm,y=1.0cm,scale=0.6]
\begin{axis}[
x=1.0cm,y=1.0cm,
axis lines=middle,
ymajorgrids=true,
xmajorgrids=true,
xmin=-1.5,
xmax=3.5,
ymin=-3.2,
ymax=3.2,
xtick={-1.0,0.0,...,3.0},
ytick={-3.0,-2.0,...,3.0},]
\clip(-1.5,-3.2) rectangle (3.5,3.2);
\draw[line width=2.pt,color=blue,smooth,samples=100,domain=-1.5:3.5] plot(\x,{0-1*((\x)+0.0)^(2.0)+2.0});
\begin{scriptsize}
\draw[color=blue] (-1,2.5) node {\large{$\mathcal{C}_i$}};
\end{scriptsize}
\end{axis}
\end{tikzpicture}
}

\newcommand{\tikzfigmWht}{
\begin{tikzpicture}[line cap=round,line join=round,>=triangle 45,x=1.0cm,y=1.0cm,scale=0.8]
\clip(2.5,0.5) rectangle (9.5,5.5);
\fill[line width=2.pt,color=ffffff,fill=ffffff,fill opacity=1.0] (3.,5.) -- (9.,5.) -- (9.,1.) -- (3.,1.) -- cycle;
\fill[line width=0.pt,color=ffqqqq,fill=ffqqqq,fill opacity=1.0] (3.,5.) -- (3.,3.5) -- (4.5,3.5) -- (4.5,5.) -- cycle;
\fill[line width=0.pt,color=ffqqqq,fill=ffqqqq,fill opacity=1.0] (3.,1.) -- (3.,2.5) -- (4.5,2.5) -- (4.5,1.) -- cycle;
\fill[line width=0.pt,color=ffqqqq,fill=ffqqqq,fill opacity=1.0] (5.5,1.) -- (5.5,2.5) -- (9.,2.5) -- (9.,1.) -- cycle;
\fill[line width=0.pt,color=ffqqqq,fill=ffqqqq,fill opacity=1.0] (5.5,5.) -- (5.5,3.5) -- (9.,3.5) -- (9.,5.) -- cycle;
\draw [line width=2.pt] (3.,5.)-- (9.,5.);
\draw [line width=2.pt] (9.,5.)-- (9.,1.);
\draw [line width=2.pt] (9.,1.)-- (3.,1.);
\draw [line width=2.pt] (3.,1.)-- (3.,5.);
\end{tikzpicture}
}

\newcommand{\tikzfigFuya}{
\begin{tikzpicture}[line cap=round,line join=round,>=triangle 45,x=1.0cm,y=1.0cm,scale=1]
\clip(-0.1,-0.1) rectangle (4.1,4.1);
\fill[line width=1.pt,color=blue,fill=blue!30,fill opacity=1] (0,0) -- (3.,0.) -- (0.,1.) -- cycle;
\fill[line width=1.pt,color=red,fill=blue!30,fill opacity=1] (4,0) -- (4,3) -- (3,0) -- cycle;
\fill[line width=1.pt,color=blue,fill=blue!30,fill opacity=1] (4,4) -- (1,4) -- (4,3) -- cycle;
\fill[line width=1.pt,color=blue,fill=blue!30,fill opacity=1] (0,4) -- (0,1) -- (1,4) -- cycle;
\draw [line width=1.pt,color=blue] (0,0)-- (4,0);
\draw [line width=1.pt,color=blue] (4,0)-- (4,4);
\draw [line width=1.pt,color=blue] (4,4.)-- (0,4);
\draw [line width=1.pt,color=blue] (0,4)-- (0,0);
\draw [line width=1.pt,color=blue] (0,1)-- (3,0);
\draw [line width=1.pt,color=blue] (3,0)-- (4,3);
\draw [line width=1.pt,color=blue] (4,3)-- (1,4);
\draw [line width=1.pt,color=blue] (1,4)-- (0,1);
\end{tikzpicture}
}



\hypersetup{
    pdfauthor={R.Deschamps},
    pdfsubject={},
    pdfkeywords={},
    pdfproducer={LuaLaTeX},
    pdfcreator={Boum Factory}
}


% ========================================
% MACROS POUR RÉSOLUTION AUTOMATIQUE DE TRINÔMES
% ========================================

\usepackage{siunitx} % Pour la notation avec virgule
\usepackage{xfp}

% Configuration siunitx pour affichage français
\sisetup{%
  output-decimal-marker={,}
}

% Commande personnalisée pour affichage avec virgule
\newcommand{\cperso}[1]{\num{\fpeval{#1}}}

% ========================================
% ACCOLADES GRANDES ET ENSEMBLE DISCRET
% ========================================

% Accolade gauche grande avec TikZ
\newcommand{\accoladeG}[1][1]{%
    \tikz[baseline=0.1ex,scale=#1]{
        \draw[line width=0.8pt,line cap=round]
            (0.15,0.45) .. controls (0.08,0.42) and (0.05,0.35) ..
            (0.05,0.25) .. controls (0.05,0.15) and (0.02,0.08) ..
            (0,0) .. controls (0.02,-0.08) and (0.05,-0.15) ..
            (0.05,-0.25) .. controls (0.05,-0.35) and (0.08,-0.42) ..
            (0.15,-0.45);
    }%
}

% Accolade droite grande avec TikZ
\newcommand{\accoladeD}[1][1]{%
    \tikz[baseline=0.1ex,scale=#1]{
        \draw[line width=0.8pt,line cap=round]
            (0,0.45) .. controls (0.07,0.42) and (0.1,0.35) ..
            (0.1,0.25) .. controls (0.1,0.15) and (0.13,0.08) ..
            (0.15,0) .. controls (0.13,-0.08) and (0.1,-0.15) ..
            (0.1,-0.25) .. controls (0.1,-0.35) and (0.07,-0.42) ..
            (0,-0.45);
    }%
}

% Ensemble discret avec grandes accolades à hauteur fixe
\newcommand{\ensembleDiscret}[2][1]{%
    \ifcase#1\relax
        \mathopen{\{}\,#2\,\mathclose{\}}%
    \or
        \mathopen{\{}\,#2\,\mathclose{\}}%
    \or
        \mathopen{\big\{}\,#2\,\mathclose{\big\}}%
    \or
        \mathopen{\Big\{}\,#2\,\mathclose{\Big\}}%
    \fi
}

% ========================================
% Variables globales pour les coefficients
\newcommand{\seta}[1]{\def\coeffa{#1}}
\newcommand{\setb}[1]{\def\coeffb{#1}}
\newcommand{\setc}[1]{\def\coeffc{#1}}

% Commandes pour afficher les coefficients avec signe
\newcommand{\affsignea}{\ifdim\coeffa pt<0pt-\fi}
\newcommand{\affsigneb}{\ifdim\coeffb pt<0pt-\fi}
\newcommand{\affsignec}{\ifdim\coeffc pt<0pt-\fi}

% Valeurs absolues
\newcommand{\absa}{\fpeval{abs(\coeffa)}}
\newcommand{\absb}{\fpeval{abs(\coeffb)}}
\newcommand{\absc}{\fpeval{abs(\coeffc)}}

% Affichage avec parenthèses si négatif (après un signe)
\newcommand{\parencoeffa}{\ifdim\coeffa pt<0pt(\coeffa)\else\coeffa\fi}
\newcommand{\parencoeffb}{\ifdim\coeffb pt<0pt(\coeffb)\else\coeffb\fi}
\newcommand{\parencoeffc}{\ifdim\coeffc pt<0pt(\coeffc)\else\coeffc\fi}

% Affichage direct (début de ligne ou sans multiplication)
\newcommand{\directcoeffa}{\coeffa}
\newcommand{\directcoeffb}{\coeffb}
\newcommand{\directcoeffc}{\coeffc}

% Calcul du discriminant
\newcommand{\calcdelta}{\fpeval{\coeffb*\coeffb - 4*\coeffa*\coeffc}}
\newcommand{\deltasimpl}{\cperso{round(\calcdelta,4)}}

% Affichage du calcul du discriminant avec gestion intelligente des signes
\newcommand{\affcalcdelta}{%
    \directcoeffb^2-4\times\parencoeffa\times\parencoeffc = %
    \cperso{\coeffb*\coeffb}%
    \ifdim\fpeval{4*\coeffa*\coeffc} pt<0pt
        +\cperso{abs(4*\coeffa*\coeffc)}
    \else
        -\cperso{4*\coeffa*\coeffc}
    \fi = %
    \deltasimpl
}

% Signe de delta
\newcommand{\signedelta}{%
    \ifnum\numexpr\fpeval{round(\calcdelta,0)}<0
        <
    \else
        \ifnum\numexpr\fpeval{round(\calcdelta,0)}>0
            >
        \else
            =
        \fi
    \fi
}

% Test si delta >= 0
\newcommand{\ifdeltapositif}[2]{%
    \ifnum\numexpr\fpeval{round(\calcdelta,0)}<0
        #2%
    \else
        #1%
    \fi
}

% Calcul de x0 (cas delta = 0)
\newcommand{\calcxzero}{\fpeval{round(-\coeffb/(2*\coeffa),4)}}
\newcommand{\calcxzerofrac}{\cperso{round(-\coeffb/(2*\coeffa),4)}}

% Affichage du calcul de x0
\newcommand{\affcalcxzero}{%
    -\dfrac{b}{2a} = -\dfrac{\coeffb}{2\times\parencoeffa} = %
    \ifdim\coeffb pt<0pt
        \dfrac{\absb}{2\times\parencoeffa}
    \else
        -\dfrac{\coeffb}{2\times\parencoeffa}
    \fi = %
    \ifdim\fpeval{-\coeffb} pt<0pt
        \dfrac{\cperso{abs(-\coeffb)}}{\cperso{2*\coeffa}}
    \else
        -\dfrac{\cperso{-\coeffb}}{\cperso{abs(2*\coeffa)}}
    \fi = %
    \calcxzerofrac
}

% Calcul des racines (cas delta > 0)
\newcommand{\calcxun}{\fpeval{round((-\coeffb - sqrt(\calcdelta))/(2*\coeffa),4)}}
\newcommand{\calcxdeux}{\fpeval{round((-\coeffb + sqrt(\calcdelta))/(2*\coeffa),4)}}

% Racine carrée de delta arrondie
\newcommand{\sqrtdelta}{\fpeval{round(sqrt(\calcdelta),4)}}
\newcommand{\sqrtdeltafrac}{\cperso{round(sqrt(\calcdelta),4)}}

% Affichage du calcul de x1
\newcommand{\affcalcxun}{%
    \dfrac{-b-\sqrt{\Delta}}{2a} = %
    \dfrac{%
        \ifdim\coeffb pt<0pt
            \absb
        \else
            -\coeffb
        \fi
        -\sqrt{\deltasimpl}%
    }{2\times\parencoeffa} = %
    \ifdim\issimplexun pt=1pt
        % Si simple, montrer les étapes de calcul
        \dfrac{%
            \ifdim\coeffb pt<0pt
                \absb
            \else
                -\coeffb
            \fi
            -\sqrtdeltafrac%
        }{\cperso{2*\coeffa}} = %
        \dfrac{\cperso{-\coeffb-\sqrtdelta}}{\cperso{2*\coeffa}} = %
        \xunaffichage
    \else
        % Si complexe, afficher directement la forme exacte simplifiée
        \xunaffichage
    \fi
}

% Affichage du calcul de x2
\newcommand{\affcalcxdeux}{%
    \dfrac{-b+\sqrt{\Delta}}{2a} = %
    \dfrac{%
        \ifdim\coeffb pt<0pt
            \absb
        \else
            -\coeffb
        \fi
        +\sqrt{\deltasimpl}%
    }{2\times\parencoeffa} = %
    \ifdim\issimplexdeux pt=1pt
        % Si simple, montrer les étapes de calcul
        \dfrac{%
            \ifdim\coeffb pt<0pt
                \absb
            \else
                -\coeffb
            \fi
            +\sqrtdeltafrac%
        }{\cperso{2*\coeffa}} = %
        \dfrac{\cperso{-\coeffb+\sqrtdelta}}{\cperso{2*\coeffa}} = %
        \xdeuxaffichage
    \else
        % Si complexe, afficher directement la forme exacte simplifiée
        \xdeuxaffichage
    \fi
}

% Simplification des racines si possibles (affichage fraction avec virgule)
\newcommand{\xunfrac}{%
    \cperso{abs(round(\calcxun,0) - \calcxun) < 0.001 ? round(\calcxun,0) : \calcxun}%
}

\newcommand{\xdeuxfrac}{%
    \cperso{abs(round(\calcxdeux,0) - \calcxdeux) < 0.001 ? round(\calcxdeux,0) : \calcxdeux}%
}

% Plus petite et plus grande racine (pour calculs)
\newcommand{\xpetit}{\fpeval{min(\calcxun,\calcxdeux)}}
\newcommand{\xgrand}{\fpeval{max(\calcxun,\calcxdeux)}}

% Plus petite et plus grande racine (pour affichage avec virgule)
\newcommand{\xpetitfrac}{\cperso{min(\calcxun,\calcxdeux)}}
\newcommand{\xgrandfrac}{\cperso{max(\calcxun,\calcxdeux)}}

% ========================================
% AFFICHAGE FORME EXACTE DES RACINES
% ========================================

% Test si une racine est "simple" (entière ou 1 décimale max)
\newcommand{\issimplexun}{%
    \fpeval{abs(round(\calcxun,1) - \calcxun) < 0.01 ? 1 : 0}%
}
\newcommand{\issimplexdeux}{%
    \fpeval{abs(round(\calcxdeux,1) - \calcxdeux) < 0.01 ? 1 : 0}%
}

% Forme exacte de x1 = (-b - √Δ)/(2a)
\newcommand{\xunexact}{%
    \ifdim\fpeval{2*\coeffa} pt=-1pt
        % Cas 2a = -1 : résultat = b + √Δ
        \ifdim\coeffb pt<0pt
            \fpeval{abs(\coeffb)} - \sqrt{\deltasimpl}%
        \else
            \coeffb + \sqrt{\deltasimpl}%
        \fi
    \else\ifdim\fpeval{2*\coeffa} pt=1pt
        % Cas 2a = 1 : résultat = -b - √Δ
        \ifdim\coeffb pt<0pt
            \fpeval{abs(\coeffb)} - \sqrt{\deltasimpl}%
        \else
            -\coeffb - \sqrt{\deltasimpl}%
        \fi
    \else
        % Cas général : fraction
        \dfrac{%
            \ifdim\coeffb pt<0pt
                \fpeval{abs(\coeffb)}%
            \else
                -\coeffb
            \fi
            - \sqrt{\deltasimpl}%
        }{\cperso{2*\coeffa}}%
    \fi\fi
}

% Forme exacte de x2 = (-b + √Δ)/(2a)
\newcommand{\xdeuxexact}{%
    \ifdim\fpeval{2*\coeffa} pt=-1pt
        % Cas 2a = -1 : résultat = b - √Δ
        \ifdim\coeffb pt<0pt
            \fpeval{abs(\coeffb)} + \sqrt{\deltasimpl}%
        \else
            \coeffb - \sqrt{\deltasimpl}%
        \fi
    \else\ifdim\fpeval{2*\coeffa} pt=1pt
        % Cas 2a = 1 : résultat = -b + √Δ
        \ifdim\coeffb pt<0pt
            \fpeval{abs(\coeffb)} + \sqrt{\deltasimpl}%
        \else
            -\coeffb + \sqrt{\deltasimpl}%
        \fi
    \else
        % Cas général : fraction
        \dfrac{%
            \ifdim\coeffb pt<0pt
                \fpeval{abs(\coeffb)}%
            \else
                -\coeffb
            \fi
            + \sqrt{\deltasimpl}%
        }{\cperso{2*\coeffa}}%
    \fi\fi
}

% Affichage intelligent de x1 (forme exacte si complexe, décimale si simple)
\newcommand{\xunaffichage}{%
    \ifdim\issimplexun pt=1pt
        \xunfrac
    \else
        \xunexact
    \fi
}

% Affichage intelligent de x2 (forme exacte si complexe, décimale si simple)
\newcommand{\xdeuxaffichage}{%
    \ifdim\issimplexdeux pt=1pt
        \xdeuxfrac
    \else
        \xdeuxexact
    \fi
}

% Affichage intelligent du petit (min)
\newcommand{\xpetitaffichage}{%
    \ifdim\calcxun pt<\calcxdeux pt
        \xunaffichage
    \else
        \xdeuxaffichage
    \fi
}

% Affichage intelligent du grand (max)
\newcommand{\xgrandaffichage}{%
    \ifdim\calcxun pt>\calcxdeux pt
        \xunaffichage
    \else
        \xdeuxaffichage
    \fi
}

% Signe de a (pour tableau de signe)
\newcommand{\signea}{%
    \ifdim\coeffa pt<0pt
        -
    \else
        +
    \fi
}

% Signe de -a
\newcommand{\signemoina}{%
    \ifdim\coeffa pt<0pt
        +
    \else
        -
    \fi
}

% ========================================
% COMMANDE PRINCIPALE : RÉSOLUTION ÉQUATION
% ========================================
\newcommand{\resoudreequation}[4]{%
    \seta{#1}\setb{#2}\setc{#3}%
    \begin{tcbenumerate}[2]
    \tcbitem \textbf{Identification des coefficients :}

    $a=\coeffa$, $b=\coeffb$ et $c=\coeffc$

    \tcbitem \textbf{Calcul du discriminant :}

    $\Delta = b^2-4ac = \affcalcdelta$

    \ifdeltapositif{%
        \ifnum\numexpr\fpeval{round(\calcdelta,0)}=0
            % Cas delta = 0
            \tcbitem[raster multicolumn=2]\textbf{Détermination de la racine :}   $\Delta = 0$ donc l'équation admet une solution double :

            $x_0 = \affcalcxzero$

            \tcbitem[raster multicolumn=2]\textbf{Ensemble des solutions :}

            $S = \ensembleDiscret{ \calcxzero }$
        \else
            % Cas delta > 0
            \tcbitem[raster multicolumn=2]\textbf{Détermination des racines :}
            $\Delta = \deltasimpl > 0$ donc l'équation admet deux solutions réelles distinctes :

            \begin{MultiColonnes}{2}[halign=center]
            \tcbitem $x_1 = \affcalcxun$
            \tcbitem $x_2 = \affcalcxdeux$
            \end{MultiColonnes}

            \tcbitem[raster multicolumn=2]\textbf{Ensemble des solutions :} \encadrer[red]{$S = \ensembleDiscret{ \xunaffichage ; \xdeuxaffichage }$}
        \fi
    }{%
        % Cas delta < 0
        \tcbitem[raster multicolumn=2]\textbf{Conclusion :}
        $\Delta = \deltasimpl < 0$ donc l'équation n'admet pas de solution réelle.

        $S = \emptyset$
    }%
    \end{tcbenumerate}
}

% ========================================
% COMMANDE PRINCIPALE : RÉSOLUTION INÉQUATION
% ========================================
\newcommand{\resoudreinequation}[4]{%
    % #1 = a, #2 = b, #3 = c, #4 = type (geq, leq, g, l)
    \seta{#1}\setb{#2}\setc{#3}%
    \def\typeineg{#4}%
    \begin{tcbenumerate}[2]
    \tcbitem \textbf{Identification des coefficients :}

    $a=\coeffa$, $b=\coeffb$ et $c=\coeffc$

    \tcbitem\textbf{Calcul du discriminant :}

    $\Delta = b^2-4ac = \affcalcdelta$

    \tcbitem[raster multicolumn=2]\textbf{Détermination des racines :}

    \ifdeltapositif{%
        \ifnum\numexpr\fpeval{round(\calcdelta,0)}=0
            $\Delta = 0$ donc le trinôme admet une racine double :

            $x_0 = \affcalcxzero$
        \else
            $\Delta = \deltasimpl > 0$ donc le trinôme admet deux racines réelles distinctes :

            \begin{MultiColonnes}{2}[halign=center,boxrule=0.4pt,colframe=black,colback=white]
            \tcbitem $x_1 = \affcalcxun$
            \tcbitem $x_2 = \affcalcxdeux$
            \end{MultiColonnes}
        \fi
    }{%
        $\Delta = \deltasimpl < 0$ donc pas de racine réelle.
    }%
        \end{tcbenumerate}
        \begin{MultiColonnes}{2}
            \tcbitem \begin{tcbenumerate}[2][4]\tcbitem[raster multicolumn=2]\textbf{Tableau de signe} de $f:x\mapsto ax^2+bx+c$ :

    On a $a=\coeffa$ \ifdim\coeffa pt<0pt{$<0$}\else{$>0$}\fi{} et %
    \ifdeltapositif{%
        \ifnum\numexpr\fpeval{round(\calcdelta,0)}=0
            $x_0 = \calcxzero$
        \else
            $x_1 = \xunaffichage$ et $x_2 = \xdeuxaffichage$ donc $\xpetitaffichage < \xgrandaffichage$
        \fi
    }{pas de racine}

    \ifdeltapositif{%
        \ifnum\numexpr\fpeval{round(\calcdelta,0)}=0
            % Delta = 0
            \begin{tikzpicture}
            \tkzTabInit[espcl=2,lgt=1.5]{$x$/0.8,$f(x)$/0.8}
            {$-\infty$,$\calcxzero$,$+\infty$}
            \tkzTabLine{,\signea,z,\signea}
            \end{tikzpicture}
        \else
            % Delta > 0
            \begin{tikzpicture}
            \tkzTabInit[espcl=2,lgt=1.5]{$x$/0.8,$f(x)$/0.8}
            {$-\infty$,$\xpetitaffichage$,$\xgrandaffichage$,$+\infty$}
            \tkzTabLine{,\signea,z,\signemoina,z,\signea}
            \end{tikzpicture}
        \fi
    }{%
        % Delta < 0
        \begin{tikzpicture}
        \tkzTabInit[espcl=3,lgt=1.5]{$x$/0.8,$f(x)$/0.8}
        {$-\infty$,$+\infty$}
        \tkzTabLine{,\signea}
        \end{tikzpicture}
    }%

    \tcbitem[raster multicolumn=2]\textbf{Solution de l'inéquation :}

    % Solution selon le type et le signe
    \def\tempgeq{geq}\def\templeq{leq}\def\tempg{g}\def\templ{l}%
    \ifx\typeineg\tempgeq
        On cherche où le trinôme est $\geq 0$, donc les signes $+$ et $0$ :
    \fi
    \ifx\typeineg\templeq
        On cherche où le trinôme est $\leq 0$, donc les signes $-$ et $0$ :
    \fi
    \ifx\typeineg\tempg
        On cherche où le trinôme est $> 0$, donc les signes $+$ (strictement) :
    \fi
    \ifx\typeineg\templ
        On cherche où le trinôme est $< 0$, donc les signes $-$ (strictement) :
    \fi

    % Déterminer la solution selon delta et type
    \ifdeltapositif{%
        \ifnum\numexpr\fpeval{round(\calcdelta,0)}=0
            % Delta = 0 : solution selon signe de a et type
            \ifx\typeineg\tempgeq
                \ifdim\coeffa pt>0pt
                    $S = \R$
                \else
                    $S = \ensembleDiscret{\calcxzerofrac \right\rbrace$}
                \fi
            \fi
            \ifx\typeineg\templeq
                \ifdim\coeffa pt<0pt
                    $S = \R$
                \else
                    $S = \ensembleDiscret{ \calcxzerofrac }$
                \fi
            \fi
            \ifx\typeineg\tempg
                \ifdim\coeffa pt>0pt
                    $S = \R \setminus \ensembleDiscret{\calcxzerofrac}$
                \else
                    $S = \emptyset$
                \fi
            \fi
            \ifx\typeineg\templ
                \ifdim\coeffa pt<0pt
                    $S = \R \setminus \ensembleDiscret{\calcxzerofrac}$
                \else
                    $S = \emptyset$
                \fi
            \fi
        \else
            % Delta > 0 : solution selon signe de a et type
            \ifx\typeineg\tempgeq
                \ifdim\coeffa pt>0pt
                    $S = \CrochetD-\infty;\xpetitaffichage\,\,\CrochetD \cup \CrochetG\xgrandaffichage;+\infty\,\,\CrochetG$
                \else
                    $S = \CrochetG\xpetitaffichage;\xgrandaffichage\,\,\CrochetD$
                \fi
            \fi
            \ifx\typeineg\templeq
                \ifdim\coeffa pt<0pt
                    $S = \CrochetD-\infty;\xpetitaffichage\CrochetD \cup \CrochetG\xgrandaffichage;+\infty\,\,\CrochetG$
                \else
                    $S = \CrochetG\xpetitaffichage;\xgrandaffichage\,\,\CrochetD$
                \fi
            \fi
            \ifx\typeineg\tempg
                \ifdim\coeffa pt>0pt
                    $S = \CrochetD-\infty;\xpetitaffichage\,\,\CrochetG \cup \CrochetD\xgrandaffichage;+\infty\,\,\CrochetG$
                \else
                    $S = \CrochetD\xpetitaffichage;\xgrandaffichage\,\,\CrochetG$
                \fi
            \fi
            \ifx\typeineg\templ
                \ifdim\coeffa pt<0pt
                    $S = \CrochetD-\infty;\xpetitaffichage\,\,\CrochetG \cup \CrochetD\xgrandaffichage;+\infty\,\,\CrochetG$
                \else
                    $S = \CrochetD\xpetitaffichage;\xgrandaffichage\,\,\CrochetG$
                \fi
            \fi
        \fi
    }{%
        % Delta < 0 : solution selon signe de a et type
        \ifx\typeineg\tempgeq
            \ifdim\coeffa pt>0pt
                $S = \R$
            \else
                $S = \emptyset$
            \fi
        \fi
        \ifx\typeineg\templeq
            \ifdim\coeffa pt<0pt
                $S = \R$
            \else
                $S = \emptyset$
            \fi
        \fi
        \ifx\typeineg\tempg
            \ifdim\coeffa pt>0pt
                $S = \R$
            \else
                $S = \emptyset$
            \fi
        \fi
        \ifx\typeineg\templ
            \ifdim\coeffa pt<0pt
                $S = \R$
            \else
                $S = \emptyset$
            \fi
        \fi
    }%
        \end{tcbenumerate}
            \tcbitem[halign=center,valign=center] \setrdcrep{seyes=false,correction color=black}\begin{crep}[colback=white,halign=center]
\begin{tikzpicture}[scale=0.7]
    % Calcul des bornes adaptatives
    \pgfmathsetmacro{\ecart}{abs(\xgrand - \xpetit)}
    \pgfmathsetmacro{\marge}{max(1.5, \ecart * 0.4)}
    \pgfmathsetmacro{\xmin}{\xpetit - \marge}
    \pgfmathsetmacro{\xmax}{\xgrand + \marge}

    % Axes avec bornes adaptatives
    \draw[->] (\xmin,0) -- (\xmax,0) node[right] {$x$};
    \draw[->] (0,-3) -- (0,4) node[above] {};
    \draw[thick,blue,domain=\xmin:\xmax,samples=100] plot (\x,{\coeffa*(\x)^2+\coeffb*(\x)+\coeffc});
    \draw (\xpetit,0) node {$+$} node[below=3pt] {$\xpetitaffichage$};
    \draw (\xgrand,0) node {$+$} node[below=3pt] {$\xgrandaffichage$};
    \node[blue,above right] at (0.5,3) {$f(x)=\coeffa x^2
  \ifdim\coeffb pt=1pt
      +x
  \else\ifdim\coeffb pt=-1pt
      -x
  \else\ifdim\coeffb pt<0pt
      \coeffb x
  \else
      +\coeffb x
  \fi\fi\fi
  \ifdim\coeffc pt<0pt
      \coeffc
  \else
      +\coeffc
  \fi$};
    % Variables pour la logique d'affichage
    \def\tempgeq{geq}\def\templeq{leq}\def\tempg{g}\def\templ{l}%

    % Dessiner ligne rouge selon type
    % Pour geq (≥0)
    \ifx\typeineg\tempgeq
        \ifdim\coeffa pt>0pt
            % a>0, f(x)≥0 : extérieur (]-∞,x1]∪[x2,+∞[)
            \draw[red,line width=0.8pt,<-] (\xpetit,0) -- (\xmin,0);
            \draw[red,line width=0.8pt,->] (\xgrand,0) -- (\xmax,0);
            % Crochets fermés
            \draw[red, line width=1.5pt] (\xpetit-0.1,-0.3) -- (\xpetit,-0.3) -- (\xpetit,0.3) -- (\xpetit-0.1,0.3);
            \draw[red, line width=1.5pt] (\xgrand+0.1,-0.3) -- (\xgrand,-0.3) -- (\xgrand,0.3) -- (\xgrand+0.1,0.3);
        \else
            % a<0, f(x)≥0 : intérieur ([x1,x2])
            \draw[red,line width=0.8pt] (\xpetit,0) -- (\xgrand,0);
            % Crochets fermés
            \draw[red, line width=1.5pt] (\xpetit-0.1,-0.3) -- (\xpetit,-0.3) -- (\xpetit,0.3) -- (\xpetit-0.1,0.3);
            \draw[red, line width=1.5pt] (\xgrand+0.1,-0.3) -- (\xgrand,-0.3) -- (\xgrand,0.3) -- (\xgrand+0.1,0.3);
        \fi
    \fi

    % Pour leq (≤0)
    \ifx\typeineg\templeq
        \ifdim\coeffa pt<0pt
            % a<0, f(x)≤0 : extérieur (]-∞,x1]∪[x2,+∞[)
            \draw[red,line width=0.8pt,<-] (\xpetit,0) -- (\xmin,0);
            \draw[red,line width=0.8pt,->] (\xgrand,0) -- (\xmax,0);
            % Crochets fermés
            \draw[red, line width=1.5pt] (\xpetit-0.1,-0.3) -- (\xpetit,-0.3) -- (\xpetit,0.3) -- (\xpetit-0.1,0.3);
            \draw[red, line width=1.5pt] (\xgrand+0.1,-0.3) -- (\xgrand,-0.3) -- (\xgrand,0.3) -- (\xgrand+0.1,0.3);
        \else
            % a>0, f(x)≤0 : intérieur ([x1,x2])
            \draw[red,line width=0.8pt] (\xpetit,0) -- (\xgrand,0);
            % Crochets fermés
            \draw[red, line width=1.5pt] (\xpetit-0.1,-0.3) -- (\xpetit,-0.3) -- (\xpetit,0.3) -- (\xpetit-0.1,0.3);
            \draw[red, line width=1.5pt] (\xgrand+0.1,-0.3) -- (\xgrand,-0.3) -- (\xgrand,0.3) -- (\xgrand+0.1,0.3);
        \fi
    \fi

    % Pour g (>0)
    \ifx\typeineg\tempg
        \ifdim\coeffa pt>0pt
            % a>0, f(x)>0 : extérieur (]-∞,x1[∪]x2,+∞[)
            \draw[red,line width=0.8pt,<-] (\xpetit,0) -- (\xmin,0);
            \draw[red,line width=0.8pt,->] (\xgrand,0) -- (\xmax,0);
            % Crochets ouverts
            \draw[red, line width=1.5pt] (\xpetit,-0.3) -- (\xpetit,0.3) -- (\xpetit+0.1,0.3) (\xpetit,-0.3) -- (\xpetit+0.1,-0.3);
            \draw[red, line width=1.5pt] (\xgrand,-0.3) -- (\xgrand,0.3) -- (\xgrand-0.1,0.3) (\xgrand,-0.3) -- (\xgrand-0.1,-0.3);
        \else
            % a<0, f(x)>0 : intérieur (]x1,x2[)
            \draw[red,line width=0.8pt] (\xpetit,0) -- (\xgrand,0);
            % Crochets ouverts
            \draw[red, line width=1.5pt] (\xpetit,-0.3) -- (\xpetit,0.3) -- (\xpetit+0.1,0.3) (\xpetit,-0.3) -- (\xpetit+0.1,-0.3);
            \draw[red, line width=1.5pt] (\xgrand,-0.3) -- (\xgrand,0.3) -- (\xgrand-0.1,0.3) (\xgrand,-0.3) -- (\xgrand-0.1,-0.3);
        \fi
    \fi

    % Pour l (<0)
    \ifx\typeineg\templ
        \ifdim\coeffa pt<0pt
            % a<0, f(x)<0 : extérieur (]-∞,x1[∪]x2,+∞[)
            \draw[red,line width=0.8pt,<-] (\xpetit,0) -- (\xmin,0);
            \draw[red,line width=0.8pt,->] (\xgrand,0) -- (\xmax,0);
            % Crochets ouverts
            \draw[red, line width=1.5pt] (\xpetit,-0.3) -- (\xpetit,0.3) -- (\xpetit+0.1,0.3) (\xpetit,-0.3) -- (\xpetit+0.1,-0.3);
            \draw[red, line width=1.5pt] (\xgrand,-0.3) -- (\xgrand,0.3) -- (\xgrand-0.1,0.3) (\xgrand,-0.3) -- (\xgrand-0.1,-0.3);
        \else
            % a>0, f(x)<0 : intérieur (]x1,x2[)
            \draw[red,line width=0.8pt] (\xpetit,0) -- (\xgrand,0);
            % Crochets ouverts
            \draw[red, line width=1.5pt] (\xpetit,-0.3) -- (\xpetit,0.3) -- (\xpetit+0.1,0.3) (\xpetit,-0.3) -- (\xpetit+0.1,-0.3);
            \draw[red, line width=1.5pt] (\xgrand,-0.3) -- (\xgrand,0.3) -- (\xgrand-0.1,0.3) (\xgrand,-0.3) -- (\xgrand-0.1,-0.3);
        \fi
    \fi
\end{tikzpicture}
\end{crep}
        \end{MultiColonnes}
    }
\begin{document}

\setcounter{pagecounter}{0}
\setcounter{ExoMA}{0}



\def\rdifficulty{1}
\chapitre[
    $\mathbf{1^{\text{ère}}}$% : $\mathbf{6^{\text{ème}}}$,$\mathbf{5^{\text{ème}}}$,$\mathbf{4^{\text{ème}}}$,$\mathbf{3^{\text{ème}}}$,$\mathbf{2^{\text{nde}}}$,$\mathbf{1^{\text{ère}}}$,$\mathbf{T^{\text{Le}}}$,
    ]{
    Factorisation des trinômes du second degré% : ,Equations
    }{
    Lycée% : Collège,Lycée
    }{
    Camille Claudel% : Amadis Jamyn,Eugène Belgrand
    }{
    % : ,\tableauPresenteEvalSixieme{}{10},\tableofcontents
    }{
    Cours :% : Cours :,Exercices
    }

    \tableofcontents
    
    \vfill
    \tableaucompetence{
        \competence{Calculer le discriminant associé à un polynôme de degré 2}
        \competence{Connaître le lien entre discriminant et signe d'un polynôme de degré 2}
        \competence{Dresser le tableau de signes d'une fonction polynôme de degré 2}
        \competence{Résoudre des équations et des inéquations du second degré}
    }
    \vfill
    \printvocindex
    \vfill
    \newpage

\begin{EXO}{Problème ouvert}{}
    \vspace{-0.5cm}\begin{flushright}\textit{(source: académie Aix-Marseille)}\end{flushright}
    \vspace{-0.4cm}\tcbitempoint{20} Le laboratoire d'une aciérie étudie la dilatation d'un acier fabriqué par l'entreprise.

Les mesures effectuées donnent les résultats suivants pour une tige d'acier :
\begin{center}
\begin{tcbtab}{|c|c|c|c|c|c|}
Température en $^\circ$C & 0 & 50 & 100 & 200 & 400 \\\hline
Longueur en cm & \num{50} & \num{50.03} & \num{50.06} & \num{50.12} & \num{50.24} \\\hline
\end{tcbtab}
\end{center}

\begin{tcbenumerate}
\tcbitem A \acc{quelle température} faut-il porter la tige pour que sa longueur soit égale à \num{50,15}cm ?


\tcbitem La température la plus basse que l'on peut atteindre est appelé le zéro absolue. Sa température est de \num{-273.15}$^\circ$C.

\acc{Est-il possible} de réduire la taille de la tige d'acier d'\acc{un centième} de sa taille initiale ?
\end{tcbenumerate}
\exocorrection
\begin{MultiColonnes}{2}
    \tcbitem L'\acc{allongement} $A$ de la tige d'acier semble être \acc{proportionnel} à la \acc{température} de la tige. 
    
    En effet : 
    \tcbitem[halign=center] \begin{tcbtab}{c|c|c|c|c}
\acc{Température} en $^\circ$C &50 & 100 & 200 & 400 \\\hline
Longueur en cm & \num{50.03} & \num{50.06} & \num{50.12} & \num{50.24} \\\hline
\acc{Allongement} en cm & \num{0.03} & \num{0.06} & \num{0.12} & \num{0.24} \\
\end{tcbtab}
\end{MultiColonnes}
On vérifie que les lignes Température et Allongement sont proportionnelles en calculant les rapports $\dfrac{A}{T}$ dans chaque colonne : 
\vspace{-0.35cm}\begin{multicols}{2}
\foreach \x/\y in {50/0.03 , 100/0.06 , 200/0.12 , 400/0.24 } {
    $\dfrac{\num{\y}}{\x} = \displaystyle{\Simplification{\fpeval{\y*100}}{\fpeval{\x*100}}}=\num{\fpeval{\y/\x}}$ \\ \phantom{a} \\
}
\end{multicols}

\vspace{-1cm}Le \acc{coefficient de proportionnalité} est $\num{0.0006}=\num{6d-4}$ et on en déduit : 

$A = \num{6d-4} \times T$. 

On peut observer le lien suivant : L = \encadrer[defi]{Longueur initiale} + \encadrer[blue]{Allongement}

\begin{tcolorbox}[
    colback=green!10,
    colframe=green!50!black,
    title={\faCalculator\space Modélisation mathématique}
]
À partir de ces observations et calculs, on peut déterminer l'expression de la \acc{fonction} donnant la \acc{longueur} $L$ de la tige en fonction de la \acc{température} $T \in \intervalle[ff]{0}{400}$:
\begin{center}
$L:T\mapsto \underbrace{50}_{L_0} + \underbrace{\num{6d-4}T}_{\text{Allongement}}$
\end{center}

Puisque l'expression est définie sur $\R$, il est \acc{naturel} d'étendre son domaine de définition pour 

$T\in \intervalle[ff]{\num{-273.15}}{400}$ ce qui permet de répondre aux deux questions de façon raisonnable. 
\end{tcolorbox}

\textbf{\large Graphique des données expérimentales}

\begin{center}
\begin{tikzpicture}[scale=0.9]
    % Axes
    \draw[thick,->] (-0.5,0) -- (11,0) node[right] {$T$ ($^\circ$C)};
    \draw[thick,->] (0,-0.5) -- (0,9) node[above] {$L$ (cm)};

    % Graduations axe x
    \foreach \x/\xtext in {0/0, 2/100, 4/200, 6/300, 8/400, 10/500} {
        \draw (\x,0.1) -- (\x,-0.1) node[below] {\num{\xtext}};
    }

    % Graduations axe y
    \foreach \y/\ytext in {0/50.00, 2/50.06, 4/50.12, 6/50.18, 6.67/50.20, 8/50.24} {
        \draw (0.1,\y) -- (-0.1,\y) node[left] {\num{\ytext}};
    }

    % Points expérimentaux
    \coordinate (A) at (0,0);
    \coordinate (B) at (1,1);
    \coordinate (C) at (2,2);
    \coordinate (D) at (4,4);
    \coordinate (E) at (8,8);

    % Tracer les points
    \foreach \point/\x/\y in {A/0/0,B/1/1,C/2/2,D/4/4,E/8/8} {
        \fill[red] (\point) circle (2pt);
        \draw[gray!70!black,dashed] (\point) -- (\x,0) node[below] {};
        \draw[gray!70!black,dashed] (\point) -- (0,\y) node[left] {};
    }

    % Points avec leurs coordonnées
    %\node[above right] at (A) {$(0, 50.00)$};
    %\node[above right] at (B) {$(50, 50.03)$};
    %\node[above right] at (C) {$(100, 50.06)$};
    %\node[above right] at (D) {$(200, 50.12)$};
    %\node[above right] at (E) {$(400, 50.24)$};

    % Régression linéaire
    \draw[blue,thick,dashed] (-1.5,-1.5) -- (8.5,8.5) node[right] {$L(T) = 50 + 6 \times 10^{-4} T$};

    % Point recherché pour question 1
    \coordinate (F) at (5,5);
    \fill[green!70!black] (F) circle (2pt);
    \draw[green!70!black,dashed] (F) -- (5,0) node[below] {$250$};
    \draw[green!70!black,dashed] (F) -- (0,5) node[left] {$\num{50.15}$};
    \node[above right,green!70!black,yshift=-10pt] at (F) {Solution Q1};
\end{tikzpicture}
\end{center}

\begin{tcbenumerate}[2]
\tcbitem Température pour $L = \num{50.15}$ cm
\begin{tcolorbox}[
    colback=blue!10,
    colframe=blue!50!black,
    title={\faThermometerHalf\space Résolution :}
]
Nous cherchons $T$ tel que $L(T) = \num{50.15}$ cm.
\begin{align*}
50 + 6 \times 10^{-4} T &= \num{50.15}\\[0.15cm]
6 \times 10^{-4} T &= \num{0.15}\\[0.15cm]
T &= \frac{\num{0.15}}{6 \times 10^{-4}}\\[0.15cm]
T &= \frac{\num{0.15} \times 10^{4}}{6}\\[0.15cm]
T &= \frac{1500}{6}\\[0.15cm]
T &= \boxed{250 ^\circ\text{C}}
\end{align*}

\textbf{Vérification :} $L(250) = 50 + 6 \times 10^{-4} \times 250 = 50 + \num{0.15} = \num{50.15}$ 

\bcoeil On pouvait aussi résoudre \acc{graphiquement} cette question ( voir graphique précédent ).
\end{tcolorbox}

\tcbitem Réduction d'un centième de la taille initiale

\begin{tcolorbox}[
    colback=red!10,
    colframe=red!50!black,
    title={\faSnowflake\space Analyse de la contraction thermique :}
]
Une réduction d'un centième signifie que la tige devrait mesurer : 

$L_{\text{reduit}} = L_{\text{initial}} - \frac{L_{\text{initial}}}{100} = 50 - 0.5 = \num{49.5}$ cm

    Cherchons la température nécessaire :
\begin{align*}
50 + 6 \times 10^{-4} T &= \num{49.5}\\[0.15cm]
6 \times 10^{-4} T &= -\num{0.5}\\[0.15cm]
T &= \frac{-\num{0.5}}{6 \times 10^{-4}}\\[0.15cm]
T &= -\frac{5000}{6}\\[0.15cm]
T &\approx \boxed{-\num{833.33} ^\circ\text{C}}
\end{align*}

Or, Le zéro absolu est à $-\num{273.15}$ $^\circ$C et aucune température ne peut descendre en dessous. 

\textbf{Conclusion :} Cette température étant impossible à atteindre, on ne peut pas réduire la tige d'un centième de sa taille initiale.

\end{tcolorbox}
\end{tcbenumerate}

\vspace{0.5cm}

\begin{tcolorbox}[
    colback=purple!10,
    colframe=purple!50!black,
    title={\faGraduationCap\space Remarques :}
]
\begin{tcbenumerate}
    \tcbitem[colback=purple!10] \textbf{Modèle affine :} La dilatation de cet acier peut être \acc{modélisé} par une fonction affine $L(T) = 50 + \num{6d-4}T$

    Dans le cadre de ce devoir et d'après les données disponibles, \acc{on a supposé que} le modèle affine suggéré par les données est valable pour $T>-273{,}15^\circ$.

    Ce n'est pas nécessairement vrai, mais cela \acc{suffit} à répondre à la question posée. 

    En effet il y a peu de chances que la contraction thermique se produise plus facilement au fur et à mesure que la température diminue. 
    \tcbitem[colback=purple!10] \textbf{Contraction maximale :} À $T = -273.15$ $^\circ$C, $L \approx \num{49.836}$ cm , ce qui correspond à une contraction maximale de seulement $0{,}033\%$ de la longueur initiale.
    \tcbitem[colback=purple!10] En l'absence d'informations supplémentaires, le problème reste ouvert. 

    En effet, nos conlusions sont vraies dans le cadre de notre modèle et une simple hypothèse supplémentaire peut tout changer : par exemple qu'en est-il si la pression varie également ?
\end{tcbenumerate}
\end{tcolorbox}

\begin{tcbtab}[Barème pour les traces de recherche]{|p{13cm}|c|}
\textbf{Critère} & \textbf{Points} \\\hline\hline
Formulation tableau de proportionnalité ou lien affine. & /2 \\\hline
Calculs de vérification (ou partiel) & /1 \\\hline
Lien avec l'expression de la fonction ou avec le graphique. & /2 \\\hline
Modélisation par une fonction affine : explicitation par l'élève. & /2 \\\hline\hline
\textbf{Total} & \textbf{/7} \\\hline
\end{tcbtab}

\vspace{0.5cm}

\begin{tcbtab}[Barème pour la question 1]{|p{13cm}|c|}
\textbf{Critère} & \textbf{Points} \\\hline\hline
Expliquer le lien entre la question posée et le modèle déterminé. & /1 \\\hline
Trouver l'équation à résoudre pour répondre à la question 1. Ou mise en lien avec le graphique. & /2 \\\hline
Résoudre cette équation ou la résoudre graphiquement (traits de construction apparents). Pour obtenir tous les points l'élève doit avoir mentionné ou contourné (par du papier millimétré par exemple) la légère imprécision de la méthode, mais qui permet tout de même de conclure. & /2 \\\hline
Conclure & /1 \\\hline\hline
\textbf{Total} & \textbf{/6} \\\hline
\end{tcbtab}

\vspace{0.5cm}

\begin{tcbtab}[Barème pour la question 2]{|p{13cm}|c|}
\textbf{Critère} & \textbf{Points} \\\hline\hline
Expliquer le lien entre la question posée et le modèle déterminé. & /1 \\\hline
Trouver l'équation à résoudre & /2 \\\hline
Résoudre l'équation & /2 \\\hline
Conclure & /2 \\\hline\hline
\textbf{Total} & \textbf{/7} \\\hline
\end{tcbtab}


\acc{Remarques pour la question 1 : } 

\begin{itemize}
    \item Le lien entre la température et la longueur est \acc{affine}. Il était attendu que l'élève le mentionne. 
    \item Il était attendu que des calculs soient présents pour les traces de recherche.
\end{itemize}

\acc{Remarques pour la question 2 : } 

\begin{itemize}
    \item La méthode graphique est toujours possible pour la question 2 mais plus compliquée à mettre en \oe uvre. L'élève recevra tous les points s'il justifie comme dans la première question.
    \item Dans la question 2, le calcul de la longueur à $T=\num{-273.15}^\circ$ ne suffit pas : il faut en plus préciser que la fonction étant croissante, la longueur voulue ne peut effectivement être atteinte pour $T>\num{-273.15}^\circ$.
    \item Une autre stratégie plus efficace est de calculer directement la température à atteindre pour obtenir la longueur souhaitée. Cela permettait de ne pas justifier que la fonction $L(T)$ est croissante. 
\end{itemize}

\end{EXO}
% : 
\begin{EXO}{Problème ouvert}{}
    \vspace{-0.5cm}\begin{flushright}\textit{(source: académie Aix-Marseille)}\end{flushright}
    \vspace{-0.4cm}\tcbitempoint{20} Le laboratoire d'une aciérie étudie la dilatation d'un acier fabriqué par l'entreprise.

Les mesures effectuées donnent les résultats suivants pour une tige d'acier :
\begin{center}
\begin{tcbtab}{|c|c|c|c|c|c|}
Température en $^\circ$C & 0 & 50 & 100 & 200 & 400 \\\hline
Longueur en cm & \num{50} & \num{50.03} & \num{50.06} & \num{50.12} & \num{50.24} \\\hline
\end{tcbtab}
\end{center}

\begin{tcbenumerate}
\tcbitem A \acc{quelle température} faut-il porter la tige pour que sa longueur soit égale à \num{50,15}cm ?


\tcbitem La température la plus basse que l'on peut atteindre est appelé le zéro absolue. Sa température est de \num{-273.15}$^\circ$C.

\acc{Est-il possible} de réduire la taille de la tige d'acier d'\acc{un centième} de sa taille initiale ?
\end{tcbenumerate}
\exocorrection
\begin{MultiColonnes}{2}
    \tcbitem L'\acc{allongement} $A$ de la tige d'acier semble être \acc{proportionnel} à la \acc{température} de la tige. 
    
    En effet : 
    \tcbitem[halign=center] \begin{tcbtab}{c|c|c|c|c}
\acc{Température} en $^\circ$C &50 & 100 & 200 & 400 \\\hline
Longueur en cm & \num{50.03} & \num{50.06} & \num{50.12} & \num{50.24} \\\hline
\acc{Allongement} en cm & \num{0.03} & \num{0.06} & \num{0.12} & \num{0.24} \\
\end{tcbtab}
\end{MultiColonnes}
On vérifie que les lignes Température et Allongement sont proportionnelles en calculant les rapports $\dfrac{A}{T}$ dans chaque colonne : 
\vspace{-0.35cm}\begin{multicols}{2}
\foreach \x/\y in {50/0.03 , 100/0.06 , 200/0.12 , 400/0.24 } {
    $\dfrac{\num{\y}}{\x} = \displaystyle{\Simplification{\fpeval{\y*100}}{\fpeval{\x*100}}}=\num{\fpeval{\y/\x}}$ \\ \phantom{a} \\
}
\end{multicols}

\vspace{-1cm}Le \acc{coefficient de proportionnalité} est $\num{0.0006}=\num{6d-4}$ et on en déduit : 

$A = \num{6d-4} \times T$. 

On peut observer le lien suivant : L = \encadrer[defi]{Longueur initiale} + \encadrer[blue]{Allongement}

\begin{tcolorbox}[
    colback=green!10,
    colframe=green!50!black,
    title={\faCalculator\space Modélisation mathématique}
]
À partir de ces observations et calculs, on peut déterminer l'expression de la \acc{fonction} donnant la \acc{longueur} $L$ de la tige en fonction de la \acc{température} $T \in \intervalle[ff]{0}{400}$:
\begin{center}
$L:T\mapsto \underbrace{50}_{L_0} + \underbrace{\num{6d-4}T}_{\text{Allongement}}$
\end{center}

Puisque l'expression est définie sur $\R$, il est \acc{naturel} d'étendre son domaine de définition pour 

$T\in \intervalle[ff]{\num{-273.15}}{400}$ ce qui permet de répondre aux deux questions de façon raisonnable. 
\end{tcolorbox}

\textbf{\large Graphique des données expérimentales}

\begin{center}
\begin{tikzpicture}[scale=0.9]
    % Axes
    \draw[thick,->] (-0.5,0) -- (11,0) node[right] {$T$ ($^\circ$C)};
    \draw[thick,->] (0,-0.5) -- (0,9) node[above] {$L$ (cm)};

    % Graduations axe x
    \foreach \x/\xtext in {0/0, 2/100, 4/200, 6/300, 8/400, 10/500} {
        \draw (\x,0.1) -- (\x,-0.1) node[below] {\num{\xtext}};
    }

    % Graduations axe y
    \foreach \y/\ytext in {0/50.00, 2/50.06, 4/50.12, 6/50.18, 6.67/50.20, 8/50.24} {
        \draw (0.1,\y) -- (-0.1,\y) node[left] {\num{\ytext}};
    }

    % Points expérimentaux
    \coordinate (A) at (0,0);
    \coordinate (B) at (1,1);
    \coordinate (C) at (2,2);
    \coordinate (D) at (4,4);
    \coordinate (E) at (8,8);

    % Tracer les points
    \foreach \point/\x/\y in {A/0/0,B/1/1,C/2/2,D/4/4,E/8/8} {
        \fill[red] (\point) circle (2pt);
        \draw[gray!70!black,dashed] (\point) -- (\x,0) node[below] {};
        \draw[gray!70!black,dashed] (\point) -- (0,\y) node[left] {};
    }

    % Points avec leurs coordonnées
    %\node[above right] at (A) {$(0, 50.00)$};
    %\node[above right] at (B) {$(50, 50.03)$};
    %\node[above right] at (C) {$(100, 50.06)$};
    %\node[above right] at (D) {$(200, 50.12)$};
    %\node[above right] at (E) {$(400, 50.24)$};

    % Régression linéaire
    \draw[blue,thick,dashed] (-1.5,-1.5) -- (8.5,8.5) node[right] {$L(T) = 50 + 6 \times 10^{-4} T$};

    % Point recherché pour question 1
    \coordinate (F) at (5,5);
    \fill[green!70!black] (F) circle (2pt);
    \draw[green!70!black,dashed] (F) -- (5,0) node[below] {$250$};
    \draw[green!70!black,dashed] (F) -- (0,5) node[left] {$\num{50.15}$};
    \node[above right,green!70!black,yshift=-10pt] at (F) {Solution Q1};
\end{tikzpicture}
\end{center}

\begin{tcbenumerate}[2]
\tcbitem Température pour $L = \num{50.15}$ cm
\begin{tcolorbox}[
    colback=blue!10,
    colframe=blue!50!black,
    title={\faThermometerHalf\space Résolution :}
]
Nous cherchons $T$ tel que $L(T) = \num{50.15}$ cm.
\begin{align*}
50 + 6 \times 10^{-4} T &= \num{50.15}\\[0.15cm]
6 \times 10^{-4} T &= \num{0.15}\\[0.15cm]
T &= \frac{\num{0.15}}{6 \times 10^{-4}}\\[0.15cm]
T &= \frac{\num{0.15} \times 10^{4}}{6}\\[0.15cm]
T &= \frac{1500}{6}\\[0.15cm]
T &= \boxed{250 ^\circ\text{C}}
\end{align*}

\textbf{Vérification :} $L(250) = 50 + 6 \times 10^{-4} \times 250 = 50 + \num{0.15} = \num{50.15}$ 

\bcoeil On pouvait aussi résoudre \acc{graphiquement} cette question ( voir graphique précédent ).
\end{tcolorbox}

\tcbitem Réduction d'un centième de la taille initiale

\begin{tcolorbox}[
    colback=red!10,
    colframe=red!50!black,
    title={\faSnowflake\space Analyse de la contraction thermique :}
]
Une réduction d'un centième signifie que la tige devrait mesurer : 

$L_{\text{reduit}} = L_{\text{initial}} - \frac{L_{\text{initial}}}{100} = 50 - 0.5 = \num{49.5}$ cm

    Cherchons la température nécessaire :
\begin{align*}
50 + 6 \times 10^{-4} T &= \num{49.5}\\[0.15cm]
6 \times 10^{-4} T &= -\num{0.5}\\[0.15cm]
T &= \frac{-\num{0.5}}{6 \times 10^{-4}}\\[0.15cm]
T &= -\frac{5000}{6}\\[0.15cm]
T &\approx \boxed{-\num{833.33} ^\circ\text{C}}
\end{align*}

Or, Le zéro absolu est à $-\num{273.15}$ $^\circ$C et aucune température ne peut descendre en dessous. 

\textbf{Conclusion :} Cette température étant impossible à atteindre, on ne peut pas réduire la tige d'un centième de sa taille initiale.

\end{tcolorbox}
\end{tcbenumerate}

\vspace{0.5cm}

\begin{tcolorbox}[
    colback=purple!10,
    colframe=purple!50!black,
    title={\faGraduationCap\space Remarques :}
]
\begin{tcbenumerate}
    \tcbitem[colback=purple!10] \textbf{Modèle affine :} La dilatation de cet acier peut être \acc{modélisé} par une fonction affine $L(T) = 50 + \num{6d-4}T$

    Dans le cadre de ce devoir et d'après les données disponibles, \acc{on a supposé que} le modèle affine suggéré par les données est valable pour $T>-273{,}15^\circ$.

    Ce n'est pas nécessairement vrai, mais cela \acc{suffit} à répondre à la question posée. 

    En effet il y a peu de chances que la contraction thermique se produise plus facilement au fur et à mesure que la température diminue. 
    \tcbitem[colback=purple!10] \textbf{Contraction maximale :} À $T = -273.15$ $^\circ$C, $L \approx \num{49.836}$ cm , ce qui correspond à une contraction maximale de seulement $0{,}033\%$ de la longueur initiale.
    \tcbitem[colback=purple!10] En l'absence d'informations supplémentaires, le problème reste ouvert. 

    En effet, nos conlusions sont vraies dans le cadre de notre modèle et une simple hypothèse supplémentaire peut tout changer : par exemple qu'en est-il si la pression varie également ?
\end{tcbenumerate}
\end{tcolorbox}

\begin{tcbtab}[Barème pour les traces de recherche]{|p{13cm}|c|}
\textbf{Critère} & \textbf{Points} \\\hline\hline
Formulation tableau de proportionnalité ou lien affine. & /2 \\\hline
Calculs de vérification (ou partiel) & /1 \\\hline
Lien avec l'expression de la fonction ou avec le graphique. & /2 \\\hline
Modélisation par une fonction affine : explicitation par l'élève. & /2 \\\hline\hline
\textbf{Total} & \textbf{/7} \\\hline
\end{tcbtab}

\vspace{0.5cm}

\begin{tcbtab}[Barème pour la question 1]{|p{13cm}|c|}
\textbf{Critère} & \textbf{Points} \\\hline\hline
Expliquer le lien entre la question posée et le modèle déterminé. & /1 \\\hline
Trouver l'équation à résoudre pour répondre à la question 1. Ou mise en lien avec le graphique. & /2 \\\hline
Résoudre cette équation ou la résoudre graphiquement (traits de construction apparents). Pour obtenir tous les points l'élève doit avoir mentionné ou contourné (par du papier millimétré par exemple) la légère imprécision de la méthode, mais qui permet tout de même de conclure. & /2 \\\hline
Conclure & /1 \\\hline\hline
\textbf{Total} & \textbf{/6} \\\hline
\end{tcbtab}

\vspace{0.5cm}

\begin{tcbtab}[Barème pour la question 2]{|p{13cm}|c|}
\textbf{Critère} & \textbf{Points} \\\hline\hline
Expliquer le lien entre la question posée et le modèle déterminé. & /1 \\\hline
Trouver l'équation à résoudre & /2 \\\hline
Résoudre l'équation & /2 \\\hline
Conclure & /2 \\\hline\hline
\textbf{Total} & \textbf{/7} \\\hline
\end{tcbtab}


\acc{Remarques pour la question 1 : } 

\begin{itemize}
    \item Le lien entre la température et la longueur est \acc{affine}. Il était attendu que l'élève le mentionne. 
    \item Il était attendu que des calculs soient présents pour les traces de recherche.
\end{itemize}

\acc{Remarques pour la question 2 : } 

\begin{itemize}
    \item La méthode graphique est toujours possible pour la question 2 mais plus compliquée à mettre en \oe uvre. L'élève recevra tous les points s'il justifie comme dans la première question.
    \item Dans la question 2, le calcul de la longueur à $T=\num{-273.15}^\circ$ ne suffit pas : il faut en plus préciser que la fonction étant croissante, la longueur voulue ne peut effectivement être atteinte pour $T>\num{-273.15}^\circ$.
    \item Une autre stratégie plus efficace est de calculer directement la température à atteindre pour obtenir la longueur souhaitée. Cela permettait de ne pas justifier que la fonction $L(T)$ est croissante. 
\end{itemize}

\end{EXO}
,

\end{document}