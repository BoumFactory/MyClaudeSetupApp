
\section{Factorisation d'un trinôme}

\begin{Definition}[Discriminant]
    On considère un trinôme du second degré $ax^2+bx+c$.

    On appelle\voc{discriminant} du polynôme, le nombre \encadrer[defi]{$\Delta = b^2-4ac$}. 
\end{Definition}
\begin{Propriete}[Discriminant et forme factorisée]
On considère un trinôme du second degré $ax^2+bx+c$ ; et $\Delta$ son \acc{discriminant}
\begin{tcbenumerate}
\tcbitem  Si \encadrer[red]{$\Delta<0$}, on sait que le trinôme n'a \acc{pas de racine réelle}. 

Il n'est \acc{pas factorisable} dans $\R$.
\tcbitem  Si \encadrer[prop]{$\Delta = 0$}, on sait que le trinôme a \acc{une racine double} : $x_0=-\dfrac{b}{2a}$. 

Pour tout réel $x$, $ax^2+bx+c=a(x-x_0)^2$.
\tcbitem  Si \encadrer[defi]{$\Delta > 0$}, on sait que le trinôme a \acc{deux racines} : $x_1=\dfrac{-b-\sqrt{\Delta}}{2a} $ \hspace*{1mm}et\hspace*{1mm} $x_2=\dfrac{-b+\sqrt{\Delta}}{2a}$. 

Pour tout réel $x$, $ax^2+bx+c=a(x-x_1)(x-x_2)$.
\end{tcbenumerate}
\end{Propriete}

\begin{Demonstration}[Forme factorisée]
\begin{MultiColonnes}{2}
    \tcbitem On sait que $ax^2 + bx + c = a\left[\left(x+\dfrac{b}{2a}\right)^2-\dfrac{\Delta}{4a^2}\right]$ (Forme canonique du trinôme).

\noindent On raisonne par disjonction des cas :
\begin{tcbenumerate}
\tcbitem Dans le cas ou $\Delta = 0$ : 
\edef\currentseyesoption{\ifcorrection false\else true\fi}
\setrdcrep{seyes=\currentseyesoption,correction color=black, correction font=\normalsize}
\begin{crep}[colback=white,colframe=white,boxrule=-0.2pt]
\begin{align*}
&ax^2 + bx + c \\
=&  a\left[\left(x+\dfrac{b}{2a}\right)^2-\dfrac{0}{4a^2}\right]  \\
=& a\left(x+\dfrac{b}{2a}\right)^2  \\
=&  a(x- x_0)^2
\end{align*}
\end{crep}
\end{tcbenumerate}
\tcbitem \begin{tcbenumerate}[1][2]
\tcbitem Dans le cas ou $\Delta > 0$ : 
\edef\currentseyesoption{\ifcorrection false\else true\fi}
\setrdcrep{seyes=\currentseyesoption,correction color=black, correction font=\normalsize}
\begin{crep}[colback=white,colframe=white,boxrule=-0.2pt]\begin{eqnarray*}
&ax^2 + bx + c \\
=& a\left[\left(x+\dfrac{b}{2a}\right)^2-\dfrac{\Delta}{4a^2}\right] \\
=& a\left[\left(x+\dfrac{b}{2a}\right)^2-\sqrt{\dfrac{\Delta}{4a^2}}^2\right] \\
=& a\left[\left(x+\dfrac{b}{2a}-\sqrt{\dfrac{\Delta}{4a^2}}\right)\left(x+\dfrac{b}{2a}+\sqrt{\dfrac{\Delta}{4a^2}}\right)\right]\\
=& a\left[\left(x+\dfrac{b}{2a}-\dfrac{\sqrt{\Delta}}{2a}\right)\left(x+\dfrac{b}{2a}+\dfrac{\sqrt{\Delta}}{2a}\right)\right]\\
=& a\left[\left(x+\dfrac{b-\sqrt{\Delta}}{2a}\right)\left(x+\dfrac{b+\sqrt{\Delta}}{2a}\right)\right]\\
=& a(x- x_1)(x- x_2)
\end{eqnarray*}
\end{crep}
\end{tcbenumerate}
\end{MultiColonnes}
\end{Demonstration}

\begin{EXO}{Racines et forme factorisée}{}
\begin{MultiColonnes}{3}
    \tcbitem[raster multicolumn=2] \tcbitempoint{3}Soit $f$ la fonction trinôme définie sur $\R$ par $f(x)=2x^2+x-3$.

\noindent Ici, $a=2$, $\Delta=25$, $x_1=-\dfrac{3}{2}$ et $x_2=1$ donc pour tout réel $x$, on a : $$f(x)=2\left(x+\dfrac{3}{2}\right)(x-1)$$
    \tcbitem \acc{Vérifier} les informations fournies par l'énoncé, en \acc{calculant} $\Delta$ et en déterminant les racines de $f$. 
\end{MultiColonnes}

\exocorrection

\begin{tcbenumerate}
\tcbitem Calcul du discriminant :

$\Delta = b^2 - 4ac = 1^2 - 4 \times 2 \times (-3) = 1 + 24 = 25$

\tcbitem Calcul des racines :

\begin{MultiColonnes}{2}
\tcbitem $x_1 = \dfrac{-b-\sqrt{\Delta}}{2a} = \dfrac{-1-\sqrt{25}}{2\times 2} = \dfrac{-1-5}{4} = \dfrac{-6}{4} = -\dfrac{3}{2}$
\tcbitem $x_2 = \dfrac{-b+\sqrt{\Delta}}{2a} = \dfrac{-1+\sqrt{25}}{2\times 2} = \dfrac{-1+5}{4} = \dfrac{4}{4} = 1$
\end{MultiColonnes}

\tcbitem Forme factorisée : $f(x) = 2\left(x-\left(-\dfrac{3}{2}\right)\right)(x-1)$, soit :
\begin{center}\encadrer[defi]{$f(x) = 2\left(x+\dfrac{3}{2}\right)(x-1)$}\end{center}

\end{tcbenumerate}

\end{EXO}

\newpage
\section{Signe du trinôme et discriminant}

\begin{Propriete}[Tableau de signes et discriminant]
Soit $f$ une fonction polynôme de degré 2 définie par $f(x)=a x^2+b x + c$ et $ \Delta $ son discriminant  associé.
\begin{MultiColonnes}{4}
    \tcbitem[raster multicolumn=3]\begin{tcbenumerate}
\tcbitem Si \encadrer[red]{$\Delta<0$}, pas de racine réelle.
\begin{center}
\begin{tikzpicture}
\tkzTabInit[espcl=3,lgt=2]{$x$/0.9,$f(x)$/0.9}
{$-\infty$,$+\infty$}
\tkzTabLine{,\textcolor{black}{\text{signe de $a$}}}
%\tkzTabVar{+/\textcolor{blue}{$+\infty$},-/$\text{Min}$,+/\textcolor{blue}{$+\infty$}}
\end{tikzpicture}
\end{center}
\end{tcbenumerate}
    \tcbitem[halign=center,valign=center] \begin{tikzpicture}[scale=0.6]
        \draw[->] (-2,0) -- (3,0) node[right] {$x$};
        \draw[->] (0,-0.5) -- (0,3) node[above] {};%$y$};
        \draw[thick,blue,domain=-1:2.5,samples=100] plot (\x,{0.6*(\x-0.7)^2+0.8});
        \node[blue,above] at (1.5,2.5) {\encadrer[red]{$\Delta<0$} et $a>0$};
    \end{tikzpicture}
\tcbitem[raster multicolumn=3]\begin{tcbenumerate}[1][2]
\tcbitem Si \encadrer[prop]{$\Delta=0$}, une racine double $x_0$.
\begin{center}
\begin{tikzpicture}
\tkzTabInit[espcl=3,lgt=2]{$x$/0.9,$f(x)$/0.9}
{$-\infty$,$x_0$,$+\infty$}
\tkzTabLine{,\textcolor{black}{\text{signe de $a$}},z,\textcolor{black}{\text{signe de $a$}}}
%\tkzTabVar{+/\textcolor{blue}{$+\infty$},-/$\text{Min}$,+/\textcolor{blue}{$+\infty$}}
\end{tikzpicture}
\end{center}
\end{tcbenumerate}
\tcbitem[halign=center,valign=center] \begin{tikzpicture}[scale=0.6]
        \draw[->] (-2,0) -- (3,0) node[right] {$x$};
        \draw[->] (0,-0.5) -- (0,3) node[above] {};%$y$};
        \draw[thick,blue,domain=-0.8:2.8,samples=100] plot (\x,{0.7*(\x-1)^2});
        \draw (1,0) node {$+$} node[below=3pt] {$x_0$};
        \node[blue,above] at (1.5,2.5) {\encadrer[prop]{$\Delta=0$} et $a>0$};
    \end{tikzpicture}

\tcbitem[raster multicolumn=3]\begin{tcbenumerate}[1][3]
    \tcbitem Si \encadrer[defi]{$\Delta>0$}, deux racines distinctes $x_1 < x_2$.
\begin{center}
\begin{tikzpicture}
\tkzTabInit[espcl=3,lgt=2]{$x$/0.9,$f(x)$/0.9}
{$-\infty$,$x_1$,$x_2$,$+\infty$}
\tkzTabLine{,\textcolor{black}{\text{signe de $a$}},z,\textcolor{purple}{\text{signe de -$a$}},z,\textcolor{black}{\text{signe de $a$}}}
%\tkzTabVar{+/\textcolor{blue}{$+\infty$},-/$\text{Min}$,+/\textcolor{blue}{$+\infty$}}
\end{tikzpicture}
\end{center}
\end{tcbenumerate}
\tcbitem[halign=center,valign=center] \begin{tikzpicture}[scale=0.6]
        \draw[->] (-2,0) -- (4,0) node[right] {$x$};
        \draw[->] (0,-1.5) -- (0,2) node[above] {};%$y$};
        \draw[thick,blue,domain=-0.8:3.8,samples=100] plot (\x,{0.5*(\x-0.5)*(\x-2.5)});
        \draw (0.5,0) node {$+$} node[below=3pt] {$x_1$};
        \draw (2.5,0) node {$+$} node[below=3pt] {$x_2$};
        \node[blue,above] at (2,1.5) {\encadrer[defi]{$\Delta>0$} et $a>0$};
    \end{tikzpicture}
\end{MultiColonnes}

\end{Propriete}

\begin{Methode}[Établir le tableau de signes d'une fonction $f:x\mapsto a x^2+bx+c$]
Construisons le tableau de signes de la fonction $f$ définie sur $\R$ par $f(x)=-3x^2+6x+45$
\begin{tcbenumerate}
\tcbitem Identifier les coefficients $a$, $b$ et $c$. Attention aux signes de ces quantités !

$a=\repsim[2cm]{-3}$, $b=\repsim[2cm]{6}$ et $c=\repsim[2cm]{45}$
\tcbitem Déterminer les racines éventuelles de $f$.
\begin{crep}[colback=white]
\[\Delta  =  b^2-4ac  =  6^2-4 \times (-3) \times 45  =  576\]
$\Delta$ est positif, donc l'équation $f(x)=0$ à exactement $2$ solutions.\\

$x_1  =  \dfrac{-b-\sqrt{\Delta}}{2a}  =  \dfrac{-6-\sqrt{576}}{2\times (-3)} =  5$ \hfill $x_2 = \dfrac{-b+\sqrt{\Delta}}{2a}  =  \dfrac{-6+\sqrt{576}}{2\times (-3)}  =  -3$
\end{crep}
\end{tcbenumerate}
\begin{tcbenumerate}[2][3]
\tcbitem[boxrule=0.4pt,colframe=black] Déterminer le signe de $a$ et la plus grande des deux valeurs $x_1$ et $x_2$.
\begin{tcbenumerate}[1][1][alph]
\tcbitem  \tcfillcrep{$a=-3<0$ $\rightarrow$ signe \og $-$ \fg{}.}
\tcbitem  \tcfillcrep{$x_1=5$ et $x_2=-3$ donc $x_2 < x_1$.}
\end{tcbenumerate}
\tcbitem Utiliser la propriété ci-dessus pour construire le tableau de signes, en remplaçant $a$, $x_1$ et $x_2$  par leurs valeurs :
\setrdcrep{seyes=false}\begin{crep}[colback=white,colframe=white,halign=center]
\begin{tikzpicture}
\tkzTabInit[espcl=1.8,lgt=1.5]{$x$/0.8,$f(x)$/0.8}
{$-\infty$,$-3$,$5$,$+\infty$}
\tkzTabLine{,\textcolor{black}{-},z,\textcolor{purple}{\text{+}},z,\textcolor{black}{-}}
%\tkzTabVar{+/\textcolor{blue}{$+\infty$},-/$\text{Min}$,+/\textcolor{blue}{$+\infty$}}
\end{tikzpicture}
\end{crep}

\end{tcbenumerate}
\end{Methode}

\section{Equations et inéquations du second degré}
\subsection{Définitions}
\begin{Definition}[\'Equation et inéquation du second degré]
\begin{MultiColonnes}{2}
    \tcbitem Une\voc{équation du second degré} est une équation qui peut s'écrire sous la forme :

    $$ax^2+bx+c=0$$

    où $a$, $b$ et $c$ sont des réels et $a\neq 0$.

    \tcbitem Une\voc{inéquation du second degré} est une inéquation qui peut s'écrire sous l'une des formes :

    $$ax^2+bx+c>0 \quad \text{ou} \quad ax^2+bx+c<0$$

    $$ax^2+bx+c\geq 0 \quad \text{ou} \quad ax^2+bx+c\leq 0$$

    où $a$, $b$ et $c$ sont des réels et $a\neq 0$.
\end{MultiColonnes}
\end{Definition}

\subsection{Résoudre une équation du second degré}
\begin{Methode}[Résoudre une équation du second degré]
Pour résoudre une équation du second degré, on procède en plusieurs étapes :

\begin{tcbenumerate}
\tcbitem\textbf{Réécrire l'équation} sous forme \acc{développée} $ax^2+bx+c=0$ avec $a\neq 0$.

\tcbitem\textbf{Identifier les coefficients} $a$, $b$ et $c$. Attention aux signes !

\tcbitem\textbf{Calculer le discriminant} $\Delta = b^2-4ac$.

\tcbitem\textbf{Déterminer les racines} selon le signe de $\Delta$ :

\begin{MultiColonnes}{2}[boxrule=0.4pt,colframe=black,colback=white]
    \tcbitem[valign=center] Si \encadrer[red]{$\Delta<0$} : pas de solution réelle, donc \encadrer[defi]{$S = \emptyset$}
    \tcbitem Si \encadrer[prop]{$\Delta=0$} : une solution double $x_0=-\dfrac{b}{2a}$, donc \encadrer[defi]{$S = \left\lbrace x_0 \right\rbrace$}
    \tcbitem[raster multicolumn=2] Si \encadrer[defi]{$\Delta>0$} : deux solutions $x_1=\dfrac{-b-\sqrt{\Delta}}{2a}$ et $x_2=\dfrac{-b+\sqrt{\Delta}}{2a}$, donc \encadrer[defi]{$S = \left\lbrace x_1 ; x_2 \right\rbrace$}
\end{MultiColonnes}
\end{tcbenumerate}
\end{Methode}
\begin{EXO}{Résoudre une équation du second degré}{}
\tcbitempoint{5}\acc{Résoudre} l'équation : $2x^2-7x+3=0$
\setrdcrep{correction font=\normalsize}\begin{crep}[colback=white]
    \resoudreequation{2}{-7}{3}


\end{crep}
\exocorrection

\resoudreequation{2}{-7}{3}


\end{EXO}
\subsection{Résoudre une inéquation du second degré}
\begin{Methode}[Résoudre une inéquation du second degré]
Pour résoudre une inéquation du second degré, on procède en plusieurs étapes :

\begin{tcbenumerate}
\tcbitem\textbf{Réécrire l'inéquation} sous la forme $ax^2+bx+c>0$ (ou $<0$, $\geq 0$, $\leq 0$) avec $a\neq 0$.

\tcbitem\textbf{Identifier les coefficients} $a$, $b$ et $c$. Attention aux signes !

\tcbitem\textbf{Calculer le discriminant} $\Delta = b^2-4ac$.

\tcbitem\textbf{Déterminer les racines} selon le signe de $\Delta$ (voir propriété précédente).

\tcbitem\textbf{Construire le tableau de signes} du trinôme $ax^2+bx+c$ en utilisant :
\begin{MultiColonnes}{1}
    \tcbitem Le signe de $a$ (hors des racines)
    \tcbitem Les racines (si elles existent)
    \tcbitem La règle : entre les racines, le signe est celui de $-a$
\end{MultiColonnes}

\tcbitem\textbf{Lire la solution} dans le tableau de signes selon l'inéquation demandée.
\end{tcbenumerate}
\end{Methode}

\def\rdifficulty{2}
\begin{EXO}{Résoudre une inéquation du second degré}{}
\tcbitempoint{6}\acc{Résoudre} l'inéquation : $-x^2+4x-3 \geq 0$
\setrdcrep{correction font=\normalsize}
\begin{crep}[colback=white]
    \resoudreinequation{-1}{4}{-3}{geq}
\end{crep}

\exocorrection

\resoudreinequation{-1}{4}{-3}{geq}
\end{EXO}

\def\rdifficulty{2}
\begin{EXO}{Résoudre une inéquation du second degré}{}
\tcbitempoint{6}\acc{Résoudre} l'inéquation : $-\dfrac{1}{2}x^2+2x+1 < 0$
%\setrdcrep{correction font=\normalsize}
%\begin{crep}[colback=white]
%    \resoudreinequation{-3}{5}{1}{<}
%\end{crep}

\exocorrection

\resoudreinequation{-0.5}{2}{1}{l}

\end{EXO}