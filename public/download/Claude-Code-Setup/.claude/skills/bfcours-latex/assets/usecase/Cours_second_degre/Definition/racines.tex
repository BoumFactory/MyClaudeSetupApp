\begin{Definition}[Racines d'un polynôme]
    Les\voc{racines} d'un polynôme sont les nombres réels qui \acc{annulent} ce polynôme.
    
    \begin{Remarque}[]
        Soit $f$ la fonction définie sur $\R$ par $f(x) = ax^2+bx+c$ avec $a \in \R^*$ et $b,c \in \R$. 
        

        Les \acc{racines} de la fonction polynôme $f$ sont les \acc{solutions} de l'équation $f(x)=0$. 


        Graphiquement, ce sont les abscisses des points d'intersections entre la courbe représentative de $f$, notée $\mathcal{C}_f$, et l'axe des \acc{abscisses}.
    \end{Remarque}
\end{Definition}