\begin{EXO}{Forme canonique - Entraînement}{}
\tcbitempoint{4} Déterminer la forme canonique des fonctions suivantes (utiliser un brouillon). 
\begin{tcbenumerate}[2]
\tcbitem $f(x) = x^2-6x+5$
\begin{crep}
$f(x) = ( x - 3 )^2 - 4$
\end{crep}
\tcbitem $f(x) = x^2+5x+4$
\begin{crep}
$f(x) = ( x - \dfrac{5}{2} )^2 - \dfrac{9}{4}$ 
\end{crep}
\end{tcbenumerate}

\exocorrection

D'après le cours : 
L'écriture $a(x-\alpha)^{2} +\beta$ est la\voc{forme canonique} de la fonction $f:x\mapsto ax^{2} + bx + c$.
    
    \begin{tcbenumerate}[1]
        \tcbitem[halign=center] $\alpha = -\dfrac{b}{2a}$
        \tcbitem[halign=center] $\beta = -\dfrac{b^2-4ac}{4a}$
    \end{tcbenumerate}
    \begin{tcbenumerate}[2]
\tcbitem[boxrule=0.4pt,colframe=black,halign=left] $f(x) = x^2-6x+5$
\begin{tcbenumerate}[2][1][alph]
        \tcbitem[halign=center] $\alpha = -\dfrac{-6}{2\times 1} = 3$
        \tcbitem[halign=center] $\beta = -\dfrac{6^2-4\times 1 \times 5}{4\times 1}$
        
        $\phantom{\beta}= -\dfrac{36-20}{4} = -4$
    \end{tcbenumerate}
Ainsi $f(x) = ( x - 3 )^2 - 4$

\tcbitem[boxrule=0.4pt,colframe=black,halign=left] $f(x) = x^2+5x+4$
    \begin{tcbenumerate}[2][1][alph]
        \tcbitem[halign=center] $\alpha = -\dfrac{5}{2}$
        \tcbitem[halign=center] $\beta = -\dfrac{5^2-4\times 1 \times 4}{4\times 1}$
        
        $\phantom{\beta}= -\dfrac{25-16}{4} = -\dfrac{9}{4}$
    \end{tcbenumerate}
$f(x) = ( x - \dfrac{5}{2} )^2 - \dfrac{9}{4}$ 

\end{tcbenumerate}
\end{EXO}