\begin{EXO}{Forme canonique guidée}{}
Soit $f$ la fonction définie sur $\R$ par $ f(x) = x^2+4x+5$.
\begin{tcbenumerate}[2]
\tcbitem \tcbitempoint{1} Compléter l'égalité ci-contre avec des réels : 
\begin{center}
$x^2+4x+\repsim{4} = (x+\repsim{2})^2$
\end{center}

\tcbitem \tcbitempoint{2} En déduire la forme canonique de $f$.

\begin{crep}
$f(x) = (x+2)^2 + 1$
\end{crep}
\end{tcbenumerate}

\exocorrection

\begin{tcbenumerate}[2]
\tcbitem Pour compléter $(x+\dots)^2$, on cherche $a$ tel que $(x+a)^2 = x^2+2ax+a^2$.

On veut $2ax = 4x$, donc $2a = 4$, donc $a = 2$.

Alors $(x+2)^2 = x^2+4x+4$.

Donc $x^2+4x+4 = (x+2)^2$.

\tcbitem On a $f(x) = x^2+4x+5 = x^2+4x+4+1 = (x+2)^2+1$.

La forme canonique est $f(x) = (x+2)^2 + 1$.
\end{tcbenumerate}
\end{EXO}