\begin{EXO}{Reconnaissance de fonctions du second degré}{}
\tcbitempoint{3} Pour chaque fonction ci-dessous, déterminer si c'est une fonction polynôme de degré 2.
\begin{tcbenumerate}[3]
\tcbitem $f(x) = x^2+2x-\sqrt{2}$

\begin{crep}
Oui, $a=1$, $b=2$, $c=-\sqrt{2}$
\end{crep}

\tcbitem $g(x)= x^2+\dfrac{1}{x} -1$

\begin{crep}
Non, présence de $\dfrac{1}{x}$
\end{crep}

\tcbitem $h(x)=3x^2-3x-2x^2+2x-x^2-x+5$

\begin{crep}
Non, degré 0 après réduction
\end{crep}
\end{tcbenumerate}

\exocorrection

\begin{tcbenumerate}[3]
\tcbitem $f(x) = x^2+2x-\sqrt{2}$ est de la forme $ax^2+bx+c$ avec $a=1 \neq 0$, $b=2$, $c=-\sqrt{2}$.

Donc $f$ est une fonction polynôme de degré 2.

\tcbitem $g(x)= x^2+\dfrac{1}{x} -1 = x^2+x^{-1}-1$

La présence du terme $x^{-1}$ (exposant négatif) fait que $g$ n'est pas une fonction polynôme.

\tcbitem $h(x)=3x^2-3x-2x^2+2x-x^2-x+5$

Réduisons : $h(x) = (3-2-1)x^2 + (-3+2-1)x + 5 = 0x^2 - 2x + 5 = -2x + 5$

$h$ est une fonction affine (degré 1), pas une fonction polynôme de degré 2.
\end{tcbenumerate}
\end{EXO}