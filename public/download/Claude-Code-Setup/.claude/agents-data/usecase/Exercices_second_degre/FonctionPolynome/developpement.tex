\begin{EXO}{Développement et identification}{}
Soit $f$ la fonction définie sur $\R$ par $ f(x) = 2(x+2)^2 - 3(x+1)$.
\begin{tcbenumerate}[2]
\tcbitem \tcbitempoint{2} Développer $f(x)$.

\begin{crep}
$f(x) = 2(x^2+4x+4) - 3x - 3 $

$\phantom{f(x)}= 2x^2+8x+8-3x-3 $

$\phantom{f(x)}= 2x^2+5x+5$
\end{crep}

\tcbitem \tcbitempoint{1} En déduire que $f$ est une fonction polynôme de degré $2$ et déterminer ses coefficients.

\begin{crep}
$f(x) = 2x^2+5x+5$ donc $a=2$, $b=5$, $c=5$

Comme $a=2 \neq 0$, $f$ est bien de degré 2.
\end{crep}
\end{tcbenumerate}

\exocorrection

\begin{tcbenumerate}[2]
\tcbitem Développement :

$f(x) = 2(x+2)^2 - 3(x+1)$

$= 2(x^2+4x+4) - 3x - 3$

$= 2x^2+8x+8-3x-3$

$= 2x^2+5x+5$

\tcbitem $f(x) = 2x^2+5x+5$ est de la forme $ax^2+bx+c$ avec :

$a = 2 \neq 0$, $b = 5$, $c = 5$

Donc $f$ est bien une fonction polynôme de degré 2.
\end{tcbenumerate}
\end{EXO}