\begin{EXO}{Forme canonique par développement inverse}{}
Soit $f$ la fonction définie sur $\R$ par $ f(x) = -3x^2+24x-41$.
\begin{tcbenumerate}[2]
\tcbitem \tcbitempoint{2} \acc{Développer} l'expression $-3(x-4)^2+7$.

\begin{crep}
$-3(x-4)^2+7 $

$= -3(x^2-8x+16)+7 $

$= -3x^2+24x-48+7 $

$= -3x^2+24x-41$
\end{crep}

\tcbitem \tcbitempoint{1} En déduire la forme canonique de $f$.

\begin{crep}
$f(x) = -3(x-4)^2+7$
\end{crep}
\end{tcbenumerate}

\exocorrection

\begin{tcbenumerate}[2]
\tcbitem Développement de $-3(x-4)^2+7$ :

$-3(x-4)^2+7 = -3(x^2-8x+16)+7$

$= -3x^2+24x-48+7$

$= -3x^2+24x-41$

\tcbitem On constate que $-3(x-4)^2+7 = -3x^2+24x-41 = f(x)$.

Donc la forme canonique de $f$ est : $f(x) = -3(x-4)^2+7$.
\end{tcbenumerate}
\end{EXO}