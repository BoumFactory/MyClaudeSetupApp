\begin{EXO}{Détermination d'une parabole par son sommet}{}
Soit $f$ une fonction polynôme de degré 2. La courbe représentative de $f$ a pour sommet le point $A(1;3)$ et passe par le point $B(0;5)$. 

\tcbitempoint{3} Déterminer la forme canonique de $f$.

\begin{crep}
La forme canonique est $f(x) = a(x-\alpha)^2 + \beta$ avec sommet $(\alpha;\beta)$.

Ici $\alpha = 1$ et $\beta = 3$, donc $f(x) = a(x-1)^2 + 3$.

La courbe passe par $B(0;5)$ donc $f(0) = 5$

$a(0-1)^2 + 3 = 5$, donc $a \times 1 + 3 = 5$ 

Ainsi, $a = 2$ et $f(x) = 2(x-1)^2 + 3$
\end{crep}

\exocorrection

Puisque $f$ est une fonction polynôme de degré 2, sa forme canonique est :
\begin{center}$f(x) = a(x-\alpha)^2 + \beta$\end{center}

où $(\alpha;\beta)$ sont les coordonnées du sommet.

D'après l'énoncé, le sommet est $A(1;3)$, donc :
- $\alpha = 1$
- $\beta = 3$

La forme canonique devient : $f(x) = a(x-1)^2 + 3$

Il reste à déterminer le coefficient $a$. 

La courbe passe par le point $B(0;5)$, donc $f(0) = 5$ :

$f(0) = a(0-1)^2 + 3 = a \times 1 + 3 = a + 3$

Puisque $f(0) = 5$ :
\begin{center}$a + 3 = 5 \Leftrightarrow a = 2$\end{center}

La forme canonique de $f$ est donc : \begin{center}$f(x) = 2(x-1)^2 + 3$\end{center}
\end{EXO}