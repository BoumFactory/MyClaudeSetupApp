\begin{Propriete}[Forme factorisée]
    La fonction $f$ définie sur $\R$ par $f(x)=a(x-x_1)(x-x_2)$ est une fonction polynôme de degré 2, avec $a$, $x_1$ et $x_2$ des réels tels que $a\neq 0$.
    
    Cette écriture de $f(x)$ est appelée\voc{forme factorisée} de $f(x)$.
\end{Propriete}