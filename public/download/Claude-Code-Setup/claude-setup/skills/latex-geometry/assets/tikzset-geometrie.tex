% ============================================
% TIKZSET POUR LA GÉOMÉTRIE - BFCOURS
% ============================================
% À inclure dans le préambule de vos documents LaTeX
% Nécessite : \usepackage{bfcours}, \usetikzlibrary{decorations.markings,angles,quotes,positioning}

% IMPORTANT : Le package esvect (commande \vv) est incompatible avec LuaLaTeX + bfcours-fonts
% Alternatives pour les vecteurs :
% 1. \overrightarrow{#1} - Flèche extensible standard (compatible)
% 2. \vec{#1} - Petite flèche standard (compatible)
% 3. \mathbf{#1} - Vecteur en gras, notation anglo-saxonne (compatible)
% 4. \boldsymbol{#1} - Vecteur en gras italique (compatible)
\newcommand{\Vecteur}[1]{\overrightarrow{#1}}

\tikzset{
    % ===== GRILLES ET AXES =====
    quadrillage/.style={help lines, gray!40},
    epais/.style={thick, line width=1.2pt},
    axe/.style={->, >=stealth, thick, black},
    %
    % ===== POINTS =====
    point/.style={cross out, draw, minimum size=5pt, line width=0.8pt, inner sep=0pt},
    point correction/.style={cross out, draw, minimum size=5pt, line width=1.2pt, inner sep=0pt,red},
    label point/.style={font=\normalsize, inner sep=2pt},
    %
    % ===== SEGMENTS ET VECTEURS =====
    segment/.style={thick},
    vecteur/.style={->, >=stealth, thick},
    %
    % ===== COULEURS =====
    couleur1/.style={blue},
    couleur2/.style={red},
    couleur3/.style={green!60!black},
    prop/.style={blue!60!cyan},
    %
    % ===== CERCLES =====
    cercle/.style={draw, thick},
    cercle epais/.style={draw, line width=1.2pt},
    cercle pointille/.style={draw, thick, dashed},
    %
    % ===== ANGLES =====
    angle/.style={draw, thick},
    angle droit/.style={draw, thick},
    %
    % ===== CODAGES GÉOMÉTRIQUES =====
    codage segment/.style={decoration={markings, mark=at position 0.5 with {\draw[-] (0,-2pt) -- (0,2pt);}}, postaction={decorate}},
    codage segment double/.style={decoration={markings, mark=at position 0.5 with {\draw[-] (-1pt,-2pt) -- (-1pt,2pt); \draw[-] (1pt,-2pt) -- (1pt,2pt);}}, postaction={decorate}},
    codage segment triple/.style={decoration={markings, mark=at position 0.5 with {\draw[-] (-2pt,-2pt) -- (-2pt,2pt); \draw[-] (0,-2pt) -- (0,2pt); \draw[-] (2pt,-2pt) -- (2pt,2pt);}}, postaction={decorate}},
    %
    % ===== LIGNES REMARQUABLES =====
    mediatrice/.style={thick, dashed, gray},
    bissectrice/.style={thick, dotted, blue!60},
    hauteur/.style={thick, dotted, red!60},
    mediane/.style={thick, dotted, green!60!black},
    %
    % ===== STYLE GÉNÉRAL =====
    general/.style={point, segment, couleur1},
}

% ===== COMMANDES POUR REPÈRES =====

% Axe des abscisses avec graduations
% Usage: \XAxe{min}{max}{graduations}
% Exemple: \XAxe{-1}{5}{1,2,3,4,5}
\newcommand{\XAxe}[3]{
    \draw[->, thick] (#1-0.5,0) -- (#2+0.5,0) node[right] {$x$};
    \foreach \x in {#3} {
        \draw (\x,0.1) -- (\x,-0.1) node[below] {$\x$};
    }
}

% Axe des ordonnées avec graduations
% Usage: \YAxe{min}{max}{graduations}
% Exemple: \YAxe{-1}{4}{1,2,3,4}
\newcommand{\YAxe}[3]{
    \draw[->, thick] (0,#1-0.5) -- (0,#2+0.5) node[above] {$y$};
    \foreach \y in {#3} {
        \draw (0.1,\y) -- (-0.1,\y) node[left] {$\y$};
    }
}

% Origine du repère
% Usage: \origine
\newcommand{\origine}{
    \fill (0,0) circle (2pt) node[below left] {O};
}

% ===== COMMANDES POUR POINTS =====

% Point avec label
% Usage: \point{x}{y}{label}{position}
% Exemple: \point{3}{2}{A}{above right}
% Positions possibles: above, below, left, right, above left, above right, below left, below right
\newcommand{\point}[4]{
    \node[point] (#3-point) at (#1,#2) {};
    \node[label point,#4=2mm of #3-point] {#3};
}

% Point correction (rouge) avec label
% Usage: \pointCorrection{x}{y}{label}{position}
% Exemple: \pointCorrection{3}{2}{B}{above}
\newcommand{\pointCorrection}[4]{
    \node[point correction] (#3-point) at (#1,#2) {};
    \node[label point,#4=2mm of #3-point] {#3};
}

% ===== COMMANDES POUR ANGLES DROITS =====

% Angle droit avec orientation
% Usage: \angleDroit[orientation]{sommet}{taille}
% orientation (optionnel, défaut=1) :
%   1 = en haut à droite (0°)
%   2 = en haut à gauche (90°)
%   3 = en bas à gauche (180°)
%   4 = en bas à droite (270°)
% Exemple: \angleDroit[2]{5,1}{0.3}
\newcommand{\angleDroit}[3][1]{
    \ifcase#1\relax
        % Cas 0 (ne devrait pas arriver)
        \draw (#2) -- ++(0:#3) -- ++(90:#3) -- ++(180:#3);
    \or
        % Cas 1: en haut à droite (0°)
        \draw (#2) -- ++(0:#3) -- ++(90:#3) -- ++(180:#3);
    \or
        % Cas 2: en haut à gauche (90°)
        \draw (#2) -- ++(90:#3) -- ++(180:#3) -- ++(270:#3);
    \or
        % Cas 3: en bas à gauche (180°)
        \draw (#2) -- ++(180:#3) -- ++(270:#3) -- ++(0:#3);
    \or
        % Cas 4: en bas à droite (270°)
        \draw (#2) -- ++(270:#3) -- ++(0:#3) -- ++(90:#3);
    \fi
}
