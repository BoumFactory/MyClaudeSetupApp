\begin{EXO}{Minimum et maximum de fonctions}{}
\tcbitempoint{4} Dire pour chaque fonction si elle admet un minimum ou un maximum et en quelle valeur il est atteint.

\begin{tcbenumerate}[4]
\tcbitem $f(x) = 3x^2+4$

\begin{crep}
Minimum en $x=0$,

vaut $f(0)=4$
\end{crep}

\tcbitem $g(x)= -2(x-4)^2+8$

\begin{crep}
Maximum en 

$x=4$, vaut $g(4)=8$
\end{crep}

\tcbitem $h(x)= -2x^2+8x-1$

\begin{crep}
Maximum en 

$x=2$, vaut $h(2)=7$
\end{crep}

\tcbitem $k(x)= 7(x+1)^2-25$

\begin{crep}
Minimum en $x=-1$, 

vaut $k(-1)=-25$
\end{crep}
\end{tcbenumerate}

\exocorrection

\begin{tcbenumerate}[2]
\tcbitem $f(x) = 3x^2+4 = 3(x-0)^2+4$

Forme canonique : $a=3>0$, donc minimum.

Minimum atteint en $x=0$ avec $f(0)=4$.

\tcbitem $g(x)= -2(x-4)^2+8$

Forme canonique : $a=-2<0$, donc maximum.

Maximum atteint en $x=4$ avec $g(4)=8$.

\tcbitem $h(x)= -2x^2+8x-1$

Forme canonique : $\alpha = -\dfrac{8}{2 \times (-2)} = 2$

$\beta = h(2) = -2 \times 4 + 8 \times 2 - 1 = 7$

Donc $h(x) = -2(x-2)^2+7$ avec $a=-2<0$.

Maximum atteint en $x=2$ avec $h(2)=7$.

\tcbitem $k(x)= 7(x+1)^2-25$

Forme canonique : $a=7>0$, donc minimum.

Minimum atteint en $x=-1$ avec $k(-1)=-25$.
\end{tcbenumerate}
\end{EXO}